\chapter{\abstractname}
\begin{itemize}
	\item Good quality \textit{h}-BN islands require a clean and flat surface to grow on - so techniques to chemically polish the surface are shortly reviewed. After the etching process the growth of sub-monolayer \textit{h}-BN islands is investigated on poly crystalline copper surfaces by means of STM and XPS. Their use as insulation decoupling substrate is shown by reproducing molecular adsorption known on \textit{h}-BN grown on single crystalline copper.
	\item Different molecular species are investigated with regard to their electronic properties and structures formed by self-assembly on metallic and insulating 2D \textit{h}-BN substrates.
	\item Bis- \& Tetra-pyridin4-ylethynyl functionalized pyrene molcules are adsorped on \textit{h}-BN/Cu(111)  to show the diversity of self-assembled molecules structures that can be steered by the number and position of functional groups. It is shown that their opto-electronic properties and assembly after adsorption are well determined by the chemical design of the molecule and show the same trend as gas-phase calculations: An decrease in electronic band gap with increasing number of functional sites. Frontier orbital resolution in STM with modified tip conditions and a wide band gap in STS show efficient decoupling from the metallic substrate by the \textit{h}-BN layer that modifies the substrate surface potential. As a result, the band gap can be changed locally in a well defined periodic manner, following the periodicity of \textit{h}-BN's superstructure on Cu(111).
	\item Furthermore coronene molecules are used to determine the influence of a BN modified molecular center on the self-assembly and electronic structure on Ag(111) and Au(111) substrates and shows the importance of the side groups. Investigation with nc-AFM is used to clarify sub-molecular structure and the formation of linked structures after annealing treatment in UHV.
	\item Single and bis- nitro functionalized porphine molecules are adsorbed on Cu(111), Ag(111) and \textit{h}-BN/Cu(111). Additional di-tert-butyl-phenyl side groups are used to further decouple the molecule from the substrate layer and increase mobility on \textit{h}-BN at low temperatures. The molecular self-assembly is controlled by the number of functional groups and leads to the formation of superstructures with hexagonal symmetry formed on a metallic substrate, mismatching its 2-fold symmetric (100) surface symmetry.
	\item Helicene molecules are used to investigate the influence of chiral properties in conjunction with a permanent dipole moment introduced by functinoalization with two cyano groups at the helicene's central carbon ring. Depending on the substrate, molecular assemblies varies from chains formed parallel to the metal substrate's high symmetry directions. Annealing resulted in a ring-closure reaction at the helicene's spiral terminations and lifts chirality. Dense packed islands are formed after adsroptino on \textit{h}-BN/Cu(111).
\end{itemize}
