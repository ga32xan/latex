\usepackage[utf8]{inputenc}
% when latex complains about unicode char U+2212 is not configured for use in latex use the line below
\DeclareUnicodeCharacter{2212}{-}% support older LaTeX versions
\DeclareUnicodeCharacter{2010}{-}% support older LaTeX versions
%\usepackage[latin1]{inputenc}
\usepackage[T1]{fontenc}
\usepackage{lmodern}
\usepackage[english]{babel}	%loads AMERICAN english, same as american USenglish
\usepackage{csquotes}
\usepackage{amsmath}
\usepackage{textcomp}		%enables \textdegree to use as �
\usepackage{amsfonts}
\usepackage{amssymb}
\usepackage{graphicx}
%%%%%%%%%%%%%%%%%%%%%%%%%%%%%
\usepackage[onehalfspacing]{setspace}
%%%%%%%%%%%%%%%%%%%%%%%%%%%%%
\usepackage{rotating}
% Sidewaystable environment for appendix
%%%%%%%%%%%%%%%%%%%%%%%%%%%%%
\usepackage{listings}
\usepackage{pdflscape}
%%%%%%%%%%%%%%%%%%%%%%%%%%%%%%%%%%%%%%%%%%%%%%%%%%%%%%%
\usepackage{xcolor}
%%%%%%%%%%%%%%%%%%%%%%%%%%%%%%%%%%%%%%%%%%%%%%%%%%%%%%%
% Do not place graphics and tables on the bottom float of a page if footnotes are present!
\toks0{\ifvoid\footins\else\suppressfloats[b]\fi}
\output\expandafter{\the\toks0\the\output}
% https://tex.stackexchange.com/questions/436004/avoid-bottom-floats-on-pages-with-footnotes
%%%%%%%%%%%%%%%%%%%%%%%%%%%%%%%%%%%%%%%%%%%%%%%%%%%%%%%
%%%%%%%%%%%%%%%%%%%%%%%%%%%%%%%%%%%%%%%%%%%%%%%%%%%%%%%
\definecolor{deepblue}{rgb}{0,0,0.5}
\definecolor{deepred}{rgb}{0.6,0,0}
\definecolor{deepgreen}{rgb}{0,0.5,0}
% Check commands for further commands in listing environment (python/appenidx)
% Sidewaystable environment for appendix
%%%%%%%%%%%%%%%%%%%%%%%%%%%%%
\usepackage{xhfill}			% provides /hrulefill (disclaimer)
%\usepackage{wasysym}		
\usepackage{braket}		% fur <A|H|B> <A| |A> oder <A>
%\DeclareGraphicsExtensions{.pdf,.png,.jpg}
%%%%%%%%%%%%%%%%%%%%%%%%%%%%%
\usepackage{siunitx}
\DeclareSIUnit\langmuir{L}
%%%%%%%%%%%%%%%%%%%%%%%%%%%%%
\usepackage[hidelinks,breaklinks=true]{hyperref}
%\usepackage{url}
\usepackage[section]{placeins} %definiert \floatbarrier, mit option automatisch bei jeder section
%%%%%%%%%%%%%%%%%%%%%%%%%%%%%%%%%%%%%%%%%%%

\newcommand{\subfigureautorefname}{\figureautorefname}
%There is nothing like a subfigure autoref name, so define one. Figures are cited with \autoref{label} and preceeded with a "Figure" in-text.,
\usepackage{wrapfig}
%%%%%%%%%%%%%%%%%%%%%%%%%%%%%%%%%%%%%%%%%%%
% Updated definition, see explanation below
\newcommand*{\fullref}[1]{\hyperref[{#1}]{\autoref*{#1} \nameref*{#1}}} % One single link
% see https://tex.stackexchange.com/questions/121865/nameref-how-to-display-section-name-and-its-number

\usepackage{microtype}

\usepackage[format=plain,labelsep=endash,font=it,labelfont=bf]{caption}			% Change figure captions
\usepackage{subfigure}					%replaced by subfig?!
% Use subfigure to do (a), (b) and such graphics.
% will apply to all captions

%%%%%%%%% Change Text type of autorefencing figures for english language
\renewcaptionname{english}\figureautorefname{\texttt{Figure}}
%%%%%%%%%%%%%%%%%%%%%%%%%%%%%%%%%%%%%%%%%%%%%
% will apply to all subcaptions
%\usepackage[labelfont=bf,font=it,singlelinecheck=off,justification=raggedright]{subcaption}
% won't work in conjunction with subfigure!!!
%%%%%%%%%%%%%%%%%%%%%%%%%%%%%%%%%%%%%%%%%%%%%
\usepackage{multicol}
\usepackage{multirow}
%%%%%%%%%%%%%%%%%%%%%%%%%%%%
% fuer Stichwortverzeichnis
\usepackage{makeidx}
% Stichwortverzeichnis erstellen
\makeindex
%%%%%%%%%%%%%%%%%%%%%%%%%%%%%%%%%%%%%%%%%%%%%%%%%%%%%%%
\usepackage[style=numeric-comp		% bibliogryphy-styles: alphabetic, numeric, chem-angew, ieee, nature, science, numeric-comp
,backend=biber
%,refsection=chapter			% setzt bibliographies nach chaptern getrennt, nach jedem chapter muss ein 
								% printbibliogrphy stehen
,sorting=none					% Der Reihenfolge im Dokument nach ordnen
]{biblatex} 	
\addbibresource{./bib.bib}  	% relative to root directory (where the file that includes this file is located)! 
								%do NOT OMIT .bib ending

%avoids ugly line breaks within bibligraphy
\addto\bibsetup{\setlength{\emergencystretch}{1.5em}} 	
%%%%%%%%%%%%%% COLORED BOXES IN APPENDIX %%%%%%%%%%%%%
\usepackage{tcolorbox}% http://ctan.org/pkg/tcolorbox
\makeatletter					% WHY?!
\newcommand{\mybox}[1]{%
	\setbox0=\hbox{#1}%
	\setlength{\@tempdima}{\dimexpr\wd0+13pt}%
	\begin{tcolorbox}[colframe=deepblue,boxrule=0.5pt,arc=4pt,
		left=6pt,right=6pt,top=6pt,bottom=6pt,boxsep=0pt,width=\@tempdima]
		#1
	\end{tcolorbox}
}
\makeatother
%%%%%%%%%%%%%%%%%%%%%%%%%%%%%%%%%%%%%%%%%%%%%%%%%%%%%%%
% Zum Verwalten der Zitate benutze ich Zotero, zum Erzeugen der .bib-Datein f�r Latex wird die Exportfunktion von Zotero benutzt (rechtsklick auf ``Meine Bibliothek'' im linken Reiter: Option Biblatex in die Datei bib-zotero-export.bib aus welcher ich dann die betreffenden Zitate auf Richtigkeit \"uberpr\"ufe und in die bib.bib kopiere.
%%%%%%%%%%%%%%%%%%%%%%%%%%%%%%%%%%%%%%%%%%%%%%%%%%%%%%
\usepackage[draft=false]{scrlayer-scrpage}		%deaktiviert ruler in der draft version
\pagestyle{scrheadings}
%%%%%%%%%%%%%%%%%%%%%%%%%%%%%%%%%%%%%%%%%%%%%%%%%%%%%%
%%%%%%%%%%%%%%%%% DAUMENKINO Footer %%%%%%%%%%%%%%%%%%
%	\usepackage{etex}
%	\usepackage{intcalc} 
%	\newcommand*{\AnzBilder}{200}		            		%<--Variablen anpassen
\newcommand*{\KinoPfad}{./images/animation/lumo/} 	%<--Variablen anpassen

%%%%Quelltext%%%
\newcommand*{\SafeboxName}{sbKino}

\makeatletter
%Erzeugt neue Saveboxen und füllt sie mit includegraphics-Anweisungen
%Aufruf: \NewSaveBoxes{sbKino}{5}{daumenkino/kino}
\newcommand*{\NewSaveBoxes}[3]{%
	\@tempcnta 1
	\@whilenum \@tempcnta< \numexpr(#2+1) \do{%
		%Savebox anlegen
		\expandafter\newsavebox\csname #1\the\@tempcnta\endcsname
		%Savebox mit Leben füllen
		\expandafter\savebox\csname #1\the\@tempcnta\endcsname{%
			\includegraphics[width=0.5cm]{#3\the\@tempcnta}%
		}%
		\advance\@tempcnta 1
	}%
}

\newcommand*{\bildnr}{\numexpr\intcalcMod{\numexpr\value{page}}{\numexpr\AnzBilder}\relax}
\newcommand*{\lumoseries}{%
	\usebox{\@nameuse{\SafeboxName\the\bildnr}}%
}
\makeatother
\NewSaveBoxes{\SafeboxName}{\AnzBilder}{\KinoPfad}
%	%inner side of odd pages
%	\lofoot[\lumoseries]{\lumoseries} % Add flicker books to [plain.scrheadins] and {scrheadins}!
%	% inner side of even pages	
%	\newcommand*{\AnzBilderLogo}{200}		            		%<--Variablen anpassen
\newcommand*{\KinoPfadLogo}{./images/animation/logo/} 	%<--Variablen anpassen
%%%%Quelltext%%%
\newcommand*{\SafeboxNameLogo}{sbKinologo}

\makeatletter
%Erzeugt neue Saveboxen und füllt sie mit includegraphics-Anweisungen
%Aufruf: \NewSaveBoxesLogo{sbKino}{5}{daumenkino/kino}
\newcommand*{\NewSaveBoxesLogo}[3]{%
	\@tempcntb 1
	\@whilenum \@tempcntb< \numexpr(#2+1) \do{%
		%Savebox anlegen
		\expandafter\newsavebox\csname #1\the\@tempcntb\endcsname
		%Savebox mit Leben füllen
		\expandafter\savebox\csname #1\the\@tempcntb\endcsname{%
			\includegraphics[width=0.5cm]{#3\the\@tempcntb}%
		}%
		\advance\@tempcntb 1
	}%
}
%intcalc-version
\newcommand*{\bildnrLogo}{\numexpr\intcalcMod{\numexpr\value{page}}{\numexpr\AnzBilderLogo}\relax}

\newcommand*{\logoseries}{%
	\usebox{\@nameuse{\SafeboxNameLogo\the\bildnrLogo}}%
}
\makeatother
\NewSaveBoxesLogo{\SafeboxNameLogo}{\AnzBilderLogo}{\KinoPfadLogo}
%%%Aufruf%%%%%%%
%	\refoot[\logoseries]{\logoseries} % Add flicker books to [plain.scrheadins] and {scrheadins}!
%%%%%%%%%%%%%%%%%%%%%%%%%%%%%%%%%%%%%%%%%%%%%%%%%%%%%%
%%%%%%%%%%%%%%%%%%%%%%%%%%%%%%%%%%%%%%%%%%%%%%%%%%%%%%