Here the setup of the peltier-cooling unit is described. It is used for cooled  storage of borazine, a liquid precursor used to grow \textit{h}-BN via CVD.

The liquid containing glass tube is surrounded by a hollow cylinder made of copper that is cooled via the cold side of a peltier element at the bottom. The hot side is connected to a thermal mass which is held at RT by a CPU socket fan (LGA775). For thermal insulation to the environment, the cold hollow cylinder is surrounded by a acrylic glass. When possible, the volume between copper and acrylic cylinder should be sealed to prevent water vapor to condense and freeze. 

When used in the setup shown below, operating temperatures down to \SI{-5}{\celsius} are achievable with ambient temperatures around \SI{20}{\celsius}. If lower temperatures are needed, one may either increase the power of the peltier element (either by increasing the single element power or stacking several ones) or decrease the temperature of the hot side below RT (by water-cooling).

Peltier elements should be sealed to prevent moisture uptake that would quickly degrade the elements or shortcut the contacts.