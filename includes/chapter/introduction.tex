Science strives to achieve more and more detailed information on matter at the smallest length scales. Understanding interaction between atoms and molecules resulted in the development of applications in sizes that would never be reached by engineers without applying the knowledge gained by foundational research. These applications are used in our every day life, often behind the scenes without need of the consumer knowing the physical details (LED lights, self-cleaning window glasses, gas sensors to detect explosives or harmful gas concentrations with very low detection levels (down to low ppm concentrations)).

While atomically precise structures were created by single atom manipulation \cite{Crommie_confinement_1993} this process is not applicable at larger scales.
Here the forces between adsorbates and between adsorbate and substrate is used to form regular structures. Inter-molecular links are specified by the chemical design of the molecule and result in structures that assemble themselves without the need to manipulate every single molecule with the STM tip.

Metal substrates interact with the molecules and influence electronic properties and molecular assembly. An investigation of these properties can be decoupled from the substrate by introducing a insulating layer to adsorb the molecules on.
Development of on surface synthesis on metal substrates lead to 2D-materials like \textit{h}-BN that show wide band gap insulating properties. This is used here to electronically decouple the molecule from the metal substrate.

Molecular chemistry evolved to a point where molecules can be designed on the drawing table and synthesized with demanded properties so that assembly at atomic level is predetermined by synthesis. 

This work is placed at the interface between chemistry and computational physics to provide experimental insights into molecular interactions present at atomic levels that are used to benchmark theoretical calculations.