\documentclass[
10pt,					% Schriftgröße
a4paper,				% A4 paper
twoside,				% Zweiseitig
BCOR=8mm,				% Bindekorrektur inkl. Biegefalz
headings=normal,		% Kleinere Kapitelüberschriften => check preamble
headsepline,			% Enable line to seperate head ...
footsepline,			% ... and foot
plainfootsepline,		% Footseperation line on chapter start
%,draft
]{scrbook}

\usepackage[
top=30mm,
bottom=55mm,
inner=25mm,
outer=30mm,
%marginparsep=7mm,
%marginparwidth=48mm,
%paperwidth=210mm,
%paperheight=245mm, 286mm (A4)
]{geometry}

%%%%%%%%%%%%%%%%%%%%%%%%%%%%%%%%%%%%%%%%%%%%%%%%%%%%%%%
\author{Domenik Matthias Zimmermann}
\title{Molecular fictionalization of h-BN}
%%%%%%%%%%%%%%%%%%%%%%%%%%%%%%%%%%%%%%%%%%%%%%%%%%%%%%%
\usepackage[utf8]{inputenc}
% when latex complains about unicode char U+2212 is not configured for use in latex use the line below
\DeclareUnicodeCharacter{2212}{-}% support older LaTeX versions
%\usepackage[latin1]{inputenc}
\usepackage[T1]{fontenc}
\usepackage{lmodern}
\usepackage[english]{babel}
\usepackage{csquotes}
\usepackage{amsmath}
\usepackage{textcomp}		%enables \textdegree to use as �
\usepackage{amsfonts}
\usepackage{amssymb}
\usepackage{graphicx}
\usepackage{xhfill}			% provides /hrulefill (disclaimer)
%\usepackage{wasysym}		
\usepackage{braket}		% fur <A|H|B> <A| |A> oder <A>
%\DeclareGraphicsExtensions{.pdf,.png,.jpg}
%%%%%%%%%%%%%%%%%%%%%%%%%%%%%
\usepackage{siunitx}
\DeclareSIUnit\langmuir{L}
%%%%%%%%%%%%%%%%%%%%%%%%%%%%%
\usepackage[hidelinks,breaklinks=true]{hyperref}
%\usepackage{url}
\usepackage[section]{placeins} %definiert \floatbarrier, mit option automatisch bei jeder section
%%%%%%%%%%%%%%%%%%%%%%%%%%%%%%%%%%%%%%%%%%%
\usepackage{subfigure}
\usepackage{wrapfig}
%%%%%%%%%%%%%%%%%%%%%%%%%%%%%%%%%%%%%%%%%%%
\usepackage{caption}
\usepackage{microtype}
%\usepackage{subcaption}
\usepackage{multicol}
\usepackage{multirow}
%%%%%%%%%%%%%%%%%%%%%%%%%%%%
% fuer Stichwortverzeichnis
\usepackage{makeidx}
% Stichwortverzeichnis erstellen
\makeindex
%%%%%%%%%%%%%%%%%%%%%%%%%%%%%%%%%%%%%%%%%%%%%%%%%%%%%%%
\usepackage[style=numeric		% bibliogryphy-styles: alphabetic, numeric, chem-angew, ieee, nature, science
,backend=biber
%,refsection=chapter			% setzt bibliographies nach chaptern getrennt, nach jedem chapter muss ein 
								% printbibliogrphy stehen
]{biblatex} 	
\addbibresource{./bib.bib}  	% relative to root directory (where the file that includes this file is located)! 
								%do NOT OMIT .bib ending

%avoids ugly line breaks within bibligraphy
\addto\bibsetup{\setlength{\emergencystretch}{1.5em}} 	

% Zum Verwalten der Zitate benutze ich Zotero, zum Erzeugen der .bib-Datein f�r Latex wird die Exportfunktion von Zotero benutzt (rechtsklick auf ``Meine Bibliothek'' im linken Reiter: Option Biblatex in die Datei bib-zotero-export.bib aus welcher ich dann die betreffenden Zitate auf Richtigkeit \"uberpr\"ufe und in die bib.bib kopiere.
%%%%%%%%%%%%%%%%%%%%%%%%%%%%%%%%%%%%%%%%%%%%%%%%%%%%%%%
% \usepackage[top=2.5cm,left=2.5cm,right=3.5cm,bottom=3.5cm]{geometry}
%%%%%%%%%%%%%%%%%%%%%%%%%%%%%%%%%%%%%%%%%%%%%%%%%%%%%%%
\usepackage{xcolor}
%%%%%%%%%%%%%%%%%%%%%%%%%%%%%%%%%%%%%%%%%%%%%%%%%%%%%%
%%%%%%%%%%%%%%%%%%%%%%%%%%%%%%%%%%%%%%%%%%%%%%%%%%%%%%
\usepackage[draft=false]{scrlayer-scrpage}		%deaktiviert ruler in der draft version
\pagestyle{scrheadings}
%%%%%%%%%%%%%%%%%%%%%%%%%%%%%%%%%%%%%%%%%%%%%%%%%%%%%%
\ifdefined\daumenkino
	\usepackage{etex}
	\usepackage{intcalc} 
	\newcommand*{\AnzBilder}{200}		            		%<--Variablen anpassen
\newcommand*{\KinoPfad}{./images/animation/lumo/} 	%<--Variablen anpassen

%%%%Quelltext%%%
\newcommand*{\SafeboxName}{sbKino}

\makeatletter
%Erzeugt neue Saveboxen und füllt sie mit includegraphics-Anweisungen
%Aufruf: \NewSaveBoxes{sbKino}{5}{daumenkino/kino}
\newcommand*{\NewSaveBoxes}[3]{%
	\@tempcnta 1
	\@whilenum \@tempcnta< \numexpr(#2+1) \do{%
		%Savebox anlegen
		\expandafter\newsavebox\csname #1\the\@tempcnta\endcsname
		%Savebox mit Leben füllen
		\expandafter\savebox\csname #1\the\@tempcnta\endcsname{%
			\includegraphics[width=0.5cm]{#3\the\@tempcnta}%
		}%
		\advance\@tempcnta 1
	}%
}

\newcommand*{\bildnr}{\numexpr\intcalcMod{\numexpr\value{page}}{\numexpr\AnzBilder}\relax}
\newcommand*{\lumoseries}{%
	\usebox{\@nameuse{\SafeboxName\the\bildnr}}%
}
\makeatother
\NewSaveBoxes{\SafeboxName}{\AnzBilder}{\KinoPfad}
	\lofoot{\lumoseries} %<-- Eigentlicher Aufruf für Fußzeile
	
	\newcommand*{\AnzBilderLogo}{200}		            		%<--Variablen anpassen
\newcommand*{\KinoPfadLogo}{./images/animation/logo/} 	%<--Variablen anpassen
%%%%Quelltext%%%
\newcommand*{\SafeboxNameLogo}{sbKinologo}

\makeatletter
%Erzeugt neue Saveboxen und füllt sie mit includegraphics-Anweisungen
%Aufruf: \NewSaveBoxesLogo{sbKino}{5}{daumenkino/kino}
\newcommand*{\NewSaveBoxesLogo}[3]{%
	\@tempcntb 1
	\@whilenum \@tempcntb< \numexpr(#2+1) \do{%
		%Savebox anlegen
		\expandafter\newsavebox\csname #1\the\@tempcntb\endcsname
		%Savebox mit Leben füllen
		\expandafter\savebox\csname #1\the\@tempcntb\endcsname{%
			\includegraphics[width=0.5cm]{#3\the\@tempcntb}%
		}%
		\advance\@tempcntb 1
	}%
}
%intcalc-version
\newcommand*{\bildnrLogo}{\numexpr\intcalcMod{\numexpr\value{page}}{\numexpr\AnzBilderLogo}\relax}

\newcommand*{\logoseries}{%
	\usebox{\@nameuse{\SafeboxNameLogo\the\bildnrLogo}}%
}
\makeatother
\NewSaveBoxesLogo{\SafeboxNameLogo}{\AnzBilderLogo}{\KinoPfadLogo}
%%%Aufruf%%%%%%%
	\refoot{\logoseries} %<-- Eigentlicher Aufruf für Fußzeile
\fi
%%%%%%%%%%%%%%%%%%%%%%%%%%%%%%%%%%%%%%%%%%%%%%%%%%%%%%
%%%%%%%%%%%%%%%%%%%%%%%%%%%%%%%%%%%%%%%%%%%%%%%%%%%%%%
% Basic information for cover & title page
\newcommand*{\getUniversity}{Technische Universit\"at M\"unchen}
\newcommand*{\getFaculty}{Department of physics}
\newcommand*{\getFacultyger}{Fakult\"at f\"ur Physik}
\newcommand*{\getTitle}{Molecular adsorption on \textit{h}-BN}
%newcommand*{\getTitleger}{TODO: Titel der Abschlussarbeit}
\newcommand*{\getAuthor}{Domenik Matthias Zimmermann}
\newcommand*{\getDoctype}{Dissertation}
\newcommand*{\getDoctypeger}{Vollst\"andiger Abdruck der von der Fakult\"at für Physik der Technischen Universit\"at M\"unchen zur Erlangung des akademischen Grades eines Doktors der Naturwissenschaften (Dr. rer. nat.) genehmigten Dissertation.}
\newcommand*{\getSupervisor}{Prof.\ Dr.\ Wilhelm Auw\"arter}
\newcommand*{\getChairman}{TODO: Chairman}
\newcommand*{\getFirstExaminer}{TODO: 1. Examiner}
\newcommand*{\getSecondExaminer}{TODO: 2. Examiner}
\newcommand*{\getSubmissionDate}{TODO: Submission date}
\newcommand*{\getSubmissionLocation}{Munich}
\newcommand*{\getDisclaimer}{I assure the single handed composition of this \MakeLowercase{\getDoctype{}} only supported by declared resources.}
% TODO: add custom commands etc.

%%%%%%%%%%%%%%%%%%%%%%%%%%%%%%%%%%%%%%%%%%%%%%%%%%%%%%%%%%%%%%%%%%%%%%%%%%%%%%%%%%%%%%%%%%%%%
% Change \autoref (Figure x.x) to FIGURE x.x

%%%%%%%%%%%%%%%%%%%%%%%%%%%%%%%%%%%%%%%%%%%%%%%%%%%%%%%%%%%%%%%%%%%%%%%%%%%%%%%%%%%%%%%%%%%%%
% Python style for highlighting
% Check https://tex.stackexchange.com/questions/83882/how-to-highlight-python-syntax-in-latex-listings-lstinputlistings-command#83883

% Default fixed font does not support bold face
\DeclareFixedFont{\ttb}{T1}{txtt}{bx}{n}{12} % for bold
\DeclareFixedFont{\ttm}{T1}{txtt}{m}{n}{12}  % for normal

\newcommand\pythonstyle{\lstset{
		language=Python,
		basicstyle=\small,             %\ttm, \tiny \small
		otherkeywords={self},             % Add keywords here
		keywordstyle=\small\color{deepblue},
		emph={MyClass,__init__},          % Custom highlighting
		emphstyle=\small\color{deepred},    % Custom highlighting style
		stringstyle=\color{deepgreen},
		frame=tb,                         % Any extra options here
		showstringspaces=false,           % 
		frame=single,
		numbers = left,
		numbersep=5pt
}}
% Python environment
\lstnewenvironment{python}[1][]
{
	\pythonstyle
	\lstset{#1}
}
{}
% Python for external files
\newcommand\pythonexternal[2][]{{
		\pythonstyle
		\lstinputlisting[#1]{#2}}}
% Python for inline
\newcommand\pythoninline[1]{{\pythonstyle\lstinline!#1!}}
%%%%%%%%%%%%%%%%%%%%%%%%%%%%%%%%%%%%%%%%%%%%%%%%%%%%%%%%%%%%%%%%%%%%%%%%%%%%%%%%%%%%%%%%%%%%%
%%%%%%%%% Appendix name %%%%%%%%%%%
\newcommand*{\Appendixautorefname}{Appendix}



\begin{document}
The tunneling effect describes the behavior of a particle that faces a potential barrier. In the classical picture the particle is prohibited to move across the potential barrier if its energy is lower than the barrier. In quantum mechanics, a wave character is added to the particle. When the wave now faces the potential barrier, a fraction of the wave package that constitutes the particle is transmitted through the barrier - an effect known as tunneling.

\subsection{Overview...historically}
The tunneling effect was first observed by Hund in 1926 in molecules, where he explained the sharing of an electron between atoms, each represented by a potential well.\cite{Mehra_tunneling_1982} A principle fundamental for an understanding of covalent chemical bonds. 

The first quantitative expression of the tunneling current in a metal-insulator-metal junction was introduced by Bardeen in 1961 \cite{Bardeen_tunneling_1961}. \textcolor{red}{\textbf{MORE}}

This lead to an early prototype build by Russell Young, John Ward and Fredric Scire in 1972. \cite{Young_topographiner_1972}. Here the basic principles are shown already (piezo actuators, tunneling with metal tip). The concept was further improved by Binning \& Rohrer in 1981 \cite{binning_tunneling_1982} when they were the first to report experimental evidence for the tunneling through an vaccum gap with controllable width. They showcased the excelled resolution capabilities by resolving the ($7 \times 7$) reconstruction of the Si(111) surface \cite{binnig_1983}. They received the noble prize in 1986 \cite{_noble_price_1986} for "their design of the scanning tunneling microscope" (STM). 

In the following the tunneling process through a vacuum gap between metallic tip and sample will be summarized and used as model system to describe the basic concept of STM.

\subsection{Theory...on 1D tunneling at a single point}
\index{STM!One dimensional tunneling}

\begin{figure}[]\centering
	\includegraphics[width=0.7\textwidth]{./images/tunnel-barrier}
	\caption{Energy diagram to visualize the tunneling process between sample (left) and tip (right) separated by a distance d. Work functions of sample and tip ($\Phi_s$ and $\Phi_t$) separate the filled states (shaded regions) and the vacuum level ($\epsilon_{vac}$). Since sample ($\rho_s$) and tip DOS ($\rho_t$) may not be uniform, a fictional DOS is sketched in darker colors between both. The samples energy is lifted by $eV$ after a bias is applied and results in a net electron current from the sample into the tip. One tunneling process is indicated by a wave function in the sample. After overcoming the vacuum barrier its amplitude decreases and the corresponding electron occupies a free state (not shaded) in the tip material.  Taken from \cite{diss-schunack}}
	\label{fig:STM-barrier}
\end{figure}

\textcolor{red}{\textbf{Consider a system with two metals separated by a vacuum gap. Describe more!}}
While the tip is far away from the sample (both metallic), their vacuum levels are the same. The corresponding Fermi energies of sample and tip lie below the vacuum level by the amount of their work functions ($\Phi_s$ and $\Phi_t$ for sample and tip respectively). Wave functions of electrons within the tip and sample decay exponentially in vacuum, depended on their energy with respect to the Fermi level.
If sample and tip are in thermodynamic equilibrium, their Fermi levels are the same. Electrons now face a potential barrier (approximately rectangular) which can be overcome if their energy is high enough and the barrier sufficiently narrow. When a voltage is applied across the tunneling barrier, the energy of the tip-electrons is shifted by $eV$ as illustrated in \autoref{fig:STM-barrier}. When a positive bias voltage is applied, electrons tunnel from the tip into unoccupied states in the sample - a negative bias results in a tunneling current in opposite direction. 

Following the model of \index{STM!Tersoff-Hamann} 
Tersoff-Hamann\footnote{Please's note that there are more models and corrections to them. An evolution from Bardeen's approach to the one done by Tersoff-Hamann can be found here \cite{lounis_theory_2014, wortmann_interpretation_2000} including Chen's expansion.}((1) uniform density of states in the tip, (2) temperature is low, (3) small bias voltage of some mV, (4) waveform of electrons in tip are s-waves) the tunneling current measured at the center of the curvature of the s-wave like tip results to $$I=32\pi^3\hbar^{-1}e^2V\Phi_t^2 R^2\kappa^{-4}e^{-2\kappa R}\rho_t(E_F)\rho_s(r_o,E_F)$$ Here the current is described in relation to $\rho_t$ the density of states per unit volume of the tip, R the tip radius and $\rho_s(r_0,E_F)$ the Fermi level density of states in the sample\cite{bonnell_scanning_1993}. The distance between tip and sample is denoted as \textcolor{red}{\textbf{$Z$}} and the inverse decay length of the electrons wave function is $\kappa=\frac{\sqrt{2m\Phi_t}}{\hbar}$. If $I$ is held constant one can see that the tip in principle follows a contour of constant Fermi level density of states at the sample surface. While its a good first approximation of the system, in many cases the bias is in the range of (\SIrange{1}{5}{\V}) so more than just the electrons near Fermi contribute. Also a uniform $\rho_t$ may not be accurate in all cases.

Using \index{STM!WKB} Wentzel-Kramers-Brillouin (WKB) theory\cite{wentzel_verallgemeinerung_1926, kramers_wellenmechanik_1926, brillouin_mecanique_1926} the tunneling current is given by
\begin{equation}
I=\int_0^{eV}\rho_s(r,E)\rho_t(r,eV+E)T(E,eV,r)dE
\label{WKB}
\end{equation}
where $\rho_s(\rho_t)$ is the density of states of the sample (tip) and T is the tunneling transmission probability
\begin{equation}
T(E,eV)=exp\left(-\frac{\textcolor{red}{\textbf{2}}Z\sqrt{2m}}{\hbar}\sqrt{\frac{\Phi_s+\Phi_t}{2}+\frac{eV}{2}-E}\right)
\label{Transmission-function} 
\end{equation}
describing the propability of an tunneling event between tip and sample.

If $eV<0$ the tunneling current is largest for $E=0$ (electrons on the Fermi-level of the sample), if $eV>0$ the tunneling current is largest for $E=eV$ (electrons of Fermi level in tip).

Since states with highest energy have the largest decay lengths in vacuum, most of the tunneling current is determined by electrons within close proximity to the Fermi level.\footnote{More information related to tunneling processes can be found here \cite{bonnell_scanning_1993}.}

Due to the fact that the tunneling current is proportional the density of states in the tip and the molecule one can deduce the band structure within a range of several volts in the vicinity of the Fermi energy.

Tunneling current between tip and sample depends on the LDOS of tip and surface and is therefore not implicitly maximized at the atomic positions. It may also vary with the bias voltage applied in a non-trivial manner. Investigation of this behavior led to the establishment of a new measurement technique, called scanning tunneling spectroscopy (see \autoref{section:STS}). 

\subsection{\textbf{S}canning \textbf{T}unneling \textbf{S}pectroscopy}
\paragraph{Theory...on STS}
\label{section:STS}
First changes of the tunneling current with the bias voltage were observed by Tromp et al. in 1986 \cite{tromp_atomic_1986}. They discovered a change in contrast when scanning a SI(111) surface with either positive or negative bias. The change in contrast is most apparent in semiconductors and semi metals\cite{bonnell_scanning_1993}, but adsorbates and charged areas of the sample change the DOS locally and therefore the contrast in STM. While simple results may be already obtained when comparing two images recorded at different voltages, more detailed information can be achieved. At low temperatures the vanishing lateral movement of molecules makes them also accessible to tunneling spectroscopy with submolecular lateral resolution. It is possible to deduce the electronic configuration with atomic spatial resolution.

\index{STS!Bias below work function}
First let us consider small biases.
If tunneling conditions are such that $eV\leq\Phi$, observed features in $dI/dV$ are associated with the surface DOS. Critical points in the surface projected DOS give rise to features in $dI/dV$. Interpretation of these features with the WKB theory (i.e. differentiating equation \eqref{WKB}) gives
$$dI/dV=\rho_s(r,eV)\rho_t(r,0)T(\textcolor{red}{\textbf{eV}},eV,r)+\int_0^{eV}\rho_s(eV)\rho_t(r,E-eV)\frac{dT(E,eV,r)}{dV}dE$$
The first term contains the DOS of the sample and tip and the transmission function. While it is usually unknown, a closer look to \eqref{Transmission-function} indicates a smooth, monotonically increasing function in V. This mannered dependence on V gives a smooth background described by the second term $\int_0^{eV}\rho_s(eV)\rho_t(r,E-eV)\frac{dT(E,eV,r)}{dV}dE$.
Because T is smooth and monotonic the first term $\rho_s(r,eV)\rho_t(r,0)T(eV,eV,r)$ introduces the dependence on the DOS in the sample for energies $eV$ - our desired spectrum. As for STM topography images, bias voltage can be chosen to either record occupied (-) or unoccupied states (+).

%\subsection{...and how an image is created}
%The current recorded in a certain area of the sample is translated into a contrast variation on a color scale. While some images encourage the operator to interpret points with high intensity as elevated atoms it is not that trivial. The \textbf{constant current mode}\index{STM:operating mode} is the most widely used one. The tip height is regulated with a feedback controller to achieve a constant tunneling current for the chosen bias. The recorded information is now the voltage applied to the z-piezo to maintain a plane with the same current. Sample features that increase the tunneling current cause the feedback to decrease it again by retracting the tip.

\subsection{Experimental details and machine description}

\paragraph{Experimental details}

	\textbf{Topography images} are created by raster scanning the surface pixel by pixel. 
%	They show the voltage aplied to the scan piezo in z-direction to maintain the current set point.
	We use the constant current operating mode where the regulating voltage on the height control piezo is recorded and plotted in an orange color-scale of each pixel. The brighter the color, the more the piezo had to retract the tip to maintain a constant tunneling current. Regions with the same DOS thus appear in the same contrast.

	A modified tip allows for real space imaging of molecular orbitals.

	\textbf{dI/dV} is performed in two ways - on single points (spectrum) and on areas (map). Spectroscopic information can be obtained by either changing the bias voltage and tip height (I(V,z)-spectroscopy) or the tip-sample distance (V(z)-spectroscopy) (I=const).  

	\textbf{Single point spectra} are used to measure electronic properties like molecular orbital energies and electronic band gaps.
	
	\textbf{Spectral maps} at a fixed bias show real space distribution of electronic states at the DOS that corresponds to the chosen bias. The signal intensity represents the differential conductance $\propto$ DOS.
	Therefore the bias is modulated with a sinus like waveform. \index{STS!modulation}The frequency of the low amplitude modulation of the DC bias is much larger than the feedback loop frequency (\SIrange{1}{2}{\kilo \hertz}). The AC part of the tunneling signal is than recorded with a lock in amplifier. \textcolor{red}{The in-phase component is directly the $dI/dV|_{V=V_{bias}}$}, recorded simultaneously with the topography.


\paragraph{Machine description}
Since all used UHV chambers have many common part, a typical setup is described with the LT-STM setup. Here the most experiments were carried out.

The central part of the STM setup is the commercial \textbf{Beetle-type STM scanner} \cite{zoephel_aufbau_2000}. A typical one is shown in \ref{fig:STM}. Here the helicoidal ramp is shown with the central scan piezo and STM tip attached. Three outer piezo tubes are used in slip-stick motion to circularly move on the ramp. Because the ramp is cut with an inclination of \SI{2}{\degree} the circular motion of each piezo results in the STM tip moving up and down. This is used to control the height above the sample during tip approach and lateral movement.

A separate piezo is used to control the lateral position of the tip during scanning. With it the image width, scan speed and tip-sample distance can be controlled on a continuous, sub-atomic length scale.

The measured current is translated into a voltage (I/U converter) and processed in a 20 Bit digital $\rightarrow$ analog (D/A) converter. The current intensity is feed into the DSP Board. Here the STM Softwares current set point is compared with the measured value. The tips position is controlled with an Digital signal processor (\textbf{DSP-Board}) to maintain a constant current contour (used in cc-mode). If the tips position needs to be corrected, a voltage is passed through a HV amplifier and applied to the corresponding piezo element. For dI/dV maps the feedback loop that controls the tip vertical position is not in use, so that the tip maintains an even height. The DSP is used to attenuate high frequency components. \textcolor{red}{\textbf{ More detail?}}

Differentiation of the current signal is done with an \textbf{Lock-In Amplifier}. Here the spectrum is not recorded directly by sweeping the bias and numerically differentiating the measured current. 
	
A sinusoidal modulation on top of the bias voltage with a frequency higher than the low pass frequency of the DSP is used. The modulated bias leads to a tunneling current modulation with the same frequency. The differentiation is performed by reading the AC current signal \textcolor{red}{with a 90° phase shift (as sin and cos are shifted by 90°)}. Because the Lock-In Amplifier only takes signals with the same frequency than the excitation frequency into account, the results are much less suspect to noise. Compared to numerical differentiation, the differentiation with hardware needs less computing effort, too. It is important to note that the DSP does not recognize the bias/current modulation as change in current as topographic feature and regulates as without modulation. If the modulation frequency is too low, the feedback tries to compensate the modulation by changing the distance to the sample. If the modulation frequency is too high, the capacitance between tip and sample leads to an $90\deg$ phase shifted current which increases with modulation frequency. One usually chooses the modulation frequency slightly above the cutoff frequency for the feedback loop.

\paragraph{Vacuum system}
To maintain UHV conditions in the experimental setup, a set of pumps is used.

For the preparation chamber, where high partial pressures occur during sample cleaning and preparation, a combination of roughening and turbo molecular pumps is used. First a roughening pump lowers the atmospheric pressure to \SI{1}{\milli \bar}. With this pressure on the outlet side a turbo molecular pump is used to decrease the pressure even further to the  \SI{1e-8}{\milli \bar} range. The remaining partial pressure is caused by adsorbate covered chamber walls where continuous ad-/desorption takes places and maintains a pressure equilibrium. After heating the entire chamber to temperatures above the evaporation temperature of the adsorbates (mainly water) while constantly pumping, most of the water can be desorped from the walls and is pumped. After cooling down to room temperatures, the pressure settles in the \SI{5e-10}{\milli \bar} regime.

As the pumping efficiency of turbo molecular pumps decreases for low pressures, each chamber is equipped with an ion pump. Here a high voltage \SIrange{1}{7}{\kilo \volt} is applied between two plates. Residual gas particles ionize in such a strong field and are accelerated towards the plates. Here they impinge with high velocity and are buried deep in the plate material that they can't leave. The reduced number of residual gas particles results in a lower pressure.

To further reduce the number of potential contaminations, parts of the chamber can be cooled down with liquid nitrogen. Because of the great temperature gradient, gaseous residuals condense on the much colder surface of the cooling trap and remain adsorped while the temperature is kept low. Without refilling with liquid nitrogen the temperature slowly increases over time, so that the cooling trap looses its pumping efficiency over time (usually within \SIrange{1}{2}{\hour}).

\paragraph{Cooling system} 
While low temperature (LT) STMs may be operated with solely helium, it is more resource-saving to cool the direct proximity of the sample and the STM with He and to suppress the heat flow out of the He-cryostat with a second surrounding nitrogen cryostat (boiling point: \SI{77}{\K}, compare figure \ref{fig:STM-cryo}). This diminishes consumption of globally limited and rather expensive He. To maintain a temperature of \SIrange{5}{7}{\K}, one to two liters of liquid helium are required a day, plus an additional amount of three to four liters liquid nitrogen. Evaporated helium is reclaimed in a closed circuit with a system of purifying and storage/cooling steps so that only a small amount of helium escapes the circuit and is lost.

Sample temperatures down to \SIrange{5}{7}{\K} allow for observations not possible at elevated (room) temperature. Cooling not only reduces thermal drift in the piezo elements that are used to control the tip's position on the sample. Thermal energy at low temperature is not high enough for atoms or molecules to move on most substrates. Species mobile at room temperature (and therefor not representable at room temperature in the sub-ML regime) become immobile and accessible for ST microscopy and spectroscopy. ST spectra resolution is better at low temperatures.

\begin{figure}[ht]\centering
	\subfigure[LT-STM setup mainly used in this work. Different functional groups are colored in different colors. A low base pressure in achieved with a combined pumping system comprised of ion pumps and turbo molecular pumps (cyan). The liquid helium/nitrogen bath cryostat (red) is used to maintain low temperatures. Sample holders are operated with a rote able, variable temperature manipulator (green). Sample preparation is done in the preparation chamber (blue). After transfer to the LT-STM chamber (yellow) a gate valve is used to seal the LT-STM from remaining residual gas that may be present in the preparation chamber. Vibration isolation of the frame is achieved with legs floating on pressurized cylinders (orange).]{
		\includegraphics[width=0.45\textwidth]{./images/chamber-sketch.jpg}
		\label{fig:chamber-sketch}
	} \quad
	\subfigure[Scheme of a liquid bath cryostat. While in the inner stage a temperature of \SIrange{5}{7}{\K} is achieved with a liquid helium reservoir, an outer liquid nitrogen cryostat is used to isolate the inner cryostat from the surrounding room temperature and to reduce the amount of liquid helium used to maintain cryogenic temperatures.]{
		\includegraphics[width=0.45\textwidth]{./images/sketch-cryo.jpg}
		\label{fig:STM-cryo}
	}
	\caption{Typical setup for low temperature measurements. A vibration isolated UHV chamber is used to prepare samples and investigate them in a separable chamber with either STM or AFM. A liquid bath cryostat is used to maintain low temperatures. Images stem from \textcolor{red}{\cite{diss-knud}}}
	\label{fig:STM}
\end{figure}

\paragraph{Piezo elements}
\begin{wrapfigure}{O}{5cm} \centering
	\includegraphics[width=4cm]{./images/STM-sketch-2}
	\caption{STM sample stage to control the tip position. The coarse movement is controlled by exterior piezos. Each moves up/down on a heliocoidal ramp with slip-stick motion. The precise positioning is done with a central scan piezo to which the tip is attached. Taken from}
	\label{fig:stm-heliocoidal ramp}
\end{wrapfigure}
The position of the tip (x, y, z) and macroscopic movement of the stage is controlled with a set of piezos (see \textcolor{red}{\autoref{fig:STM-tip}}). In this work a central tubular piezo is used to control the tips position. The piezo length can be controlled with the voltage applied to them, which is used to choose not only the tip-sample distance, but other parameters like image size and scan speed as well. A feedback loop controls the piezo voltages. For recording an image the area is raster scanned in consecutive lines, applying a sawtooth voltage to the fast scan direction. The next lines are chosen by slowly increasing the voltage along the slow scan direction. Depending on the operating mode the tip-sample distance is controlled by piezo elements, too.

%\begin{figure}[ht]
%	\begin{center}
%	\includegraphics[width=0.45\textwidth]{./images/STM-sketch}
%	\end{center}
%	\caption{Taken from \cite{diss-manuela}}
%\end{figure}

\begin{figure}\centering
	
\includegraphics[width=0.6\textwidth]{./images/stm-rutgers-modified.jpg}
\label{fig:STM-tip}

	\caption{Operating principles of an STM. A macroscopic sketch  shows the central piezo that controls the tip position on the sample. The main piezo is divided in four parts to control movement in the x-y plane and tip-sample distance\cite{STM-rutgers}. A microscopic sketch shows the tips movement in constant current mode while moving across a atomic step edge.}
	\label{fig:STM-sketch}
\end{figure}

\paragraph{Damping stages}
Two damping stages are used, one for the chamber and a sequential one for the STM.
\begin{itemize}
	\item The whole UHV system is placed on air pressurized legs. These can be operated on demand, so that the chamber floats on four dampers and external vibrations/shocks are damped.
	\item A second stage decouples the sensitive STM scanner from the rest of the setup. First the complete STM stage hangs on springs to further limit the direct influence of vibrations. Second the remaining oscillation amplitude is damped by a eddy current damping. It is made of three magnets in close proximity to the surrounding support so that eddy currents are induced for each movement. The eddy current is typically larger at cryogenic temperatures, that results in a damping that works best at low temperatures. The kinetic energy of the oscillating system is transferred by the eddy currents into heat within the surrounding conductor. The heat is then mitigated by the external cooling of the cryostat.
\end{itemize}

\subsection{Limitations}\index{STM:resolution}The accuracy of a STM is very high with spatial resolution down to the atomic scale. Due to the fact that the tips motion is controlled with different piezos, one has to take different elongations in different directions into account. For example, if the STM scans the fast scanning direction just a bit further than the slow scan direction, the resulting image (although pixel wise square) is no longer physically square anymore. Imagine a square (1:1 side ratio, diagonal angle 45\textdegree) where one side is elongated by 5\%. The resulting square (1:1.05 side ratio, diagonal angle 43.6\textdegree) looks square because it has the equal number of pixels in both directions, but it is physically rectangular. The expression used to calculate the uncertainty with known calibration parameters is
$$\Delta \Theta = 45 - \frac{180}{\pi}\cdot\arctan(\frac{1}{1+x})$$ where x is the percentage of one side being longer. This results in an uncertainty of 0.3\textdegree(1\%), 1.4\textdegree(5\%, see example above), 2.7\textdegree(10\%). For moderate shear, conformity is almost conserved and the uncertainty below 2\textdegree.

Because STM is sensible to electronic changes, it may change the footprint of an adsorbed compound \cite{sautet_interpretation_1992}. When laterally approaching an adsorbate this results in an additional tunneling current, because now electrons do not only tunnel directly into the substrate but through the adsorbate as well. Interferences between both tunneling processes depend on the adsorbate's orbital-symmetry and tip-shape. Local density of states calculations \cite{tersoff_theory_1985, lang_theory_1986, eigler_imaging_1991} is not adapted to grasp this effect since the tip is considered far away from the surface. Moreover, the tip radius or the tip-substrate distance is optimized to fit the lateral size of the adsorbate print with the experimental image \cite{tersoff_theory_1985, eigler_imaging_1991}.

\begin{itemize}
	\item Mechanical and thermal vibrations limit the resolution of the STM \& STS
	Mechanical and thermal vibrations limit the resolution of the STM, too. Therefor several damping stages decouple the STM from the surrounding. Although STM works at room temperature, additional cooling may be applied to reduce the thermal vibrations.
	\item No chemical resolution
	\subitem As complimentary method, XPS is used for chemical identification of adsorbates
\end{itemize}

\end{document}