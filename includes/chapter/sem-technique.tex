Invented in the 1930's by Manfred von Ardenne\cite{ardenne_elektronen-rastermikroskop_1938}, Scanning electron microscopy (SEM)\index{SEM} is another versatile tool for the experimentalist. In contrast to (LT-)STM and AFM, SEM is capable of imaging huge areas of the sample within a very short time, which allows for a vast overview as well as good statistics. Magnifications reach up to 500k and above, illustrating even features in the order of \SI{1}{\nano \meter}.

As the name already discloses, SEM scans the surface with electrons. Their interaction with the material are diverse and some of them are explained in the following. While all effects are present in every measurement, not every microscope features detectors for all of these. While detectors for secondary electrons are standart equipment others may be not.

\begin{itemize}
 \item \textbf{Secondary electrons (SE)} are produced in the bulk by the high energetic primary electon beam within close proximity to the surface. This is why SEMs offer a very good resolution of the surface itself.
  \item \textbf{Backscattered electrons (BSE)} are elastically scattered primary electrons. The resolution of this mode is not as high for the secondary electrons. The intensity of the BSE depends strongly on the the atomic number Z of the specimen. It is useful for a complementary view, for example when chemical composition is of high interest.
  Electron backscatter diffraction (EBSD) is used to achieve information on the crystallographic structure of a specimen.
 \item \textbf{Characteristic X-Rays} are used to identify the composition and measure the abundance of elements in the sample, too. See section \ref{sec:XPS} and figure \ref{fig:auger-core} therein for more details.
 \item \textbf{Cathodoluminescence (CL)} happens when electrons hit a material and exite photons. This effect is used in televesion screens where high energetic electrons are accelerated onto a screen containing phosphorus. There they distribute their energy with many others, some of those loose energy in form of photons which wavelenghts are within the visible spectrum. These light is called cathodoluminescence.
 \end{itemize}

The primary electrons are created with a filament. These often consist of tungsten (metal, high melting point, low work funtion). Alternatives are lanthanum hexaboride ($\textnormal{LB}_6$) - often used in LEED setups, too - or zirconium oxide.
Electrons are accelerated (typical energies are within \SIrange{1}{40}{\kilo \eV}) and focused on the specimen surface in a spot with few \si{nm} diameter with condenser lenses. Scanning the surface is achieved with coils that deflect the electron beam and therefore the actual scanning spot.

When the electrons hit the surface, they interact with the specimen in a small volume. The volume depends on the electron's energy, the atomic number Z of the specimen and the specimen's density. It is typically in the order of \SIrange{0.1}{5}{\micro \meter}.

Drawbacks:\\
\begin{itemize}
 \item[-] Sample has to be mounted $\rightarrow$ no in-situ measurement, surface alteration in between
 \item[-] Rather ``dirty'' vacuum $\rightarrow$ surface alteration while measuring
 \item[-] Measurement destroys sample $\rightarrow$ adsorbat build-up due to chemical reaction below e-beam
\end{itemize}

