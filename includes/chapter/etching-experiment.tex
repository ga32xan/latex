\paragraph{Experiment realization}The first attempt to etch the Cu-foil was performed with the $5\%_{vol}$ EG, $25\%_{vol}\,H_2O$ filled up with phosphoric acid. The etching was performed in a \SI{200}{\ml} beaker, filled with \SI{150}{\ml} etching solution. The setup is as depicted in fig \ref{fig:etching-setup}. The potential was adjusted to be \SI{1.2}{\V} after some minutes. The current through the solution changes and is typically highest when the etching process started. 

After some minutes, the foil starts to change. The reflectivity changes, making the foil - shiny before etching - a little dull. After some additional time the foil begins to reflect light better again. This is the moment where the etching process is interrupted. The time inside the etching solution depends on the handling (like shaking the beaker or moving the foil in the solution - but was usually $\geq \SI{2}{\hour}$.\footnote{Since the perfect point to perform polishing varies in time a automated etching process has been developed \cite{palmieri_besides_2001}}

One has to be careful if reproducible results are needed. During the etching process (as more and more copper settles on the counter electrode), the current and therefore the etching rate decrease continuously. When the beaker is moved, some of the debris on the electrode changes (the electrode's) surrounding and the etching rate (limiting current) increases again. Front- and backside of the foil are suspect to different etching rates. The back side is generally more flat, the side facing the counter-electrode always shows some additional protrusions.

\paragraph{After etching treatment}
The foil is taken out and cleaned from remaining etchants with purified water first and isopropanol afterwards. Foils are be stored in ethanol to avoid oxidation. 

To further improve the quality of the foil, one can follow the documented recipe for annealing the foil in a $H_2$ atmosphere (\SI{10}{sscm}, \SI{1000}{\celsius}, 30min)\cite{kim_synthesis_2012} to increase the copper grain size and further smoothen the surface. 

So prepared foils are investigated in XPS (compare figure \ref{fig:xps-self-grown}) - (discussion of peaks can be found inside the text. Comparable experiments  are performed by \cite[8]{stables_report_2008}).
After the etching process one foil is investigated in SEM. It was stored for one day in isopropanol and blown dry with nitrogen. 

\paragraph{SEM}

\input{./includes/chapter/sem-technique}

\begin{figure}[]
	\begin{center}
		\includegraphics[height=6cm]{./images/Domenik_16031715.jpg}
		\includegraphics[height=6cm]{./images/Domenik_16031717.jpg}
	\end{center}
	\caption{SEM image of etched copper foil. Different contrast suggests different grain-orientation within the surface. a) Larger (\SI{570}{\micro \meter} x \SI{380}{\micro \meter}) image showing the contrast of different grains in the copper-foil, b) zoom (\SI{18}{\micro \meter} x \SI{12}{\micro \meter})} to a area with two different contrasts and their border.
	\label{fig:SEM-gb}
\end{figure}

One can see (\autoref{fig:SEM-gb}) that the surface imaged in different intensities which correspond to the different orientation of the copper grains within the foil\cite{wu_effects_2015}. The grain size may range from just a few \SI{}{\micro \meter} to several hundret \SI{}{\micro \meter} and in some cases even \SI{}{\milli \meter}. The grains are separated by grain boundaries. Large grains are preferred for growing graphene on copper foils because grain boundaries are subject to inhomogeneities within the graphene layer and provide a route for unwanted surface chemistry (copper oxidation for example). These effect can be also be used to highlight grain boundaries as shown in \cite{wu_effects_2015}.

Although not very rough, the copper foil shows surface variation. While some areas of the sample show some wavy surface, whereas other are much flatter and appear in a different intensity (\autoref{fig:SEM-surface}).

Neither estimation on the grainsize nor guesses for their absolute orientation have been done due to the lack of EBSD-data in the SEM setup.

\begin{figure}[]
	\begin{center}
		\includegraphics[height=6cm]{./images/Domenik_16031700.jpg}
		%\includegraphics[height=6cm]{../Daten/SEM/160317-Domenik/Domenik_16031717.jpg}
	\end{center}
	\caption{SEM image that shows different surface morphologies (\SI{5.6}{\micro \meter}x\SI{3.7}{\micro \meter})}
	\label{fig:SEM-surface}
\end{figure}

\paragraph{AFM}
\begin{figure}[] \centering
	\subfigure[RMS $\approx$\SI{13}{\nm}, contrast \SI{100}{\nm}]{
	\includegraphics[width=0.4\textwidth]{./images/as_bought0000.jpg}
	}
	\subfigure[RMS $\approx$ RMS \SI{9}{\nm}, contrast \SI{70}{\nm} in the right image]{
	\includegraphics[width=0.4\textwidth]{./images/as_bought0001.jpg}
	}
	
	\caption{AFM image of the \SI{0.25}{\mm} copper foil as bought from alfa aesar.}
	\label{fig:foil-afm-as-bought}
\end{figure}

Figure \ref{fig:foil-afm-as-bought} shows the striations that stem from the production process (from top to buttom).
\begin{figure}[] \centering
	\subfigure[RMS $\approx$\SI{9}{\nm} in the left image, contrast \SI{100}{\nm}]{
	\includegraphics[width=0.4\textwidth]{./images/polished0000.jpg}
}
	\subfigure[RMS $\approx$\SI{3}{\nm} in the right image, contrast \SI{70}{\nm}]{
	\includegraphics[width=0.4\textwidth]{./images/polished0001.jpg}
}
	\caption{AFM image of a copper foil polished 5h (according to table \ref{tab:used-etching-solution})}
	\label{fig:foil-afm-polished}
\end{figure}
After etching ($U=1.2V$,I=\SIrange{120}{250}{\mA}) for \SI{5}{\hour} in a solution (shown in table \ref{tab:used-etching-solution}) the striations have gone and the RMS value decreased by \SIrange{30}{45}{\percent} and an increase in foil quality is obvious even with bare eyes. Figure \ref{fig:foil-afm-as-bought} and \ref{fig:foil-afm-polished} show AFM images in the same size and contrast - before and after etching.
The circular hole is an effect of bubbles in the etching process where the bubble affects the rate of etching. The over all structure changes from a heterogenous sample height to a flat height contribution with only a little amount of defects. Those are sufficiently seperated in space to exhibit flat regions where the h-BN may grow unperturbated.



\paragraph{not done yet - maybe future?}
Some foil has been mechanically polished with 4k paper and several hours of Syton polishing. The roughness of these samples has been investigated also in AFM. These are comparable to the chemically polished ones, but are always slightly higher by $\approx 10\%$. Sometimes unwanted new scratches appear after mechanical polish.

\paragraph{STM}
The bought and chemically polished foils are mounted on a sample holder and loaded into the load lock. It is evacuated for \SIrange{2}{3}{\hour}, afterwards the valve is opened to the chamber. During transfer, no noteable increase in the base pressure is noted. The sample is put on the parking slot.

The sample was initially degassed with slowly increasing temperatures to remove adsorbates like $CO, CO_2$ and $H_2O$.

After some time of degassing, the sample was prepared with repeated sputter and anneal cycles. The annealing temperatures were increased up to \SI{800}{\degreeCelsius}. 
After that procedure, the sample was cooled down and observed in STM.