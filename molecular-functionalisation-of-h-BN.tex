% !TeX encoding = UTF-8
% Entwurf der Doktorarbeit von Domenik Zimmermann, TU M\"unchen, Grp. Prof. Dr. Auw\"arter 
%%%%%%%%%%%%%%%%%%%%%%%%%%%%%%%%%%%%%%%%%%%%%%%%%%%%%%%
\author{Domenik Matthias Zimmermann}
\title{Molecular adsorption on \textit{h}-BN}
%%%%%%%%%%%%%%%%%%%%%%%%%%%%%%%%%%%%%%%%%%%%%%%%%%%%%%%
% Bilder: AFM/STM 1024x1024 px exportieren, 
%
% Fonts are embedded by default, otherwise check 
% https://tex.stackexchange.com/questions/10391/how-to-embed-fonts-at-compile-time-with-pdflatex
% 
% sudo apt-get install kile texlive-latex-extra texlive-science texlive-bibtex-extra biber
% adjust kile tools to call biber %lS instead of bibtex
%
% If errors occure referring to aux files or others:
%	Delete helper files (*.aux) and maybe others!
%
% Use MikTeX version 2.9.6615 or later (check that biber is included)
% Using BiBLaTex, biber as frontend.
% If biber complains (ERROR - Error: Found biblatex control file version 2.6, expected version 3.4.)
% 	Update with your Miktex package manager (mpm_mfc.exe[single user install] or mpm_mfc_admin.exe [all user install])
% no new biblatex version there? Sync database? NO improvement!
%%%%%%%%%%%%%%%%%%%%%%%%%%%%% Setup for document %%%%%%%%%%%%%%%%%%%%%%%%%%%%%%%%%
\documentclass[
%11pt,					% Schriftgröße, 11pt is standat
%a5paper,				% a4paper is standart
twoside,				% Zweiseitig
BCOR=8mm,				% Bindekorrektur inkl. Biegefalz
headings=normal,		% Kleinere Kapitelüberschriften => check preamble
headsepline,			% Enable line to seperate head ...
footsepline,			% ... and foot
plainfootsepline,		% Footseperation line on chapter start
%,draft
]{scrbook}
%%%%%%%%%%%%%%%%%%%%%%%%%%%%% Change Page Geometry %%%%%%%%%%%%%%%%%%%%%%%%%%%%%%
\usepackage[
top=30mm,
bottom=55mm,
inner=25mm,
outer=30mm,
%marginparsep=7mm,
%marginparwidth=48mm,
%paperwidth=210mm,
%paperheight=245mm, 286mm (A4)
]{geometry}
%%%%%%%%%%%%%%%%%%%%%%%%%%%%%%%%%%%%%%%%%%%%%%%%%%%%%%%%%%%%%%%%%%%%%%%%%%%%%%%%%%
%%%%%%%%%%%%%%%%%%%%%%%%%%%%%%%%%%%%%%%%%%%%%%%%%%%%%%%
\author{Domenik Matthias Zimmermann}
\title{Molecular fictionalization of h-BN}
%%%%%%%%%%%%%%%%%%%%%%%%%%%%%%%%%%%%%%%%%%%%%%%%%%%%%%%
\usepackage[utf8]{inputenc}
% when latex complains about unicode char U+2212 is not configured for use in latex use the line below
\DeclareUnicodeCharacter{2212}{-}% support older LaTeX versions
%\usepackage[latin1]{inputenc}
\usepackage[T1]{fontenc}
\usepackage{lmodern}
\usepackage[english]{babel}
\usepackage{csquotes}
\usepackage{amsmath}
\usepackage{textcomp}		%enables \textdegree to use as �
\usepackage{amsfonts}
\usepackage{amssymb}
\usepackage{graphicx}
\usepackage{xhfill}			% provides /hrulefill (disclaimer)
%\usepackage{wasysym}		
\usepackage{braket}		% fur <A|H|B> <A| |A> oder <A>
%\DeclareGraphicsExtensions{.pdf,.png,.jpg}
%%%%%%%%%%%%%%%%%%%%%%%%%%%%%
\usepackage{siunitx}
\DeclareSIUnit\langmuir{L}
%%%%%%%%%%%%%%%%%%%%%%%%%%%%%
\usepackage[hidelinks,breaklinks=true]{hyperref}
%\usepackage{url}
\usepackage[section]{placeins} %definiert \floatbarrier, mit option automatisch bei jeder section
%%%%%%%%%%%%%%%%%%%%%%%%%%%%%%%%%%%%%%%%%%%
\usepackage{subfigure}
\usepackage{wrapfig}
%%%%%%%%%%%%%%%%%%%%%%%%%%%%%%%%%%%%%%%%%%%
\usepackage{caption}
\usepackage{microtype}
%\usepackage{subcaption}
\usepackage{multicol}
\usepackage{multirow}
%%%%%%%%%%%%%%%%%%%%%%%%%%%%
% fuer Stichwortverzeichnis
\usepackage{makeidx}
% Stichwortverzeichnis erstellen
\makeindex
%%%%%%%%%%%%%%%%%%%%%%%%%%%%%%%%%%%%%%%%%%%%%%%%%%%%%%%
\usepackage[style=numeric		% bibliogryphy-styles: alphabetic, numeric, chem-angew, ieee, nature, science
,backend=biber
%,refsection=chapter			% setzt bibliographies nach chaptern getrennt, nach jedem chapter muss ein 
								% printbibliogrphy stehen
]{biblatex} 	
\addbibresource{./bib.bib}  	% relative to root directory (where the file that includes this file is located)! 
								%do NOT OMIT .bib ending

%avoids ugly line breaks within bibligraphy
\addto\bibsetup{\setlength{\emergencystretch}{1.5em}} 	

% Zum Verwalten der Zitate benutze ich Zotero, zum Erzeugen der .bib-Datein f�r Latex wird die Exportfunktion von Zotero benutzt (rechtsklick auf ``Meine Bibliothek'' im linken Reiter: Option Biblatex in die Datei bib-zotero-export.bib aus welcher ich dann die betreffenden Zitate auf Richtigkeit \"uberpr\"ufe und in die bib.bib kopiere.
%%%%%%%%%%%%%%%%%%%%%%%%%%%%%%%%%%%%%%%%%%%%%%%%%%%%%%%
% \usepackage[top=2.5cm,left=2.5cm,right=3.5cm,bottom=3.5cm]{geometry}
%%%%%%%%%%%%%%%%%%%%%%%%%%%%%%%%%%%%%%%%%%%%%%%%%%%%%%%
\usepackage{xcolor}
%%%%%%%%%%%%%%%%%%%%%%%%%%%%%%%%%%%%%%%%%%%%%%%%%%%%%%
%%%%%%%%%%%%%%%%%%%%%%%%%%%%%%%%%%%%%%%%%%%%%%%%%%%%%%
\usepackage[draft=false]{scrlayer-scrpage}		%deaktiviert ruler in der draft version
\pagestyle{scrheadings}
%%%%%%%%%%%%%%%%%%%%%%%%%%%%%%%%%%%%%%%%%%%%%%%%%%%%%%
\ifdefined\daumenkino
	\usepackage{etex}
	\usepackage{intcalc} 
	\newcommand*{\AnzBilder}{200}		            		%<--Variablen anpassen
\newcommand*{\KinoPfad}{./images/animation/lumo/} 	%<--Variablen anpassen

%%%%Quelltext%%%
\newcommand*{\SafeboxName}{sbKino}

\makeatletter
%Erzeugt neue Saveboxen und füllt sie mit includegraphics-Anweisungen
%Aufruf: \NewSaveBoxes{sbKino}{5}{daumenkino/kino}
\newcommand*{\NewSaveBoxes}[3]{%
	\@tempcnta 1
	\@whilenum \@tempcnta< \numexpr(#2+1) \do{%
		%Savebox anlegen
		\expandafter\newsavebox\csname #1\the\@tempcnta\endcsname
		%Savebox mit Leben füllen
		\expandafter\savebox\csname #1\the\@tempcnta\endcsname{%
			\includegraphics[width=0.5cm]{#3\the\@tempcnta}%
		}%
		\advance\@tempcnta 1
	}%
}

\newcommand*{\bildnr}{\numexpr\intcalcMod{\numexpr\value{page}}{\numexpr\AnzBilder}\relax}
\newcommand*{\lumoseries}{%
	\usebox{\@nameuse{\SafeboxName\the\bildnr}}%
}
\makeatother
\NewSaveBoxes{\SafeboxName}{\AnzBilder}{\KinoPfad}
	\lofoot{\lumoseries} %<-- Eigentlicher Aufruf für Fußzeile
	
	\newcommand*{\AnzBilderLogo}{200}		            		%<--Variablen anpassen
\newcommand*{\KinoPfadLogo}{./images/animation/logo/} 	%<--Variablen anpassen
%%%%Quelltext%%%
\newcommand*{\SafeboxNameLogo}{sbKinologo}

\makeatletter
%Erzeugt neue Saveboxen und füllt sie mit includegraphics-Anweisungen
%Aufruf: \NewSaveBoxesLogo{sbKino}{5}{daumenkino/kino}
\newcommand*{\NewSaveBoxesLogo}[3]{%
	\@tempcntb 1
	\@whilenum \@tempcntb< \numexpr(#2+1) \do{%
		%Savebox anlegen
		\expandafter\newsavebox\csname #1\the\@tempcntb\endcsname
		%Savebox mit Leben füllen
		\expandafter\savebox\csname #1\the\@tempcntb\endcsname{%
			\includegraphics[width=0.5cm]{#3\the\@tempcntb}%
		}%
		\advance\@tempcntb 1
	}%
}
%intcalc-version
\newcommand*{\bildnrLogo}{\numexpr\intcalcMod{\numexpr\value{page}}{\numexpr\AnzBilderLogo}\relax}

\newcommand*{\logoseries}{%
	\usebox{\@nameuse{\SafeboxNameLogo\the\bildnrLogo}}%
}
\makeatother
\NewSaveBoxesLogo{\SafeboxNameLogo}{\AnzBilderLogo}{\KinoPfadLogo}
%%%Aufruf%%%%%%%
	\refoot{\logoseries} %<-- Eigentlicher Aufruf für Fußzeile
\fi
%%%%%%%%%%%%%%%%%%%%%%%%%%%%%%%%%%%%%%%%%%%%%%%%%%%%%%
%%%%%%%%%%%%%%%%%%%%%%%%%%%%%%%%%%%%%%%%%%%%%%%%%%%%%%
% Basic information for cover & title page
\newcommand*{\getUniversity}{Technische Universit\"at M\"unchen}
\newcommand*{\getFaculty}{Department of physics}
\newcommand*{\getFacultyger}{Fakult\"at f\"ur Physik}
\newcommand*{\getTitle}{Molecular adsorption on \textit{h}-BN}
%newcommand*{\getTitleger}{TODO: Titel der Abschlussarbeit}
\newcommand*{\getAuthor}{Domenik Matthias Zimmermann}
\newcommand*{\getDoctype}{Dissertation}
\newcommand*{\getDoctypeger}{Vollst\"andiger Abdruck der von der Fakult\"at für Physik der Technischen Universit\"at M\"unchen zur Erlangung des akademischen Grades eines Doktors der Naturwissenschaften (Dr. rer. nat.) genehmigten Dissertation.}
\newcommand*{\getSupervisor}{Prof.\ Dr.\ Wilhelm Auw\"arter}
\newcommand*{\getChairman}{TODO: Chairman}
\newcommand*{\getFirstExaminer}{TODO: 1. Examiner}
\newcommand*{\getSecondExaminer}{TODO: 2. Examiner}
\newcommand*{\getSubmissionDate}{TODO: Submission date}
\newcommand*{\getSubmissionLocation}{Munich}
\newcommand*{\getDisclaimer}{I assure the single handed composition of this \MakeLowercase{\getDoctype{}} only supported by declared resources.}
% TODO: add custom commands etc.

%%%%%%%%%%%%%%%%%%%%%%%%%%%%%%%%%%%%%%%%%%%%%%%%%%%%%%%%%%%%%%%%%%%%%%%%%%%%%%%%%%%%%%%%%%%%%
% Change \autoref (Figure x.x) to FIGURE x.x

%%%%%%%%%%%%%%%%%%%%%%%%%%%%%%%%%%%%%%%%%%%%%%%%%%%%%%%%%%%%%%%%%%%%%%%%%%%%%%%%%%%%%%%%%%%%%
% Python style for highlighting
% Check https://tex.stackexchange.com/questions/83882/how-to-highlight-python-syntax-in-latex-listings-lstinputlistings-command#83883

% Default fixed font does not support bold face
\DeclareFixedFont{\ttb}{T1}{txtt}{bx}{n}{12} % for bold
\DeclareFixedFont{\ttm}{T1}{txtt}{m}{n}{12}  % for normal

\newcommand\pythonstyle{\lstset{
		language=Python,
		basicstyle=\small,             %\ttm, \tiny \small
		otherkeywords={self},             % Add keywords here
		keywordstyle=\small\color{deepblue},
		emph={MyClass,__init__},          % Custom highlighting
		emphstyle=\small\color{deepred},    % Custom highlighting style
		stringstyle=\color{deepgreen},
		frame=tb,                         % Any extra options here
		showstringspaces=false,           % 
		frame=single,
		numbers = left,
		numbersep=5pt
}}
% Python environment
\lstnewenvironment{python}[1][]
{
	\pythonstyle
	\lstset{#1}
}
{}
% Python for external files
\newcommand\pythonexternal[2][]{{
		\pythonstyle
		\lstinputlisting[#1]{#2}}}
% Python for inline
\newcommand\pythoninline[1]{{\pythonstyle\lstinline!#1!}}
%%%%%%%%%%%%%%%%%%%%%%%%%%%%%%%%%%%%%%%%%%%%%%%%%%%%%%%%%%%%%%%%%%%%%%%%%%%%%%%%%%%%%%%%%%%%%
%%%%%%%%% Appendix name %%%%%%%%%%%
\newcommand*{\Appendixautorefname}{Appendix}


%%%%%%%%%%%%%%%%%%%% Draft Mode Compilation from command line %%%%%%%%%%%%%%%%%%%%
%%%%%%%%%%%%%%%%%%%% pdflatex "\def\draftmode{1} \input{FILE}"  %%%%%%%%%%%%%%%%%%
%%%%%% TEXSTUDIO %%% pdflatex "\def\draftmode{1} \input{%.tex}" %%%%%%%%%%%%%%%%%%
%%%%%%%%%%%%%%%%%%%%%%%%%%%%%%%%%%%%%%%%%%%%%%%%%%%%%%%%%%%%%%%%%%%%%%%%%%%%%%%%%%
%\def\daumenkino{1}
%%%%%%%%%%%%%%%%%%%%%%%%%%%%%%%%%%%%%%%%%%%%%%%%%%%%%%%%%%%%%%%%%%%%%%%%%%%%%%%%%%
%%%%%%%%%%%%%%%%%%%%%%%%%%%%%%%%%%%%%%%%%%%%%%%%%%%%%%%%%%%%%%%%%%%%%%%%%%%%%%%%%%

\begin{document}
%%%%%%%%%%%%%%%%%%%%%%%%%%%%%%%%%%%%%%%%%%%%%%%%%%%%%%%%%%%%%%%%%%%%%%%%%%%%%%%%%%%%%%%%%%%%%%%%%%%%%
%\renewcommand{\figureautorefname}{\texttt{\figureautorefname}} 

 \input{./includes/frontmatter/cover}											%					%
 \frontmatter{}			 														%					%
   \begin{titlepage}
\begin{center}
\centering
\vspace{50mm}
\includegraphics[width=40mm]{./includes/logo/tum} \\
\vspace{10mm}
{\Huge\MakeUppercase{\getUniversity{}}} \\
\vspace{15mm}
{\huge\MakeUppercase{\getFaculty{}}} \\
\vspace{20mm}
{\Large \getDoctype{}} \\
\vspace{15mm}
{\huge\bfseries \getTitle{}} \\
\vspace{15mm}
\end{center}
\getDoctypeger{} \\
\vspace{15mm}
%{\huge\bfseries \getTitle{}} \\
%\vspace{10mm}

\begin{center}
\begin{tabular}{l l}
Author: & \getAuthor{} \\
Supervisor: & \getSupervisor{} \\
Submission Date: & \getSubmissionDate{} \\ \hline
Chairman: & \getChairman{} \\
1. Examiner: & \getFirstExaminer{} \\
2. Examiner: & \getSecondExaminer{} \\
\end{tabular}
%\vspace{10mm} \\
%\includegraphics[width=20mm]{./includes/logo/physics}
\end{center}
\end{titlepage}										%					%
   \thispagestyle{empty}
\vspace*{0.75\textheight}
\noindent
I assure the single handed composition of this \MakeLowercase{\getDoctype{}} only supported by declared resources. \newline \bigskip 
\newline 
\begin{tabular}{c}
\hrulefill \\
\end{tabular}

\medskip \noindent
\getSubmissionLocation{}, \getSubmissionDate{} \hspace{2cm} \getAuthor{}
\cleardoublepage{}									% 					%
   \addcontentsline{toc}{chapter}{Acknowledgments}		%Entry in TOC
\thispagestyle{empty}
%\vspace*{2cm}
\begin{center}
{\usekomafont{section} Acknowledgments}
\end{center}
\vspace{1cm}
A lot of people are invoked in the course of the thesis and I like to thank them for their support. First and foremost I like to thank Prof.\ Dr.\ W.\ Auw\"arter for the opportunity to work in his research group. The wide field of molecular assembly and functionalization opened up many interesting insights that would never be possible without him. The time dedicated to experiments under his leadership was very exciting and his review challenged my ambition to further improve. Together with Prof.\ Dr.\ J.\ Barth a very pleasant professional environment was created with retreats and group activities framing our day-to-day work.

Besides them, many people were incorporated at different levels, starting from introducing technical aspects and measurement techniques used at the different setups via result discussion through to proof-reading and concept validation. 

	Dr.\ A.\ Wiengarten was the operator of the low temperature scanning tunneling microscope and I started under her supervision with the first measurements. Her profound understanding of the setup shared with me was consolidated by Dr.\ K.\ Seufert whose deep commitment nurtured my engagement. His enjoyment of work eased many hours in the lab.

	Measurements at the low temperature atomic force microscope are supported by M.\ P\"ortner and S.\ Synkule. Their fruitful result discussion together with Dr.\ A.\ Riss clarified topics in technical and scientific aspects easily.

	The combined room temperature STM and X-Ray photoelectron spectroscopy setup was assembled by Dr.\ M.\ Schwarz whose continuous efforts in setup details enabled the use of this machine. Together with discussion and operational help of A.\ Baklanov, many measurements were done smoothly.

	Further XPS measurements were done at the machine operated by the Nanosystems Initiative Munich represented by \textbf{\underline{LALALA}} and advocated by Dr.\ J.\ Reichert. Scientific discussion with him, K.\ Eberle and Prof.\ Dr.\ Feulner often lead to meaningful outcomes and was always a pleasure.

	The few measurements at the scanning electron microscope were done under supervision of Dr.\ \textbf{\underline{LALALA}} and Y.\ Gong.

	Please let me thank all member of E20 for a nice time together that showed me the value of social interaction combined with consuming delights and limits.

\begin{center}Thank you.
\newpage 
\thispagestyle{empty}
Dedicated to the beloved ones that made me who I am. Without you these endless lines of text would never end. 
%\epigraph{All work and no play makes Jack a dull boy}{James Howell\\Paroimiographia}
%\epigraph{Goodbye, Jake. I love you, dear.}{Stephen King, The Dark Tower} 
\epigraph{Gone. Like a candle-flame. To whatever worlds there are.}{Stephen King, The Dark Tower}
\epigraph{We go to seek a better world. May you find one, as well.}{Stephen King, The Dark Tower}
\end{center}
\cleardoublepage{}								%	 				%
   \chapter{\abstractname}
Good quality, two dimensional, hexagonal boron nitride (\textit{h}-BN) islands require a clean and flat surface to grow on. Within this thesis, techniques to chemically polish the surface are shortly reviewed. After polishing, the growth of sub-monolayer \textit{h}-BN islands is investigated on poly crystalline copper surfaces by means of STM and XPS. The use of \textit{h}-BN, grown on these foils as insulating and electronically decoupling substrate, is shown by reproducing molecular adsorption known on \textit{h}-BN grown on single crystalline copper.

Different molecular species are investigated with regard to their electronic properties and geometric structures, formed by self-assembly on metallic and insulating \textit{h}-BN substrates.

Bis- \& Tetra-pyridin-4-ylethynyl functionalized pyrene molcules are adsorped on \textit{h}-BN/Cu(111) to show the diversity of self-assembled molecular assemblies that can be steered by the number and position of functional groups. It is shown that their opto-electronic properties and assembly after adsorption are determined by the chemical design of the molecule and show the same trend as gas-phase calculations: An decrease in electronic band gap with increasing number of functional groups. Frontier orbital resolution is achieved in STM under modified tip conditions and a wide band gap in STS shows efficient decoupling from the metallic substrate by the \textit{h}-BN layer. 

Self-assembly and electronic structure of functionalized coronene molecules are investigated on Ag(111) and Au(111) surfaces. The importance of side groups in the formation process of self-assembled structures is clarified in STM. nc-AFM is used to clarify sub-molecular structure and the formation of linked structures after annealing treatment in UHV.

Single and bis- nitro functionalized porphine molecules are adsorbed on Cu, Ag and \textit{h}-BN/Cu(111). Di-tert-butyl-phenyl side groups are used to further decouple the molecule from the substrate layer and increase mobility at low temperatures. The molecular self-assembly is controlled by the number of functional groups, so that bis-functionalized molecules adsorbed on Ag(100) form hexagonal superstructures with, mismatching the 2-fold symmetric substrate symmetry.
	
Helicene molecules are used to investigate the influence of chirality and dipole moment on the formation of self-assembled islands. Depending on the substrate, molecular assembly varies from chains formed with specific orientation to the metal substrate's high symmetry directions to dense packed islands formed after adsorption on \textit{h}-BN/Cu(111). Annealing after adsorption on Ag resulted in a ring-closure reaction at the helicene's spiral terminations that lifts chirality.

At last a design for a peltier cooling unit is given, which is used to store liquids with a volume of serveral \SI{}{\milli \liter} at temperatures around \SI{0}{\celsius}.										%					%
 \microtypesetup{protrusion=false}												%					%
 \setcounter{tocdepth}{4}														%   [1=sections]	%
	\tableofcontents{}														 	% 					%
% 	\listoffigures														 		% 					%
% 	\listoftables																% 					%
 \microtypesetup{protrusion=true}												%		 			%
%%%%%%%%%%%%%%%%%%%%%%%%%%%%%%%%%%%%%%%%%%%%%%%%%%%%%%%%%%%%%%%%%%%%%%%%%%%%%%%%%%%%%%%%%%%%%%%%%%%%%
%%%%%%%%%%%%%%%%%%%%%%%%%%%%%%%%%%%%%%%%%%%%%
%\chapter{Preface}

\begin{itemize}	
	\item Science is a tool to increase knowledge
	\subitem  academic examples
	\item Science has made life easier in lots of areas 
	\subitem	examples
	\item Without translation of fundamental science none of the above mentioned features would be implemented in every day life
	\subitem examples of failed translations
\end{itemize}	
%%%%%%%%%%%%%%%%%%%%%%%%%%%%%%%%%%%%%%%%%%%%%%%%%%%%%%%%%%%%%%%%%%%%%%%%%%%%%%%%%%%%%%%%%%%%%%%%%%%%%
\mainmatter{}
\chapter{Introduction}
This is the introduction............. write some! :D
%%%%%%%%%%%%%%%%%%%%%%%%%%%%%%%%%%%%%%%%%%%%%
\chapter{Experimental methods}
Several experimental methods are used within this work to determine different physical properties. In this chapter we review the principles, benefits and limitations for each method. All of them are well known in surface science and used to analyze thin films on different substrates with respect to their geometric, electronic and chemical properties.

  \section{\textbf{S}canning \textbf{T}unneling \textbf{M}icroscopy}
    \subsection{Overview...historically}
\subsection{Theory...on 1D tunneling at a single point}
\index{STM!One dimensional tunneling}
While the tip (metal) is far away from the sample, their vacuum levels are the same. The corresponding Fermi energies of sample and tip lie below the vacuum level by the amount of their work functions ($\Phi_s$ and $\Phi_t$ for sample and tip respectively). Wave functions of electrons within the tip and sample decay exponentially in vacuum, depended on their energy with respect to the Fermi level.
If sample and tip are in thermodynamic equilibrium, their Fermi levels are the same. Electrons now face a potential barrier (approximately rectangular) which can be overcome if their energy is high enough and the barrier sufficiently narrow. When a voltage is applied across the tunneling barrier, the energy of the tip-electrons is shifted by $eV$ as illustrated in \autoref{fig:STM-barrier}. When a positive bias voltage is applied, electrons tunnel from the tip into unoccupied states in the sample - a negative bias results in a tunneling current in opposite direction. 

\begin{figure}[]\centering
	\includegraphics[width=0.7\textwidth]{./images/tunnel-barrier}
	\caption{Energy diagram to visualize the tunneling process between sample (left) and tip (right) separated by a distance d. Work functions of sample and tip ($\Phi_s$ and $\Phi_t$) separate the filled states (shaded regions) and the vacuum level ($\epsilon_{vac}$). Since sample ($\rho_s$) and tip DOS ($\rho_t$) may not be uniform, a fictional DOS is sketched in darker colors between both. The samples energy is lifted by $eV$ after a bias is applied and results in a net electron current from the sample into the tip. One tunneling process is indicated by a wave function in the sample. After overcoming the vacuum barrier its amplitude decreases and the corresponding electron occupies a free state (not shaded) in the tip material.  Taken from \cite{diss-schunack}}
	\label{fig:STM-barrier}
\end{figure}


Following the model of \index{STM!Tersoff-Hamann} Tersoff-Hamann\footnote{Please note that there are more models and corrections to them. An evolution from Bardeen's approach to the one done by Tersoff-Hamann can be found here \cite{lounis_theory_2014, wortmann_interpretation_2000} including Chen’s expansion.}((1) uniform density of states in the tip, (2) temperature is low, (3) small bias voltage of some mV, (4) waveform of electrons in tip are s-waves) the tunneling current results to 
$$I=32\pi^3\hbar^{-1}e^2V\Phi_t^2 R^2\kappa^{-4}e^{-2\kappa R}\rho_t(E_F)\rho_s(r_o,E_F)$$ where $\rho_t$ is the density of states per unit volume of the tip, R the tip radius and $\rho_s(r_0,E_F)$ the Fermi level density of states in the sample\cite{bonnell_scanning_1993}. The distance between tip and sample is denoted as $Z$ and the inverse decay length of the electrons wave function is $\kappa=\frac{\sqrt{2m\Phi_t}}{\hbar}$. If $I$ is held constant one can see that the tip in principle follows a contour of constant Fermi level density of states at the sample surface, measured at the center of the curvature of the s-wave like tip. While its a good first approximation of the system, in many cases the bias is much higher than 10mV (\SIrange{1}{5}{\V}) so more than just the electrons near Fermi contribute. Also a uniform $\rho_t$ may not be accurate in all cases.

Using \index{STM!WKB} Wentzel-Kramers-Brillouin (WKB) theory\cite{wentzel_verallgemeinerung_1926, kramers_wellenmechanik_1926, brillouin_mecanique_1926} the tunneling current is given by
\begin{equation}
I=\int_0^{eV}\rho_s(r,E)\rho_t(r,eV+E)T(E,eV,r)dE
\label{WKB}
\end{equation}
where $\rho_s(\rho_t)$ is the density of states of the sample (tip) and T is the tunneling transmission probability
\begin{equation}
T(E,eV)=exp\left(-\frac{\textcolor{red}{\textbf{2}}Z\sqrt{2m}}{\hbar}\sqrt{\frac{\Phi_s+\Phi_t}{2}+\frac{eV}{2}-E}\right)
\label{Transmission-function} 
\end{equation}
If $eV<0$ the tunneling current is largest for $E=0$ (electrons on the Fermi-level of the sample), if $eV>0$ the tunneling current is largest for $E=eV$ (electrons of Fermi level in tip).

Due to the fact that the tunneling current is proportional the density of states in the tip and the molecule one can deduce the band structure within a range of several volts in the vicinity of the Fermi energy.

Since states with highest energy have the largest decay lengths in vacuum, most of the tunneling current is determined by electrons within close proximity to the Fermi level.\footnote{More information related to tunneling processes can be found here \cite{bonnell_scanning_1993}.}

\subsection{\textbf{S}canning \textbf{T}unneling \textbf{S}pectroscopy}
\paragraph{Theory...on STS}
\label{section:STS}
First changes of the tunneling current with the bias voltage were observed by Tromp et al. in 1986 \cite{tromp_atomic_1986}. They discovered a change in contrast when scanning a SI(111) surface with either positive or negative bias. The change in contrast is most apparent in semiconductors and semi metals\cite{bonnell_scanning_1993}, but adsorbates and charged areas of the sample change the DOS locally and therefore the contrast in STM. While simple results may be already obtained when comparing two images recorded at different voltages, more detailed information can be achieved. At low temperatures the vanishing lateral movement of molecules makes them also accessible to tunneling spectroscopy. It is possible to deduce the electronic configuration on with atomic spatial resolution.

Spectroscopic information (information on the DOS) can be obtained by either changing the bias voltage (I(V,z)-spectroscopy) or the tip-sample distance (V(z)-spectroscopy).  

Therefore the bias is modulated with a sinus like waveform. \index{STS!modulation}The frequency of the low amplitude modulation of the DC bias is much larger than the feedback loop frequency (\SIrange{1}{2}{\kilo \hertz}). The AC part of the tunneling signal is than recorded with a lock in amplifier. The in-phase component is directly the $dI/dV|_{V=V_{bias}}$, recorded simultaneously with the topography.\footnote{If the modulation frequency is too low, the feedback tries to compensate the modulation by changing the distance to the sample.	If the modulation frequency is too high, the capacitance between tip and sample leads to an $90\deg$ phase shifted current which increases with modulation frequency. One usually chooses the modulation frequency slightly above the cutoff frequency for the feedback loop.}

\index{STS!Bias below work function}
First let us consider small biases.
If tunneling conditions are such that $eV\leq\Phi$, observed features in $dI/dV$ are associated with the surface DOS. Critical points in the surface projected DOS give rise to features in $dI/dV$. Interpretation of these features with the WKB theory (i.e. differentiating equation \eqref{WKB}) gives
$$dI/dV=\rho_s(r,eV)\rho_t(r,0)T(\textcolor{red}{\textbf{eV}},eV,r)+\int_0^{eV}\rho_s(eV)\rho_t(r,E-eV)\frac{dT(E,eV,r)}{dV}dE$$
The first term contains the DOS of the sample and tip and the transmission function. While it is usually unknown, a closer look to \eqref{Transmission-function} indicates a smooth, monotonically increasing function in V. This mannered dependence on V gives a smooth background described by the second term $\int_0^{eV}\rho_s(eV)\rho_t(r,E-eV)\frac{dT(E,eV,r)}{dV}dE$.
Because T is smooth and monotonic the first term $\rho_s(r,eV)\rho_t(r,0)T(eV,eV,r)$ introduces the dependence on the DOS in the sample for energies $eV$ - our desired spectrum.

If $dI/dV$ is recorded simultaneously with the topography, another contribution arises. One usually observes an decrease in atomic corrugation when the distance between tip and sample is increased. The surface looks flat. To have the same tunneling current on atom positions and in between, the decay length in the valleys $\kappa_v$ must be larger than on the atom positions $\kappa_a$. The Z-depended corrugation given by Tersoff-Hamann is $$\Delta(Z)\approx \frac{2}{\kappa}e^{-\frac{\pi^2Z}{a2\kappa}}$$ where a is the lattice constant and $\kappa$ the inverse decay length. To make both a flat looking surface one gets the expression
$$\kappa_v=\kappa_a-\frac{2\pi^2}{\kappa a^2}e^{-\frac{\pi^2\bar Z}{a2\kappa}}$$ 
As the transmission factor changes with the decay length, the tunneling current and with it the $dI/dV$ changes. This is the origin of topographic features in $dI/dV$ maps when recorded at constant current.

The origin of the strongly voltage depended background can be found in WKB theory as well.
When writing the tunneling current as 
$$ I=\int_0^{eV}\rho_s(r,E)\rho_t(r,eV+E)exp\left(-\frac{\textcolor{red}{\textbf{2}}Z\sqrt{2m}}{\hbar}\sqrt{\frac{\Phi_s+\Phi_t}{2}+\frac{eV}{2}-E}\right)dE $$
the tunneling current reduces to 
\begin{equation}
\bar I=\rho_s\rho_t \bar V exp\left(-\frac{\textcolor{red}{\textbf{2}}\sqrt{2m}}{\hbar}\sqrt{\Phi}Z\right)
\label{tc}
\end{equation}
Assuming that DOS of tip and sample $\rho_t/\rho_s$ are constant, as well as discarding the change of the tunneling barrier with the bias voltage(an assumption only valid for very small voltages with $eV<<\Phi$) the derivative of \eqref{tc} is given by
$$\frac{dI}{dV}=e\rho_s\rho_texp\left(-\frac{\textcolor{red}{\textbf{2}}\sqrt{2m}}{\hbar}\sqrt{\Phi-\frac{eV}{2}}\right)Z$$
Substituting Z with the one obtained by \eqref{tc} leads to $dI/dV= \bar I / \bar V$ - which diverges as  $1/V$ when going to very low bias voltages and gives another contribution to the background. This makes it hard to observe features in close proximity to the fermi level ($V_{bias}=\SI{0}{\volt}$). This background can be reduced when operating at constant tunneling resistance and not at constant current. When doing this, features usually obscured by the $1/V$ diverging background can be observed.\footnote{A comprehensive overview on measurement technique and analysis can be found in \cite{bonnell_scanning_1993}. For information on normalization of STS and to reduce the background close to $E_F$, see \cite{feenstra_tunneling_1987}.}

\index{STS!Bias above work function}
If the bias voltage is higher than the work function of the sample $dI/dV$ reflects mainly states that arise from interaction of electrons at the surface with the polarization they induce in the bulk. Electrons are trapped by this interaction in a region near the surface leaving their lateral movement undistorted. These waves either do interfere con- or deconstructively at the surface. Which type of interference occurs is determined by the applied bias voltage that alternates the bounding condition. The transmission alternates when going from constructive to destructive interference and therefore the tunneling current changes when changing V. 
As an interesting fact, Becker et al.\cite{becker_electron_1985} found that that numerical integration of Schr\"odingers equation could be used together with $dI/dV$ spectra to calculate the absolute distance between tip and sample - an value hard to come by with other methods.

\index{STS!Barrier Height}
Further information can be drawn from the tunneling system when the barrier height may be determined.
Taking the limit of the transmission function \eqref{Transmission-function} for low bias voltage ($eV\approx0$, $E=E_F$) results in 
$$T=exp\left(-\frac{2Z\sqrt{2m}}{\hbar}\sqrt{\frac{\Phi_s+\Phi_t}{2}}\right)$$
Using this in the WKB approximation \eqref{WKB}, one gets $$\frac{dI/dZ}{I}=\frac{2\sqrt{2m}}{\hbar}\sqrt{\Phi_s+\Phi_t}$$
As the work function of the tip usually stays constant, lateral variations in the barrier height can be boiled down to local changes in the work function. This is done by \cite{jia_variation_1998}.

Determining the barrier height in this way often results in to low values for the work function. Discussion of this is found in \cite[96]{bonnell_scanning_1993}.

\index{Gundlach oscillations}
Up to now only rectangular tunneling barriers were considered.
Already in 1966 Gundlach was the first who calculated transmission currents for trapezoidal potential barriers \cite{gundlach_zur_1966}. The oscillations named after him are due to standing wave states in the potential tip-sample potential barrier \cite{binnig_tunneling_1985,becker_electron_1985}

``When the Fermi level of the tip is close to the vacuum level of  the  sample,  the  contribution  of  the  image  potential  is significant. The superposition of the  image  potential  and the electrostatic  potential forms a specific potential well, and the lowest-order peak is a Gundlach oscillation related to a standing-wave state in this well. When the Fermi level of the tip is higher than the vacuum level of the sample, the image potential becomes negligible, and the potential well can be  approximated  by a triangular  shape. Those peaks beyond the lowest-order peak are the Gundlach oscillations related to the standing-wave states in the triangular well. Derivation  based  on  quantum  mechanics  shows  that  the energy difference of the standing-wave states in the triangular  well  is  proportional  to $F^{2/3}$,  where F is  the electric field in the tip-sample gap''\cite{lin_manifestation_2007}

\index{STS!Resolution}
The resolution of STS is determined by the range of energies electrons have when contributing to the tunneling process. When $T>0$ the DOS is smeared out and described by the Fermi-Dirac statistic\cite{fermi_zur_1926, dirac_theory_1926} $$f(E)=\frac{1}{1+exp\left(\frac{E-E_F}{k_BT}\right)}$$ 
Since electrons from occupied states (DOS is Fermi distributed) tunnel into unoccupied states the transmission function has the structure $$T(E,eV,T)=T(E,eV)f(E)[1-f(eV-E)]$$ 
When looking at the shape of the Fermi-Dirac distribution one can see that most of the electrons participating in the tunneling process arise from a rather narrow area around the Fermi level of the negatively biased electrode (broadening of fermi edge at $T=300K\,\hat=\SI{0.026}{\eV}$. Electron distribution of tip and sample are broadened by $2 k_b T=\SI{0.054}{\eV}$ thus the energetic range where electrons may come from is \SI{0.1}{\eV}. From the uncertainty relation $\Delta x \Delta k \geq 1/2$ and the dispersion relation for metals follows $$ \Delta E\ge \frac{\hbar^2k_F}{2M^*\Delta x}=0.47\ \frac{E_F-E_0}{rk_F} $$\cite{chen_introduction_2008}. ``The asymmetric form of $T(E,eV)$, with the sharp increase at $E_F$, helps to make the effective resolution of the STM somewhat higher when probing empty states of the sample than when probing filled states.''
The resolution at room temperature is estimated to be \SI{140}{\m\eV}\cite{hansma_tunneling_1982}.
As the tunneling transmission is always a factor of the tip and sample DOS, STS is always limited to the unknown electronic structure of the tip. While geometry at the tip apex is successfully enhanced with field evaporation techniques its electronic structure may differ greatly from the bulk one due to unusual bonding geometry and small size.\footnote{ Some\cite{tersoff_role_1990,ciraci_tip-sample_1990,lawunmi_theoretical_1990,kobayashi_simulation_1990} groups have calculated band structures for different tip geometries and their influence on the tunneling process. \label{section:AFM-resolution}}
\subsection{...and how an image is created}
\subsection{Experimental details and machine description}

Since all used UHV chambers have many common part, a typical setup is described with the LT-STM setup. Here the most experiments were carried out.

\paragraph{Vacuum system}
\paragraph{Cooling system} 
While low temperature (LT) STMs may be operated with solely helium, it is more resource-saving to cool the direct proximity of the sample and the STM with He, but to suppress the heat flow out of the He-cryostat with a second surrounding nitrogen cryostat (boiling point: \SI{77}{\K}, compare figure \ref{fig:STM-cryo}). This diminishes consumption of globally limited and rather expensive He. To maintain a temperature of \SIrange{5}{7}{\K}, one to two liters of liquid helium are required a day, plus an additional amount of three to four liters liquid nitrogen. Evaporated helium is reclaimed in a closed circuit with a system of purifying and storage/cooling steps so that only a small amount of helium escapes the circuit and is lost.

Sample temperatures down to \SIrange{5}{7}{\K} allow for observations not possible at elevated (room) temperature. Cooling not only reduces thermal drift in the piezo elements that are used to control the tip's position on the sample. Thermal energy at low temperature is not high enough for atoms or molecules to move on most substrates. Species mobile at room temperature (and therefor not representable at room temperature in the sub-ML regime) become immobile and accessible for ST microscopy and spectroscopy. STS spectra resolution is better at low temperatures as discussed in \ref{section:AFM-resolution}.

\begin{figure}[ht]\centering
	\subfigure[LT-STM setup mainly used in this work. Different functional groups are colored in different colors. A low base pressure in achieved with a combined pumping system comprised of ion pumps and turbo molecular pumps (cyan). The liquid helium/nitrogen bath cryostat (red) is used to maintain low temperatures. Sample holders are operated with a rote able, variable temperature manipulator. Sample preparation is done in a chamber (blue). After transfer to the LT-STM chamber (yellow) a gate valve is used to seal the LT-STM from remaining residual gas that may be present in the preparation chamber. Vibration isolation of the frame is achieved with legs floating on pressurized cylinders.]{
		\includegraphics[width=0.45\textwidth]{./images/chamber-sketch.jpg}
		\label{fig:chamber-sketch}
	} \quad
	\subfigure[Scheme of a STM liquid bath cryostat. While in the inner measurement stage a temperature of $\approx \SIrange{5}{7}{\K}$ is achieved with a liquid helium reservoir, an outer liquid nitrogen cryostat is used to isolate the evacuated inner cryostat from the surrounding room temperature.]{
		\includegraphics[width=0.45\textwidth]{./images/sketch-cryo.jpg}
		\label{fig:STM-cryo}
	}
	\caption{Typical setup for low temperature measurements. A vibration isolated UHV chamber is used to prepare samples and investigate them in a separable chamber with either STM or AFM. A liquid bath cryostat is used to maintain low temperatures. Images stem from {diss-knud}}
	\label{fig:STM}
\end{figure}

\paragraph{Damping stages}
Two damping stages are used, one for the chamber and a separate one for the STM.
\begin{itemize}
	\item The whole UHV system is placed on air pressurized legs. These can be elevated on demand, so that the chamber floats on four dampers and external vibrations/shocks are damped.
	\item A second stage decouples the sensitive STM from the rest of the setup. First the complete STM stage hangs on springs to further limit the direct influence of vibrations. Second the remaining oscillation amplitude is damped by a eddy current damping. It is made of three magnets in close proximity to the surrounding support so that eddy currents are induced for each minute movement. The eddy current is typically larger at cryogenic temperatures, that results in a damping that works best at low temperatures. The kinetic energy of the oscillating system is transferred by the eddy currents into heat within the surrounding conductor. The heat is then mitigated by the external cooling of the cryostat.
\end{itemize}

\paragraph{Piezo elements}
\begin{wrapfigure}{O}{5cm} \centering
	\includegraphics[width=5cm]{./images/STM-sketch-2}
	\caption{STM sample stage to control the tips position. The coarse movement is controlled by exterior piezos. Each move up/down on a heliocoidal ramp with slip-stick motion. The precise scanning is done with a central piezo to which the tip is attached. Taken from}
	\label{fig:stm-heliocoidal ramp}
\end{wrapfigure}
The position of the tip (x, y, z) is controlled with a set of piezos (see \autoref{fig:STM-tip}). In this work a tubular piezo stack is used to control the tips position with a central piezo element located on top of the tip. The piezo length can be controlled with the voltage applied to them, which is used to choose not only the tip-sample distance, but other parameters like image size and scan speed as well. All of these parameters are monitored with the STM software. A feedback loop controls the piezo voltages. For recording an image the area is raster scanned in consecutive lines, applying a sawtooth voltage to the fast scan direction. The next lines are chosen by slowly increasing the voltage along the slow scan direction. Depending on the operating mode the tip-sample distance is controlled by piezo elements, too.

%\begin{figure}[ht]
%	\begin{center}
%	\includegraphics[width=0.45\textwidth]{./images/STM-sketch}
%	\end{center}
%	\caption{Taken from \cite{diss-manuela}}
%\end{figure}

\begin{figure}\centering
	\subfigure[]{\includegraphics[width=0.6\textwidth]{./images/STM-rutgers-modified}\label{fig:STM-tip}}
	\subfigure[]{\includegraphics[height=0.33\textwidth]{./images/STM-sketch-cut.jpg}\label{fig:STM-modes}}
	\caption{Operating principles of an STM. \subref{fig:STM-tip} A macroscopic sketch  shows the central piezo that controls the tip position above the sample. A microscopic sketch shows the tips movement in constant current mode while moving across a atomic step edge. The main piezo is divided in four parts to control movement in the x-y plane and tip-sample distance\cite{STM-rutgers}. \subref{fig:STM-modes} The difference between constant current and constant height mode. While the tip follows the samples LDOS in constant current mode (top) the tip height remains constant in constant height mode (bottom). Taken from \cite{diss-manuela}.}
	\label{fig:STM-sketch}
\end{figure}


There are two common ways to operate an STM as shown in \autoref{fig:STM-modes}.
The \textbf{constant current mode}\index{STM:operating mode} is the most widely used one. The tip height is regulated with a feedback controller to achieve a constant tunneling current for the chosen bias. The recorded information is now the voltage applied to the z-piezo to maintain a plane with the same current. Sample features that increase the tunneling current cause the feedback to decrease it again by retracting the tip.
In \textbf{constant height mode} the tip has always the same absolute height (no feedback control), but as the tip-sample distance changes the tunneling current varies, which then is the measured quantity.
To avoid crashes when the sample is very irregular, many STM's are operated in constant current mode. All STM images in this work are recorded in constant current mode.
The current recorded in a certain area of the sample is translated into a contrast variation on a color scale. While some images encourage the operator to interpret points with high intensity as elevated atoms it is not that trivial. Tunneling current between tip and sample depends on the LDOS of tip and surface and is therefore not implicitly maximized at the atomic positions. It may also vary with the bias voltage applied in a non-trivial manner. Investigation of this behavior led to the establishment of a new measurement technique, called scanning tunneling spectroscopy (see \autoref{section:STS}). 

\subsection{Limitations}\index{STM:resolution}The accuracy of a STM is very high with spatial resolution down to the atomic scale. Due to the fact that the tips motion is controlled with different piezos, one has to take different elongations in different directions into account. For example, if the STM scans the fast scanning direction just a bit further than the slow scan direction, the resulting image (although pixel wise square) is no longer physically square anymore. Imagine a square (1:1 side ratio, diagonal angle 45\textdegree) where one side is elongated by 5\%. The resulting square (1:1.05 side ratio, diagonal angle 43.6\textdegree) looks square because it has the equal number of pixels in both directions, but it is physically rectangular. The expression used to calculate the uncertainty with known calibration parameters is
$$\Delta \Theta = 45 - \frac{180}{\pi}\cdot\arctan(\frac{1}{1+x})$$ where x is the percentage of one side being longer. This results in an uncertainty of 0.3\textdegree(1\%), 1.4\textdegree(5\%, see example above), 2.7\textdegree(10\%). For moderate shear, conformity is almost conserved and the uncertainty below 2\textdegree.

Mechanical and thermal vibrations limit the resolution of the STM, too. Therefor several damping stages decouple the STM from the surrounding. Although STM works at room temperature, additional cooling may be applied to reduce the thermal vibrations.

Because STM is sensible to electronic changes, it may change the footprint of an adsorbed compound \cite{sautet_interpretation_1992}. When laterally approaching an adsorbate this results in an additional tunneling current, because now electrons do not only tunnel directly into the substrate but through the adsorbate as well. Interferences between both tunneling processes depend on the adsorbate's orbital-symmetry and tip-shape. Local density of states calculations \cite{tersoff_theory_1985, lang_theory_1986, eigler_imaging_1991} is not adapted to grasp this effect since the tip is considered far away from the surface. Moreover, the tip radius or the tip-substrate distance is optimized to fit the lateral size of the adsorbate print with the experimental image \cite{tersoff_theory_1985, eigler_imaging_1991}.
%  \section{Experimental setup}
 %   STMs may be operated at very different temperatures. The lowest temperature is only limited by the availability of sufficient cooling. While low temperature (LT) STMs may be operated with solely helium, it is more resource-saving to cool the direct proximity of the sample and the STM with He, but to suppress the heat flow out of the He-cryostat with a second surrounding nitrogen cryostat (boiling point: \SI{77}{\K}, compare figure \ref{fig:STM-cryo}). This diminishes consumption of globally limited and rather expensive He. To maintain a temperature of \SIrange{5}{7}{\K}, one to two liters of liquid helium are required a day, plus an additional amount of three to four liters liquid nitrogen. Evaporated helium is reclaimed in a closed circuit with a system of purifying and storage/cooling steps so that only a small amount of helium escapes the circuit and is lost.

Sample temperatures down to \SIrange{5}{7}{\K} allow for observations not possible at elevated (room) temperature. While cooling not only reduces thermal drift in the piezo elements that are used to determine the tip's position on the sample. Thermal energy at low temperature is not high enough for atoms or molecules to move on the surface. Species mobile at room temperature (and therefor not representable at room temperature) are now immobile and accessible for STM imaging and spectroscopy.

\begin{figure}[ht]
	\begin{center}
		\subfigure[LT-STM setup]{
			\includegraphics[width=0.45\textwidth]{./images/chamber-sketch.jpg}
			\label{fig:STM-modes}
		}
		\subfigure[Sheme of STM liquid bath cryostats]{
			\includegraphics[width=0.45\textwidth]{./images/sketch-cryo.jpg}
			\label{fig:STM-cryo}
		}
	\end{center}
	\caption{Taken from \cite{diss-manuela} and \cite{diss-knud}}
	\label{fig:STM}
\end{figure}
  \section{\textbf{X}-ray \textbf{P}hotoelectron \textbf{S}pectroscopy}
	\label{section:XPS}\index{XPS} XPS is a tool to achieve information of the samples chemical structure.
Most of the information given here is taken from \cite{Riviere_90} in \cite{briggs_auger_1990}.
When X-rays with sufficient energy hit metals, electrons are emitted. This effect is called photoelectric effect and was first discovered by Heinrich Hertz in 1887 through the fact that electrodes illuminated with ultraviolet light create electric sparks more easily\cite{hertz_ueber_1887}. 18 years later Albert Einstein received the Nobel Price for his discovery of the law of the photoelectric effect\cite{_nobel_2015} and a scientific explanation which Hertz was missing.

The standard X-ray source is supplied with aluminum and magnesium anodes. Other materials are available that produce various X-ray energies and line widths \index{XPS!Anode materials} \cite{_x-ray_2015}. 
\begin{table}\caption{Energy and line widths of available anode materials. Taken from }
	\centering
	\begin{tabular}{cccc}
		Anode 	& 	Radiation 	& Photon Energy (eV) 	& Line Width (eV) \\ \hline
		Mg	&	K$\alpha$ 	&	1253.6	&	0.7\\
		Al	&	K$\alpha$ 	&	1486.6 	&	0.85\\
		Zr 	&	L$\alpha$ 	&	2042.4 	&	1.6\\
		Ag 	&	L$\alpha$ 	&	2984.3 	&	2.6\\
		Ti 	&	K$\alpha$ 	&	4510.9 	&	2.0\\
		Cr	&	K$\alpha$ 	&	5417 	&	2.1\\
	\end{tabular}
\end{table}

\index{XPS!Physical model}As the X-rays hit and penetrate the sample surface they excite electrons and initiate different processes. The meajor two are discussed in the following.

For the \textbf{simple core-level excitation} the X-ray removes a single electron next to the core which is then detected. Energy conservation due to elastic scattering of the electron out of the bulk results in the relation 
\begin{align}
E_{kin} &= h\nu_{\textnormal{X-ray}}-E_{B}-\Phi_{\textnormal{bulk}} \\
E_B 	&=h\nu_{\textnormal{X-ray}}-E_{kin}-\Phi_{\textnormal{bulk}}
\end{align}
 $h\nu_{\textnormal{X-ray}}$ is the energy of the incident X-ray beam, $E_B$ the binding energy of the excited electron and $\Phi_{bulk}$ the work function of the analyzer. 
 
 In the \textbf{Auger process} \index{XPS!Auger process} on the other hand the created core level (lets say at level K) vacancy is filled with a energetically higher lying electron (at for example level $L_1$). The excess energy can either be radiated away (X-ray fluorescence) or given to an electron in the same or in a more shallow level (lets say $L_{2,3}$). This electron then can leave the sample as Auger electron. Figure \ref{fig:XPS-auger} shows the mentioned process. 
 
 Due to the fact that this process has two stages these electrons are referred to as secondary electrons. The notation for this process is $KL_1L_{2,3}$. The electron taking part in the Auger process can also originate from within the valence band ($KL_{2,3}V$) or even both electrons may stem from the valence band ($KVV$). For every element there is a unique series of Auger excitations. This holds even if one or even both electrons come from a valence band, as the dominating term will always be the binding energy $E_K$. As the Auger energy $$E_{KL_1L_{2,3}}=E_K-E_{L_1}-E_{L_{2,3}}$$ is not a function of the excitation energy it will not shift when changing the X-ray energy. Even very heavy elements (as the number of atomic levels the possible Auger transitions increases) do not exhibit a very large number of Auger emission lines due to the fact that the transition probability favors only a few of this many. \cite{Briggs_90}

\begin{figure}\centering
	\subfigure[Representation of XPS process. X-rays are used to excite core level electrons. After leaving the sample these show element specific signatures in their kinetic energies. A detailed analysis of the peak shape and shift allows for identification of the chemical environment.]{\includegraphics[width=0.45\textwidth]{./images/XPS_PHYSICS}
		\label{fig:XPS-excitation}}
	%	\subfigure[Single step core electron emission after excitation with X-rays leads to detection of core electron binding energy. \cite{_whatisxps-04.jpg_2015}]{\includegraphics[height=0.3\textwidth]{./images/whatisxps-04.jpg}
	%		\label{fig:XPS-analyzer}}
	%	
	\subfigure[Two step Auger process. In contrast to a direct emission of a core electron, the Auger emission involves more  electrons from less bound states. Here an ($L_{1}$) electron fills the hole in the $K$ shell created by the initial X-ray. The resulting hole in the $L_1$ shell is then filled with an $L_{2,3}$ electron resulting in a typical Auger emission feature. Adopted from \cite{Briggs_90}.]{\includegraphics[width=0.45\textwidth]{./images/auger.jpg}
		\label{fig:XPS-auger}}
	
	\caption{XPS and Auger processes present after irradiation of the sample with X-rays. Two simultaneous processes occur. \ref{fig:XPS-excitation} shows a scheme of an X-ray gun illuminating a sample area of about \SI{0.4}{\milli \meter} x \SI{0.6}{\milli \meter}. Excited electrons within the first 20 layer escape the sample and a analyzed for their kinetic energy. Core electrons and Auger electrons are excited at the same time resulting in a chemical fingerprint of the sample.}
	\label{fig:auger-core}
\end{figure}

\cite{zemlyanov_versatile_2018}
The \index{XPS!chemical surrounding} chemical surrounding of atoms changes their binding energy, making XPS an ideal tool to detect changes in chemical surrounding. Although the analysis is averaged over the area of the incident X-rays (typically profiliate of microns up to several mm) its results are very precise. This makes it possible to distinguish differently bound atoms within single atomic species and therefore gives rise to otherwise not directly observable processes like growth, intercalation, etching and binding of for example graphene islands on Ir(111)\cite{busse_graphene_2011-1,granas_oxygen_2012}. XPS is used to identify oxidation processes of copper surfaces as the interaction of oxygen with the copper surface changes the Cu core level. \cite{deroubaix_x-ray_1992}.

With the use of aluminum X-ray sources, electrons are accelerated with typically \SI{15}{\keV} onto the target. Most of the created radiation goes into the principal characteristic line ($K\alpha_{1,2}$). Higher ones ($K\alpha_{3,4}$, $K\beta$) are also observed but with much lower intensities. In addition there is a continuous background called Bremsstrahlung extending up to the energy of the incident electron energy. This background is of no use for the XPS measurement and has to be subtracted in a more or less artificial way.

For the ease of analysis, many spectra are recorded with monochromatic radiation. This selects a certain energy for the following illumination of the sample. This technique relies on the dispersion of X-rays within a crystal. It is described by the Bragg relation $n\lambda = 2d\sin(\theta)$ where n is the diffraction order, $\lambda$ the wavelength of monochromatic radiation, d the distance between two crystal layers and $\theta$ is the so called Bragg angle. The first order peak for Al $K\alpha$ radiation ($\lambda=\SI{0,83}{\nm}$, $E=\SI{1486,6}{\eV}$) is at $\theta=\SI{78.5}{\degree}$ (using the $10\bar10$ planes of a quartz crystal with $d=\SI{0,425}{\nm}$). Therefore the angle between incident and reflected beam is about \SI{23}{\degree}.\cite{Riviere_90}

The spectra used in this work are recorded without monochromatic radiation as not stated otherwise. This is because the electron intensity is attenuated when using a monochromatic source.

The \index{XPS!binding energies} binding energies of some often observed peaks are given in table \ref{tab:XPS-intensities}
\begin{table}\centering
 \caption{Element specific transitions and binding energies for some chosen elements as reported in \cite{wanger_handbook_1979}}
 \begin{tabular}{lll}
  Eelement & excited state & $E_B$ [eV]\\ \hline 
  O & 1s & 531\\
  N & 1s & 398.1\\
  C & 1s & 285\\
  B & 1s & 189.4 \\
  Cu & 2p $\frac{1}{2} (\frac{3}{2})$ & 953 (933) \\
  Cu & LMM & 560-580 \\
  Cu & 3s & 123\\
  Cu & 3p $\frac{1}{2} (\frac{3}{2})$ & 77 (75)\\
 \end{tabular}
\label{tab:XPS-intensities}
\end{table}

\paragraph{Peak shapes}\index{XPS!Peak shapes}
The shape of the peaks typically resembles the line shape of the used X-rays (Gauss width $\approx 1eV$). In case of s-states $(l=0)$ (B1s, N1s, C1s) the peaks are singlets. With increasing $j=l+s$, the spin-orbit (j-j) coupling introduces a 'parallel' and 'anti-parallel' nature of the spin, resulting in two different $j=1/2(3/2)$ and therefore two different energies. The split in energy is expected to increase as Z increases (for constant n,l) or as l decreases (n constant). This makes the splitting of 3p orbitals larger than that of the 3d's. The ratio of the two peaks is given by their degeneracy $(2j+1)$.\cite[113]{Riviere_90}
\begin{table}
\caption{Spin-orbit splitting parameters. Ratios are reproduced from \cite{Riviere_90}}
\centering
 \begin{tabular}{ccc}
 Subshell & j values & Area ratio \\ \hline
 s & $\frac{1}{2}$ & --- \\
 p & $\frac{1}{2}$,$\frac{3}{2}$ & 1:2 \\
 d & $\frac{3}{2}$,$\frac{5}{2}$ & 2:3 \\
 f & $\frac{5}{2}$,$\frac{7}{2}$ & 3:4 \\
 \end{tabular}
\end{table}

\index{XPS!quantitative analysis}
The more atoms of a specific kind are present, the larger the signal gets. Therefore the signal intensity resembles the amount of atoms on the topmost surface layers($\approx \SI{10}{\nm}$).

As each irradiated atomic species has a different cross section for adsorption of X-rays with a certain energy they emit auger and core level spectra with a different intensity. Comparing the cross section of e.g. N with B's, one can see that it it roughly \SIrange{3}{4}{times} as large (B: \SI{6,87e3}{\barn\per atom}, N: \SI{25,82e3}{\barn\per atom} for $\textnormal{Al} K_{\alpha}$)\cite{henke_x-ray_1993}. Meaning that the signal from the N is much stronger than that of the B, although their number of atoms is equal.

A calibrated XPS is capable of measuring the surface coverage of an ad-layer (e.g. BN) compared to the bulk of the sample. Calibration works as follows:
\begin{itemize}
 \item A perfect, full layer of a known material (e.g. C) is prepared (graphene)
 \item The known cross section for C (\SI{6,87e3}{\barn\per atom})\cite{henke_x-ray_1993} relates the signal to the number of atoms in the full layer.
 \item One then has to divide the signal of the unknown coverage (of known material) by the cross section (of C) and directly receives the coverage. Keep in mind that the X-ray penetration depth (and with it the signal of the substrate) stays only constant if the illumination angle stays constant. Even small angular variations may change the signal.
\end{itemize}
Referring to \cite{ertl_low_1986} the fraction $\theta_A$ of an adsorbate A on a surface B can be calculated via
\begin{equation}\label{eq:adlayer-coverage}
 \theta_A=\frac{I_AI_B^0\cdot \exp(\frac{a_A}{\lambda_A}\cos(\Theta))}{I_AI_B^0( \exp(\frac{a_A}{\lambda_A}\cos(\Theta))-1)+I_BI_A^0\cdot \exp(\frac{a_A}{\lambda_A}\cos(\Theta))}
\end{equation}
where the parameters are given in table \ref{tab:adlayer-coverage-parameters}.

\begin{table}[h!]\centering
\caption{Description of parameters used in equation
\ref{eq:adlayer-coverage}}
\label{tab:adlayer-coverage-parameters}
 \begin{tabular}{cc}
  Parameter & Annotation \\ \hline
  $I_A$	& integrated intensity of the adsorbate peak \\
  $I_A^0$ & cross section of element A \\
  $\lambda_A$ & mean free path of electrons in material A \\
  $a_A$ & thickness of adlayer \\
 \end{tabular}
\end{table}

The mean free path of electrons with energy E in a solid is given by $\lambda_M=\SI{0,41}{}\cdot a_M^{\SI{0,41}{}}\cdot E_M^{\SI{0,5}{}} $ where $a_M$ is the atomic size of M. 
  \section{\textbf{A}tomic \textbf{F}orce \textbf{M}icroscopy}
	Introducing STM as a electronically sensitive method to investigate surfaces, the atomic force microscopy interacts in a different way with the sample.
To scan the surface of the sample, one uses an small tip (cantilever) to literally interact with it. Due to its small distance different kinds of forces occur. The force induced movement of the tip is then interpreted an information on the surface may be derived.
The layout of a typical \index{AFM} consists of the cantilever itself and some kind of deflection measurement, typically made of the position-sensitive detector (PSD) consisting of two closely spaced photo diodes whose output signal is collected by a differential amplifier.

\begin{figure}\centering
	\subfigure[Photograph of the tuning fork and cantilever. From \cite{he_bottom-up_2017}]{\includegraphics[width=0.6\textwidth]{./images/AFM-qplus-photograph}
		\label{fig:AFM-cantilever}}	
	\subfigure[QPlus sensor used to excite oscillations of the cantilever with controlled amplitude and frequency. Simultanious excitation of the tuning fork ($S_-$ and $S_+$) and measurement of the tunneling current ($I_t$) is possible. From \cite{AFM-qplus}]{\includegraphics[width=0.3\textwidth]{./images/200px-QPlusSchematic}
		\label{fig:AFM-qplus}}
	
	\caption{Photograph \ref{fig:AFM-cantilever} and sketch \ref{fig:AFM-qplus} of AFM cantilever as used in the used setup. }
	\label{fig:AFM-tuning-fork}
\end{figure}	


\begin{figure}\centering
	\subfigure[Schematic representation of cantilever tip and atomically flat sample surface. The force between both $F_{TS}$ is indicated with an arrow. The attractive/repulsive force regimes are shown for variing tip-sample distances. From \cite{pavlicek_generation_2017}]{\includegraphics[width=0.6\textwidth]{./images/s41570-016-0005-f1}%
		\label{fig:AFM-force}}
	\subfigure[Enlarged sheme of AFM tip and top most sample layer. CO functionalization of the tip leads to an decreased tip apex that increases resolution. Modified from \cite{AFM-qplus}. See text for further discussion.]{\includegraphics[width=0.3\textwidth]{./images/AFM_tip_with_CO-functionalization-mod}
		\label{fig:AFM-CO}}
	
	\caption{Basic function of a typical qplus AFM. \ref{fig:AFM-force} Sketch of tip-sample interaction together with force-distance graph. The CO functionalization of the metalic AFM tip is shown in \ref{fig:AFM-CO}}%
	\label{fig:AFM-sketch}%
\end{figure}%

When the tip moves above the surface it interacts with it due to different forces. These are typically:
\begin{itemize}
 \item van-der-waals interaction
 \item mechanical contact force
 \item capillary forces, chemical bonding, electrostatic forces, Casimir forces, solvation forces and so on
\end{itemize}

The typical resulting force between tip and sample is shown in figure \ref{fig:AFM-force}.
A magnetic tip can be used to scan for magnetic forces on the sample surface.

There are several operational modes to drive an AFM:
\begin{enumerate}
 \item \textbf{Contact (static) mode}: The cantilever is operated in contact with the sample surface. At very close proximity, repulsive forces are stronger than the attracting ones. The signal used to gain information on the sample is either the feedback loop to keep the tip at the same absolute position or directly the deflection of the cantilever.
 \item \textbf{Intermittend contact mode (tapping) mode}: While the contact mode has some disadvantages when scanning samples with an adsorbat layer (tip sticks to surface when close ehough to measure short-range forces) another mode has established. Hereby the tip oscillates with amplitudes in the \SIrange{100}{200}{\nm} regime and is not dragged the whole way across the sample. The intermittend forces acting on the tip when reaching the sample are measured. The change in amplitude when in close vicinity to the surface is a sign of the actual height.
 \item \textbf{Non-contact mode}: The cantilever is driven at its resonance frequency with amplitudes smaller than \SI{10}{\nm} and at a certain distance to the sample. Long-range forces like van-der-waals and others change the resonance frequency of the cantilever. This change is a indication of the acting force between cantilever and sample.
\end{enumerate}

\begin{itemize}
 \item AFM produces a true height-profile of the sample (and not a projection of the surface onto a 2D-map like in STM)
 \item Works in ambient pressure and even in liquids
 \item It has limited resolution especially when scanning features steeper than the tip apex
\end{itemize}

Its advantages are the comparable large image size of many hundred \si{\nm} compared to only some dozen \si{\nm} in the case of STM. Scan speeds are typically some orders of magnitude larger than those in STM so that image acquision is much faster.

To increase the resolution the tip can be functionalized with CO (see figure \ref{fig:AFM-CO}). This method is widely used \cite{kawai_multiple_2018, kawai_atomically_2015, schulz_elemental_2018, gross_chemical_2009} to investigate not only geometric features that are not accessible in STM, but also chemical differences in the sample. 
%%%%%%%%%%%%%%%%%%%%%%%%%%%%%%%%%%%%%%%%%%%%
\chapter{Substrates, ad layers and sample preparation}
To conduct experiments in the most controlled environment possible, sample preparation and investigation is done in ultrahigh vacuum (UHV) chambers. With a base pressure of $\approx \SI{5e-10}{\milli \bar}$ these stainless steel chambers are almost free of contaminants. Since the rate at which residual gas contaminates the sample is proportional to the residual partial pressure, the time needed to cover the surface is proportional to it. When the base pressure is lowered by a factor of \SI{1e-13} (\SI{1}{\bar} $\rightarrow$ \SI{1e-10}{\milli \bar}), the time needed to cover the sample is increased by a factor \SI{e13}. 

Experiments were done at different experimental chambers: (1) A LT-STM chamber. \footnote{The technical aspects of the experimental chamber are discussed elsewhere \cite{urgel_tendero_two-dimensional_2015, schwarz_assembly_2018, wiengarten_scanning_2015} and shortly summarized here.} (2) A RT-STM with XPS capabilities. The technical aspects and build details are given in \cite{schwarz_assembly_2018}. (3) The NIM-XPS chamber. 
%This chamber is chosen because of it is possible to investigate chemical properties and coverage with XPS while an UV-Vis spectrometer is used to access the electronic structure of the occupied states. 
(4) A LT-AFM machine offers a complementary investigation method for geometric information. (5) an ex-situ AFM and (6) a SEM.


  \section{Substrates and ad-layers}
Experiments are mainly done with single crystalline substrates of copper and silver, terminated at (111) and (100) surfaces. Polycrystalline copper foils and gold evaporated on mica are used as well. Experiments are done to investigate molecules on \textit{h}-BN/Cu(111) and \textit{h}-BN/Cu(foil), as well.

The following sections will describe the relevant physical properties of the various substrates and will introduce a method to prepare polycrystalline copper foils to grow \textit{h}-BN on. The geometric properties of the resulting lattice mismatched system are discussed briefly. The molecules used for experiments are introduced at the end of the chapter.
     \subsection{Single crystal substrates}
        Single crystals show a nicely ordered, clean surface - two properties important for reliable and reproducible experiments. We have chosen both silver and copper as bulk crystalline substrates. Both form fcc lattices and their surface termination can be chosen by precise cutting along a symmetry plane of choice. For the course of this thesis, experiments are conducted mainly on (111) and (100) terminated surfaces.\footnote{See \cite{riemann_ionic_2002} and appendix \fullref{appendix:crystal-facets} for another examples of vicinal metal surfaces (531), (532), (221), (311), (211).} Commercially available, single crystals guarantee a high precision in facet orientation and purity (99.999 \%) \cite{mateck}. Remaining contaminations in copper (Ag: \SI{0.8}{ppm}, Pb: \SI{0.3}{ppm}, Bi: \SI{0.8}{ppm}) and silver (Cu: \SI{2}{ppm}, Fe: \SI{2}{ppm}, Au: \SI{0.8}{ppm}, Ni: \SI{0.8}{ppm}) are removed by repeated sputter\footnote{$U_{accel}=\SIrange{800}{1000}{\volt}$, $T_{sample}\approx \SI{300}{\kelvin}$} and anneal cycles\footnote{Cu: $T_{sample}=\SI{750}{\celsius}$, Ag: $T_{sample}=\SI{450}{\celsius}$, Au\/Mica: $T_{sample}= \underline{\textbf{get value: 450?}}$} in UHV. Typical cool down temperatures $\leq 5 \frac{K}{s}$ result in a smooth, atomically flat surface with large terrace size. 

The lattice constants at room temperatures for \underline{\textbf{cite!}} Cu(\SI{3,61}{\angstrom}), Ag(\SI{4,09}{\angstrom}) and Au(\SI{4,07}{\angstrom}) are related to the environment temperatures by their expansion coefficients.
Coefficients of \SI{16,5e-6}{\per \kelvin}(Cu), \SI{18,9e-6}{\per \kelvin}(Ag) and \SI{14,2e-6}{\per \kelvin}(Au) make the substrate lattice shrink by $\approx \SI{0,5}{\percent}$ when it is cooled down from RT to low temperature measurement conditions in STM/AFM (\SIrange{5}{7}{\kelvin}). While rather negligible for bulk materials that are not heated and cooled over larger temperature ranges, the small change in substrate lattice size may introduce strain in grown ad layers since these are grown via CVD typically at elevated temperatures and may have partially negative thermal expansion coefficients\cite{farwick_zum_hagen_structure_2016}.

%\begin{table}
%\centering \index{Crystal:lattice constants}
%\caption{Inter atomic distances for Cu and Ag with respect to different surface termination. $a$ denotes the lattice constant and $\beta= \SI{60}{\deg}$ the angle within the (111) unit cell}
%  \begin{tabular}{ccccc}
%& Lattice constant a [\SI{}{\angstrom}] & Nearest neighbors [\SI{}{\angstrom}] & diagonal [\SI{}{\angstrom}]\\ \hline 
%\multicolumn{2}{c}{fcc(100)} & $\frac{\sqrt{2}a}{2}$ & a \\
%  Cu	 	& 3.61	& 2.55 | 2.55 & 3.61  \\
%  Ag		& 4.09	& 2.89 | 2.89 & 4.09 \\ \hline 
%\multicolumn{2}{c}{fcc(111)} & $\frac{\sqrt{2}a}{2} \ <110>$ & $\sqrt{2}a\sin(\frac{\beta}{2})$ | $\sqrt{2}a\cos(\frac{\beta}{2})$\\
%Cu 		& 3.61	& 2.55 | 2.55	& 2.55 | 4.42 \\
%Ag		& 4.09	& 2.89 | 2.89	& 2.89 | 5.01 \\ \hline
%%
%%\multicolumn{2}{c}{fcc(110)} & $\frac{\sqrt{2}a}{2}$ | a & $\sqrt{\frac{3}{2}}a$\\
%%  Cu	 	& 3.61	& 2.55 | 3.61	& 4.42 \\
%%  Ag		& 4.09	& 2.89 | 4.09	& 5.00 \\ \hline 
% \end{tabular}
%\end{table}

\begin{figure}\centering
	\subfigure[(111)]{\includegraphics[width=0.3\textwidth]{./images/fcc-111-persp}} \quad
	\subfigure[(100)]{\includegraphics[width=0.3\textwidth]{./images/fcc-100-persp}}
%	\subfigure[(110)]{\includegraphics[width=0.3\textwidth]{./images/fcc-110-persp}}
	\caption{Identical crystalline balls in fcc lattice configuration. The surface termination is determined by the direction of the intersecting plane (parallel to the paper plane) relative to the lattice.}
	\label{fig:crystal-termination}
\end{figure}

The surface free energy increases from the (111) surface with increasing angle of the (hkl) planes of interest with $$\cos(\phi)=\frac{h+k+l}{\sqrt{3(h^2+k^2+l^2)}}$$ \cite{jian-min_calculation_2004}. Thus, the (111) surface is the one with lowest energy, followed by (110) and (100).

%\begin{table}\centering
%\caption{Crystal properties from \cite[29ff]{riemann_ionic_2002, ma_interplay_2016, liu_oxygen_2014}}
%\label{tab:step-heights}
%\begin{tabular}{cccc}
%			&				& Copper 	 & Silver \\
%\multicolumn{2}{c}{Lattice constant}			& \SI{3.61}{\angstrom} & \SI{4.08}{\angstrom} \\
%\multicolumn{2}{c}{Nearest neighbor}			& \SI{2.55}{\angstrom} & \SI{2.89}{\angstrom} \\ \hline \\
%\multirow{3}{*}{Step height}	& (311) & \SI{4.23}{\angstrom} & \SI{4.78}{\angstrom} \\
%								& (211) & \SI{6.25}{\angstrom} & \SI{7.08}{\angstrom} \\
%								& (221) & \SI{7.66}{\angstrom} & \SI{8.65}{\angstrom} \\
%								& (110) & \SI{1.38}{\angstrom} & \\
%								& (111) & \SI{1}{\angstrom} & \\
%								& (100) & \SI{1.8}{\angstrom} & \\
%
%\end{tabular}
%\end{table}

\subsection{Dislocation lines and crystal orientation}
Due to the fact, that dislocation lines move within the crystal in a well defined manner, one can determine the crystals orientation when the orientation of a dislocation is known.
For fcc crystals the orientation of dislocation lines occurs in the {111} plane in $<110>$ direction. Its Burgers vector is $\frac{a}{2}[110]$\cite{_dislocation-theory}. \underline{ADD INFO	FOR 100!!!}

 Dense packed rows in fcc(111) are the following directions: $<\bar 1 01>$, $<01\bar 1>$, $<1\bar 1 0>$. The diagonals are found in the $<\bar 1 \bar 1 2>$ and $<1\bar 2 1>$ directions. \underline{ADD INFO	FOR 100!!!}
 
\subsection{Au(111)/Mica}
Au(111) substrates can be used either as single crystal or as thin evaporated layers on Mica. Here only the uppermost crystal layers ($\approx \SI{150}{\nano \meter}$) are gold that is supported by a mica substrate. Au(111) is known to reconstruct its surface to a herringbone shaped $22\times\sqrt{3}$\cite{Hanke_structure_2013}. This leads to the surface atoms being divided in regions with different registry (fcc/hcp) to the "bulk" atoms. When imaged in STM, these regions can be distinguished by their width, where fcc regions are wider than hcp regions which are separated by lines of higher lying, thus bright appearing, surface atoms. The two different surface terminations lead to chemical differing substrate regions, leading to a site-preferring adsorption of atoms and molecules \cite{pham_self-assembly_2014, pham_heat-induced_2015}.

\begin{figure}\centering
	\subfigure[Calculated structure of the reconstructed
	Au(111) surface, using the vdWDF/PBE density functional. (a) Site
	character c: distance from the ideal fcc and hcp site for each atom
	in the top layer, as discussed in the text. (b) Structure of the top
	layer. The color coding in the top half denotes the atom type, with
	yellow atoms being of fcc and green atoms being of hcp type. On
	the bottom half, the color indicates atomic height, with yellow atoms
	being the closest to the ideal fcc lattice continuation from below.
	(c) Calculated top-layer height with respect to the ideal fcc lattice
	and an experimental line scan.
	]{\includegraphics[width=0.45\textwidth]{./images/au-111-reconstruction}
%		\label{fig:au-111-1}
	} \quad
	\subfigure[Schematic representation of the
	Au(111)/$22\times\sqrt{3}$ reconstruction, showing how 23 surface atoms fit
	into 22 lattice sites by compressing the top layer of the surface with
	the additional atoms colored dark red. The positions corresponding
	to lined-up fcc and hcp sites are indicated by the vertical lines.
	STM image of the Au(111) herringbone reconstruction ($V_{tip} =
	0.9 eV, I_{set} = 1 nA$).
	]{\includegraphics[width=0.45\textwidth]{./images/au-111-reconstruction-2}
%		\label{fig:au-111-2}
	}
\label{fig:au11-herigbone}
\end{figure}
%     \subsection{Growing atomically thin ad layers with \textbf{C}hemical \textbf{V}apor \textbf{D}eposition}
%        	%  this is a description of a cvd process usually used in this work
There are many ways to grow ad layers with controlled stoichiometry in UHV. Here we will only focus on the most used method for sub mono layer and mono layer growth - the chemical vapor deposition (CVD). 
%#####################################################
\begin{figure}\centering
	\subfigure[]{
		\includegraphics[width=0.05\textwidth]{./images/precursor/diboran-125x125}
		\includegraphics[width=0.05\textwidth]{./images/precursor/ammonia-125x125}
		\label{fig:diboran-ammonia}
	} \qquad%
	\subfigure[]{
		\includegraphics[width=0.05\textwidth]{./images/precursor/ammonia-borane-125x125}
		\label{fig:ammonia-borane}
	} \qquad%
	\subfigure[]{
		\includegraphics[width=0.1\textwidth]{./images/precursor/borazine-250x250}
		\label{fig:borazine}
	} \qquad%
	\subfigure[]{
		\includegraphics[width=0.1\textwidth]{./images/precursor/B-Trichloroborazine-250x250}
		\label{fig:B-Trichloroborazine}
	} \qquad%
	\subfigure[]{
		\includegraphics[width=0.2\textwidth]{./images/precursor/decaborane-500x100}
		\includegraphics[width=0.05\textwidth]{./images/precursor/ammonia-125x125}
		\label{fig:decaborane-ammonia}
	} \qquad%
	\caption{Precursor molecules for \textit{h}-BN growth. \subref{fig:diboran-ammonia} A mixture of ammonia and diborane, \subref{fig:ammonia-borane} ammonia borane, \subref{fig:borazine} borazine and \subref{fig:B-Trichloroborazine} B-Trichloroborazine, \subref{fig:decaborane-ammonia} a mixture of decaborane and ammonia. Images reproduced from \cite{_pubchem}}
	\label{fig:h-BN-precursor}
\end{figure}
%#####################################################

When a gaseous precursor comes on contact with a hot transition metal surface, the precursor is split by pyrolysis into fragments. These then distribute across the surface and form new structures in a self-organized process. Choosing the right growth parameters like temperature, precursor partial pressure and time, high quality layers can be grown, whose stoichiometry is determined by the chemical structure of the precursor. Using carbon rich gases like ethylene (C2H4) \cite{ndiaye_structure_2008-1, coraux_growth_2009} and coronene (C24H12) \cite{coraux_growth_2009} as precursor and a transition metal as substrate results in a graphene layer to be formed. Using boron and nitrogen containing molecules (see \autoref{fig:h-BN-precursor})
\footnote{\textit{h}-BN CVD precursor:
	
	Borazine \cite{muller_epitaxial_2010, joshi_boron_2012, schwarz_corrugation_2017, li_grain_2015, preobrajenski_monolayer_2005, auwarter_xpd_1999, morscher_formation_2006, preobrajenski_monolayer_2007-1, corso_boron_2004, goriachko_self-assembly_2007, kidambi_situ_2014, kim_synthesis_2012}
	
	B-Trichloroborazine (${ClBNH}_3$) \cite{auwarter_synthesis_2004-1, muller_symmetry_2005}
	
	Ammonia borane (borazane) \cite{guo_controllable_2012-4, lee_large-scale_2012, kim_synthesis_2012-1} 
	
	Diborane and ammonia \cite{ismach_toward_2012}
	
	Reaction of ammonia with decaborane \cite{chatterjee_chemical_2011}
	
	Triethylborane and ammonia \cite{siegel_heterogeneous_2017}
} 
as precursor will result in \textit{h}-BN to be formed. Hereby either single crystalline \footnote{Single crystals used as growth substrate for \textit{h}-BN:
	
	(111):
	Ag \cite{muller_epitaxial_2010}, 
	Cu \cite{joshi_boron_2012, schwarz_corrugation_2017, li_grain_2015, preobrajenski_monolayer_2005,siegel_heterogeneous_2017}, 
	Ni \cite{preobrajenski_monolayer_2005, nagashima_electronic_1995, auwarter_synthesis_2004-1, auwarter_xpd_1999}, 
	Rh \cite{preobrajenski_monolayer_2007-1, corso_boron_2004},
	Pd \cite{nagashima_electronic_1995, morscher_formation_2006}, 
	Pt \cite{nagashima_electronic_1995, preobrajenski_monolayer_2007-1, muller_symmetry_2005}, 
	
	others:
	Cu(100) \cite{guo_controllable_2012-4}, 
	Ru(0001) \cite{goriachko_self-assembly_2007, preobrajenski_monolayer_2007-1}
}
or polycrystalline foils \footnote{Polycrystalline foils used as substrate:
	
	Cu: \cite{kidambi_situ_2014, lee_large-scale_2012, kim_synthesis_2012, kim_synthesis_2012-1, ismach_toward_2012, guo_controllable_2012-4, chatterjee_chemical_2011}
	
	Ni: \cite{ismach_toward_2012, chatterjee_chemical_2011}
}
are used as substrates. These play a key role in interaction strength with the ad layer. Their lattice constant determines the mismatch with the ad layer. For some of these systems, DFT calculation back up experimental results \cite{gomez_diaz_hexagonal_2013}.

\paragraph{self-limitation}
The precursors decomposes on contact with the hot substrate surface and its fragments form the ad layer. As time goes by, the ad layer grows in coverage and less free hot surface area is available for decomposing new ``building blocks'' for mono layer formation. If a mono layer is formed, no additional second layer is formed because of missing building blocks which only arise on contact with the uncovered substrate surface. Therefore this process is called self-limited. It is observed \cite{corso_h-bn_2005, cavar_single_2008, muller_epitaxial_2010} for various substrates in combination with a borazine precursor.

%While in CVD, the process of forming the ad layer (adsorption, decomposing, diffusion on surface upon coalescence with an already present nucleation seed, layer growth), \textbf{TPG} offers some other path of layer growth. 
%
%Due to the fact that the surface is covered from the beginning with the desired number of molecules, the density of present building blocks at a certain time (and same dosage as with CVD) is higher. Therefore this growth experiences other results as CVD. In direct comparison CVD has been a better way to grow full mono layers \cite{coraux_growth_2009}.

%	Hexagonal boron nitride grown on transition metals shows distinct differences between these two modes. CVD results in a more homogeneous mono layer coverage than TPG and is therefore preferred.
%

The growth by itself is well investigated on transition metal surfaces \cite{gomez_diaz_hexagonal_2013,morscher_formation_2006}, on the copper and nickel surfaces \cite{preobrajenski_monolayer_2005,joshi_boron_2012}. Even more complicated samples can be created with this technique \cite{roth_chemical_2013} and the following gives a short introduction in the occurring growth processes.

Some growth mechanics can be seen best in figure \ref{fig:borazine-TPG-on-Ir} that shows a XPS spectrum of borazine adsorbed on a Iridium surface held at \SI{170}{\kelvin} and after several annealing steps. 

At the graphics' bottom one can see the clean Ir surface with no borazine adsorbed (no B1s signal ). There are two contributions in the Ir-peak. While the low energy ($Ir_s$) peak stems from the surface atoms of the substrate, $Ir_b$ denotes the contribution from the atoms in the bulk. Upon borazine adsorption $(1)$ a broad $B1s$ emerges accompanied with a new contribution in the $Ir$-peak ($Ir_i$) which is a result of borazine-Ir interaction decreasing the area of $I_b$ and $I_s$ .
Upon annealing ($(2)$-$(6)$) $Ir_i$ looses in intensity while the $I_b$ and $I_s$ recover to their initial position. Interesting changes happen to the $B1s$ peak. While at lower temperatures, several peak contributions can be distinguished, denoted as $B_{mol}$ for entire molecules and $B_{ad}$ for molecular fragments. With increasing temperature, $B_{mol}$ decreases for a increase in the $B_0$ peaks. At lower temperature (1), $B_{mol}$ decreases and $B_{ad}$ slightly increases. 
\begin{wrapfigure}{l}{0.3\textwidth}\centering
	\includegraphics[width=0.3\textwidth]{./images/07571n_fig5.png}
	\caption{RT XPS spectra (B1s and Ir4f 7/2) of borazine adsorbed on a Iridium surface held at \SI{170}{\kelvin} and after stepwise annealing to \SI{1370}{\kelvin}. Adopted from \cite{orlando_epitaxial_2012}}
	\label{fig:borazine-TPG-on-Ir}
\end{wrapfigure}
When exceeding 620K ($\approx \SI{350}{\celsius}$, (3)) a new peak emerges and develops into $B_1$ when increasing temperatures. When temperature is high enough the only peaks left are $B_0$ and $B_1$ - the two contributions of boron atoms stem from boron atoms interact with the Iridium substrate with different strength due to different registry to the substrate.

While the growth temperature and partial pressures used to grow defect free \textit{h}-BN layers varies, the basic principle remains the same on all substrates.

The exact growth of mono- and multi layers \cite{ismach_toward_2012} is prone to discussion and may involve  

\clearpage             
     \subsection{\textit{h}-BN on Cu(111)}
		%In the last years, 2D materials with interesting properties were synthesized. One of it is hexagonal boron nitride. It is made up of the same number of nitrogen and boron atoms. They are arranged in an $\textnormal{sp}^2$-bonded honeycomb lattice so that each nitrogen is neighbored by boron atoms and vice versa. The ionic bond character between both is a direct result from the nitrogens larger electrochemical negativity, withdrawing electrons from the boron lattice site and making \textit{h}-BN an insulator with wide band 

In the last years, a new family of 2D materials arose. Its most known member, graphene, already attracted much attention in the research community. Graphene, a $\textnormal{sp}^2$-bonded carbon honeycomb lattice, has shown remarkable electronic and mechanic features determined by the atomic type and bond configuration. With careful preparation methods, large layers of defect free graphene are grown on a variety of substrates. Within this layer, a linear band dispersion is formed at the corners of the Brillouin zone (Dirac cones), that promote high charge carrier mobilities (cite).

Hexagonal boron nitride (\textit{h}-BN) is isostructural and -electronical to graphene, but every second lattice site is occupied by nitrogen, every other by boron. The nitrogens larger electrochemical negativity - withdrawing electrons from the boron side - results in an ionic bond character between B \& N. Compared to the covalent bond formation in graphene that facilitates its conductivity, the ionic bond character makes \textit{h}-BN a wide band gap ($\approx \SI{6}{\eV}$) insulator. \cite{watanabe_direct-bandgap_2004, cassabois_hexagonal_2016, blase_quasiparticle_1995} 

Free standing \textit{h}-BN is investigated with \textit{ab-initio} calculations \cite{han_effects_2014,mortazavi_investigation_2012,topsakal_first-principles_2009,peng_mechanical_2012}. Together with experiments \cite{paszkowicz_lattice_2002} a RT crystal lattice constant of $a_{\textit{h}-BN, RT}=\SI{2.504}{\angstrom}$ is derived.  It can be grown on a variety of metal surfaces.\cite{muller_epitaxial_2010,muller_one-dimensional_2008,guo_controllable_2012-4,siegel_heterogeneous_2017,schwarz_corrugation_2017,joshi_boron_2012,preobrajenski_monolayer_2005,vinogradov_one-dimensional_2012,farwick_zum_hagen_structure_2016,schulz_epitaxial_2014,Schulz_Templated_2013,gomez_diaz_hexagonal_2013,usachov_experimental_2012,orlando_epitaxial_2012,preobrajenski_monolayer_2007-1,preobrajenski_monolayer_2005,auwarter_synthesis_2004-1,auwarter_xpd_1999,nagashima_electronic_1995,corso_h-bn_2005,morscher_formation_2006,nagashima_electronic_1995,cavar_single_2008,muller_symmetry_2005,nagashima_electronic_1995,gomez_diaz_hexagonal_2013,dong_how_2010,brugger_reversible_2010,preobrajenski_monolayer_2007-1,berner_boron_2007,corso_boron_2004,brugger_comparison_2009,goriachko_self-assembly_2007}


 Depending on the substrates used, different lattice mismatches can be achieved. Many geometric corrugations can be achieved with increasing lattice mismatch and substrate-\textit{h}-BN interaction. Since this interaction is believed to have its origin in the partially filled d-states of the substrate,\textcolor{red}{\textbf{citation}} transition metal substrates are widely used. While substrates exist where the lattice constant are virtually identical (Ni: $\Delta \leq \SI{0.5}{\percent}$), other substrates show large mismatches (Ag(111): $\Delta \approx \SI{14}{\percent}$).

The growth of \textit{h}-BN on nearly lattice matched Ni(111) resulted in uniform commensurate layers. With increasing lattice mismatch, moir\'e patterns are formed on Pd and Pt. The stronger interaction of \textit{h}-BN and Rh(111) results in a corrugated nanomesh to be formed on the substrate.\textcolor{red}{\textbf{citation}} Even 1D structures are reported on Fe(110) \cite{vinogradov_one-dimensional_2012} and Cr(110) \cite{muller_one-dimensional_2008}. 

Here we consider \textit{h}-BN on Cu(111) as example system of self-limited growth and highlight most relevant insights reported in literature.\cite{joshi_boron_2012, schwarz_corrugation_2017, auwarter_hexagonal_2018}

\subsubsection{on Cu(111)}
%\begin{table}\centering
%	\caption{Lattice mismatches between \textit{h}-BN and several transition metal surfaces. The mismatch is given to describe the relative size of the \textit{h}-BN layer compared to the substrate, e.g. negative values indicate a larger lattice constant in the substrate bulk. Information adopted from \cite{_ptable}}
%	
%	\begin{tabular}{cccl}
%		Substrate 	& Mismatch [\%] 		& Electronic configuration \\ \hline
%		Ni(111)		& \SI{+0.4}{\percent} 	& [Ar] 3d8 4s2	\\
%		Cu(\left( 111)		& \SI{-1.9}{\percent} 	& [Ar] 3d10 4s1	\\	
%		 \\
%	\end{tabular}
%	\label{tab:h-BN-mismatch}
%\end{table}

	\paragraph{Stoichiometry}
	XPS measurements (see \autoref{fig:XPS-hbn-Cu111-martin}) show B\textit{1s} and N\textit{1s} components in a 1:1 ratio, indicating that the layer retains the precursor stoichiometry during growth, but all hydrogens are cleaved from the precursor prior to layer formation and desorp from the sample.\cite{Zhang_Two-dimensional_2017}
	
\begin{figure} \centering
	\includegraphics[width=0.7\textwidth]{./images/XPS-hbn-Cu111-martin}%
	\caption{XPS of a full ML \textit{h}-BN on Cu(111) grown with CVD of borazine. Fit components $B_0$ ($E_b=\SI{190.4}{\eV}$), $N_0$ ($E_b=\SI{398.0}{\eV}$) and $B_{def}$ ($E_b=\SI{191.0}{\eV}$) and $N_{def}$ ($E_b=\SI{398.5}{\eV}$) are assigned to pristine \textit{h}-BN layer and defective components respectively. Adopted from \cite{schwarz_assembly_2018}}
	\label{fig:XPS-hbn-Cu111-martin}
\end{figure}



	\paragraph{Moir\'e geometry}
	The properties of various moir\'e superstructures are well described in literature and Hermann gives a comprehensive overview in his paper.\cite{hermann_periodic_2012}\label{section:moire}
	
%	If lattice constants are equal like in the case of a graphene bilayer, the needed lattice mismatch occurs due to a rotation of the two layers. 
	A moir\'e is always present if an over layer shows a lattice mismatch with respect to the substrate. 
	
	For \textbf{isotropically scaled over layers} one can calculate the scaling factor $$p=\frac{R^{'}_{O1}}{R_{O1}}$$ which gives the size of the over layer lattice $R^{'}_{O1}$ in units of the substrate lattice $R_{O1}$. The moir\'e pattern shows the same bravais lattice type than the substrate\cite[10]{hermann_periodic_2012}. If moir\'e and ad layer lattice are aligned ($\alpha=0$\textdegree) the direction of moir\'e and substrate is aligned. If the over layer is isotropically scaled and not rotated, the period of the moir\'e calculates to $$a_{moir\'e}=\underbrace{\frac{p}{|p-1|}}_{\kappa}a_{substrate}$$
	With $a_{moir\'e}$ and $a_{substrate}$ are experimentally available, the ad layer lattice can be calculated with high precision (usually one order of magnitude more accurate than direct measurement of its period).\cite{farwick_zum_hagen_structure_2016}

Depending on the relative orientation of \textit{h}-BN and substrate the moir\'e period changes. Although large domains with uniform orientation can be grown on single crystal substrates (\autoref{fig:moire-STM-model}(a,c)), rotational domains exist(\autoref{fig:moire-STM-model}(b,d,e)).
The model representation nicely shows the change in moir\'e period when
\autoref{fig:moire-STM-model}(c) shows a case where the \textit{h}-BN ad layer has the same unit cell orientation than the copper. For a \textbf{scaled and rotated over layer} the angle between substrate and moir\'e ($\gamma$[rad]) scales with the angle between over layer and substrate ($\alpha$[rad]) as $\alpha=(1-p)\gamma$.
For rotated and isotropically scaled over layers, one can determine the $\alpha$ and $p$ from experimental observables $\gamma$(moir\'e angle to substrate) and $\kappa$(scaling factor) through relations $ \alpha=\arctan \left ( \frac{sin(\gamma)}{cos(\gamma)+\kappa} \right )\qquad p=\frac{\kappa}{\sqrt{1+\kappa^2+2\kappa cos(\gamma)}}$

\begin{figure} \centering
	\includegraphics[width=0.5\textwidth]{./images/h-BN-cvd-cu111.png}%
\caption{(a) STM topography of \textit{h}-BN on Cu(111). Large domains with uniform orientation and moir\'e period are formed after CVD growth. (b) Different rotational domains are observed that show different moir\'e periods, reproduced by three models (c-e) where different \textit{h}-BN rotations are shown together with the resulting moir\'e periods a. Adopted from \cite{joshi_boron_2012}}
\label{fig:moire-STM-model}
\end{figure}

As mentioned above the orientation of the moir\'e superstructure is determined by the relative ad layer rotation alone, while its period is determined by lattice mismatch, too. This results in a variety of moir\'e superstructure orientations and periods, strongly related to the used substrate.
	
\paragraph{Periodic change in work function}
A direct result of the lattice mismatch between \textit{h}-BN and Cu(111) is the changing registry of ad layer atoms and substrate. The periodic modulation of B/N registry to the substrate atoms results in regions of stronger and weaker interaction between \textit{h}-BN and substrate and is the reason for the nano templating effect of \textit{h}-BN on many substrates.\textcolor{red}{\textbf{citation}} In the following some effects are discussed that lay the foundation for a nano patterning effect of \textit{h}-BN and its influence on the electronic structure of adsorbates.

While a first report in 2004 \cite{corso_boron_2004}, pointed to the formation of a complicated two layer structure, later experiments \cite{roth_chemical_2013, li_grain_2015} including ours \cite{joshi_boron_2012, schwarz_corrugation_2017} and others \textit{h}-BN/Cu(111) proofed a single layer of B \& N atoms in a regular hexagonal lattice. It evolved as well investigated system to perform experiments on. It could be shown that after CVD growth it adsorbs on Cu(111) as a flat layer. Due to its  lattice mismatch, "hill" regions  (corresponding to a $N_{top}B_{fcc}$ registry) and "valleys" (corresponding to a $N_{fcc}B_{hcp}$ registry) are formed. In these regions the work function is altered in opposite directions. While larger at the hill/pore regions, the work function reduces continuously to its lowest value in the valley/wire regions.\footnote{Please note that the notation is not uniform throughout the literature. Sometimes hills are referred to as pores and valley regions are denoted as wire regions.} 

Everywhere two regions with different work functions meet, local electrostatic fields arise to compensate for the vacuum level misalignment.

After growth of \textit{h}-BN the substrates over all work function is reduced [e.g. Rh: \SIrange{5.01}{3.07}{\eV} \cite{gomez_diaz_hexagonal_2013}. Therefor a dipole moment $\mu$ pointing from the the bulk to the surface is necessary, rather likely created by a negative charge transfer from the bulk into the ad layer.\cite{roman_periodic_2013}


\begin{figure} \centering
	\subfigure[Work function variation along \textit{h}-BN/Cu(111) moir\'e. (a) STM image showing the \textit{h}-BN moir\'e with a periodicity of \SI{8.4}{\nano \meter}. Scan parameter: $U_b = \SI{4.0}{\volt}, I_t = \SI{40}{\pico \ampere}$. (b)	Field emission resonances acquired along the black dotted line in a) revealing a variation of the peak positions. (c) Work  function  differences  between bright  (“hill”/pore)  and  dark  (“valley”/wire)  regions obtained  from  the  dI/dV curves  of  the  field emission  resonances  displayed  in  b). Adopted from \cite{schwarz_corrugation_2017}]{
		\includegraphics[width=5cm]{./images/h-BN-Cu(111)-wf-change}
	\label{fig:h-BN-Cu(111)-wf-change-I}
		} \quad
	\subfigure[Position dependent energy level alignment of $NC-Ph_4-CN$ on \textit{h}-BN/Cu(111). Three spectra are compared, recorded on molecules at valley (blue) and hill positions (green). A spectrum of the bare Ag(111) is shown as reference. Taken from \cite{diss-joshi}]{
	\includegraphics[width=6cm]{./images/h-BN-Cu(111)-wf-change-II}
	\label{fig:h-BN-Cu(111)-wf-change-II}
	}
	\caption{Workfunction change for pristine \textit{h}-BN/Cu(111) (left) and for 2H-Porphines on \textit{h}-BN/Cu(111) (right). For pristine \textit{h}-BN/Cu(111) the position dependend change in workfunction is shown by a variation in the field emission resonances. After adsorption of 2H-Porphine a shift in molecular orbital energy is shown in single point spectra taken at different positions within the moir\'e unit cell.}
	\label{fig:h-BN-Cu(111)-wf-change}
\end{figure}

\subsection{Molecular adsorption and assembly on \textit{h}-BN}
\label{section:Mol-on-h-BN}
With changing work function, a lateral electric field emerges. For gr/Ru(0001) lateral dipole pointing from valley to pore sites arise.\cite{zhang_assembly_2011} It can be used to trap adsorbates with dipole moment along the field lines. This was shown for FePc and pentacene molecules on a graphene/Ru(0001) substrate. Here FePc molecules adsorp first on regions with high lateral dipole along top-fcc direction (valley), followed by regions with lower lateral dipole (on the hill). Pentacene molecules are trapped along the top-fcc direction, too.\cite{zhang_assembly_2011}  This general adsorption mechanism is applicable for other systems with periodic modulation of the work function.

\autoref{fig:h-BN-Cu(111)-wf-change} depicts the work function change measured with STS (Field emission resonances) indicating a similar modulation of the work function. In this theses TBP molecules (\autoref{section:TBP}) and helicene molecules (\autoref{section:helicene}) are used as sample molecules for specific adsorption site or orientation alignment.

It was shown that this moir\'e superstructure influences molecular assembly. 
\textcolor{red}{\textbf{Explain the Porphine adsorption on metal and on \textit{h}-BN}}

\begin{figure} \centering
	\includegraphics[width=0.7\textwidth]{./images/2H-P-hBN-Cu111-joshi}%
	\caption{STM topography of 2H-P adsorped on \textit{h}-BN/Cu(111). (a) A large moir\'e domain guides the formation of small 2H-P islands with off center vacancies. (b) High coverage overcomes the template effect of the \textit{h}-BN support. Adopted from \cite{diss-joshi}}
	\label{fig:2H-P-hBN-Cu111-joshi}
\end{figure}

\textbf{Molecules are electronically decoupled} when adsorped on a \textit{h}-BN spacer layer on top of a metal. The insulating \textit{h}-BN hinders the metal to influence molecular properties by charge transfer, image potentials etc.\textcolor{red}{\textbf{citation}}. As a result, molecular orbitals are unperturbed and can be imaged in STM/STS.
  \section{Used molecules}
    \label{chapter:used-molecules}
The following molecules have been used to conduct experiments. Images show the molecules in a \SI{4x4}{\nano \meter} area in an orthographic projection. Since all of them feature distinct properties, all of them are introduced in the following subsections. We will utilize porphyrine derivatives (\autoref{sec:TBP}), functionalized pyrene molecules (\autoref{sec:pyrene}), helicene species (\autoref{sec:helicene}) and coronene (w\/o central borazine functionalization, see \autoref{sec:hbc}).

All the depicted molecules are modeled in Hyperchem\cite{_hyperchemtm_1111} and calculated for optimized geometry with the AM1+ method. Afterwards their positions are exported and remodeled in blender. Note that this does not change their geometry. It is only for better control of the output (faster and more accurate model building especially in 3D) and for some aesthetic reasons.

%%%%%%%%%%%%%%%%%%%%%%%%%%%%%%%%%%%%%%%%%%%%%%%%%%%%%%%%%%%%%%%%%%%%%%%%%%%%%%%%%%%%%%%%%%%
%%%%%%%%%%%%%%%%%%%%%%%%%%%%%%%%%%%nitro-prophines%%%%%%%%%%%%%%%%%%%%%%%%%%%%%%%%%%%%%%%%%
\subsection{Porphine: [di-[tert-butyl]-phenyl)]-porphyrin derivatives}
\label{sec:TBP}\index{molecules!TBP}
Tetrapyrroles like porphyrins and phthalocyanines play important roles in biological systems \cite{battersby_tetrapyrroles_2000}. Both species are able to incorporate metal atoms that control the function. Not only are they interesting model systems to study interaction towards a (metallic) substrate\cite{auwarter_porphyrins_2015, auwarter_controlled_2007, diller_vacuo_2016}. Their use in metal-organic frameworks highlights the use of scientific knowledge to design "real world" sensor applications\cite{Lustig_Metal-organic_2017}. 


Tert-butyl functionals have been used in a variety of molecules \cite{moresco_conformational_2001}. Due to their bulky nature, they electronically decouple the porphyrin’s de-localized p-orbital system from the metallic surface just by lifting the molecule. They may undergo heavy conformational deformation when outer influences (like metalization of the central porphine core) act on the molecule \cite{stark_massive_2014}. Switching capabilities are well investigated \cite{loppacher_direct_2003} and it is possible to switch them with the STM tip \cite{ditze_energetics_2014}. Experiments with similar molecules investigate the heat-induced formation of 1D and 2D conglomerates on a Au(111) surface.\cite{pham_heat-induced_2015}

\begin{itemize}
	\item Free base nitrophenyl - 5,10,15 Tri [di-[tert-butyl]-phenyl)]-porphyrin \index{nitro porphin} has 3(2) di-tert-butyl-phenyl groups attached to the porphine macro cycle at the meso-positions of the molecule. The free meso-positions are occupied with nitrophenyl groups as shown in \autoref{fig:TBP-single} If more than one functional group is present, one can distinguish between trans (\autoref{fig:TBP-trans}) and cis configuration (\autoref{fig:TBP-cis}), whether the two functional groups are opposite or neighboring.
	\item The appearance of STM data is correlated to the molecular configuration according to \cite{mishra_current-driven_2015} meaning that the lobes consisting of (3,5-di-tert-butylphenyl) are imaged as bright protrusions, while the functional nitro group is imaged fainter. This holds true for cis- and trans-substituted molecules\cite{yokoyama_selective_2001}.
	\item Tert-butyl groups can rotate and form flexible legs. Interaction with the substrate results in adsorption-induced conformational changes.\cite{ecija_dynamics_2016}
\end{itemize}
Drawings for various functional groups and molecules can be found in \cite{jorgensen_salem_1973}

\begin{itemize}
	\item[one-leg:]  	 5,10,15-Tri(3,5-di-tert-butylphenyl)-   20-(Nitrophenyl)porphyrine
	\item[two-leg cis:] 	5,10-Bis(3,5-di-tert-butylphenyl)-15,20-Bis(Nitrophenyl)porphyrine
	\item[two-leg trans:] 	5,15-Bis(3,5-di-tert-butylphenyl)-10,20-Bis(Nitrophenyl)porphyrine
\end{itemize}

\begin{figure}[]\centering
	\subfigure[Single functional group]{
		\includegraphics[angle=90, width=0.3\textwidth]{./images/molecules/TBP-single}
		\label{fig:TBP-single}
	} %
	\subfigure[Trans-configuration]{
		\includegraphics[angle=0, width=0.3\textwidth]{./images/molecules/TBP-trans}
		\label{fig:TBP-trans}
	} %
	\subfigure[Cis-configuration]{
		\includegraphics[angle=0, width=0.3\textwidth]{./images/molecules/TBP-cis}
		\label{fig:TBP-cis}
	} %
	\caption{Functionalized tert-butyl-phenyl-porphines. \subref{fig:TBP-single} shows a single functionalized porphine molecules. An additional function may be added in \subref{fig:TBP-cis} cis-  and \subref{fig:TBP-trans} position.}
	\label{fig:TBP}
\end{figure}
%%%%%%%%%%%%%%%%%%%%%%%%%%%%%%%%%%%%%%%%%%%%%%%%%%%%%%%%%%%%%%%%%%%%%%%%%%%%%%%%%%%%%%%%%%%
%%%%%%%%%%%%%%%%%%%%%%%%%%%%%%%%%%%	pyrenes   %%%%%%%%%%%%%%%%%%%%%%%%%%%%%%%%%%%%%%%%%
\subsection{Pyrene: Pyridilethynyl functionalized pyrenes}
\label{sec:pyrene}\index{molecules!Pyrene}
\begin{itemize}
	\item[tetra-pyrene:] 1,3,6,8-Tetra(4-Pyridylethynyl)pyrene
	\item[cis-pyrene:] 1,8-Bis(4-Pyridylethynyl)pyrene
	\item[trans-pyrene:] 1,6-Bis(4-Pyridylethynyl)pyrene
\end{itemize}

\begin{figure}[]
	\begin{center}
		\subfigure[Tetra-configuration]{
			\includegraphics[width=0.3\textwidth]{./images/molecules/pyrene-tetra}
			\label{fig:pyrene-tetra}
		} 
		\subfigure[Trans-configuration]{
			\includegraphics[width=0.3\textwidth]{./images/molecules/pyrene-trans}
			\label{fig:pyrene-trans}
		} 
		\subfigure[Cis-configuration]{
			\includegraphics[width=0.3\textwidth]{./images/molecules/pyrene-cis}
			\label{fig:pyrene-cis}
		}
	\end{center}
	\caption{Pyridyl-Pyrene molecules in trans- \subref{fig:pyrene-trans} and cis- \subref{fig:pyrene-cis} and tetra- \subref{fig:pyrene-tetra} configuration}
	\label{fig:pyrene}
\end{figure}

Pyrene molecules, first investigated in 1973 \cite{khan_electronic_1973}, are 4 ortho-fused carbon rings to result in a rhombic structure. As many other $\pi$ conjugated systems they show interesting optoelectronic properties like \cite{crawford_experimental_2011, lee_enhanced_2012, feng_functionalization_2016, kurata_diarylamino-_2017, maeda_alkynylpyrenes_2006, kurata_diarylamino-_2017} and assembly was investigated \cite{pham_self-assembly_2014, matena_aggregation_2010, della_pia_anomalous_2014, pham_comparing_2016}. Here they are used to investigate the influence of the number and position of functional groups on these properties. The very same species have been investigated on Cu(111) albeit data adsorbed on \textit{h}-BN/Cu(111) was lacking up to this point. The nano-pattering effect of the \textit{h}-BN substrate is used here to modulate the wide band gap of the species and therefor their optical properties.
%%%%%%%%%%%%%%%%%%%%%%%%%%%%%%%%%%%%%%%%%%%%%%%%%%%%%%%%%%%%%%%%%%%%%%%%%%%%%%%%%%%%%%%%%%%
%%%%%%%%%%%%%%%%%%%%%%%%%%%%%%%%%%%	TPCN      %%%%%%%%%%%%%%%%%%%%%%%%%%%%%%%%%%%%%%%%%
%\subsection{TPCN}
%TPCN can be evaporated with an OMBE. Temperatures used are typically \SI{490}{\celsius}, evaporation time depends on the intended coverage. 
%\begin{itemize}
%	\item [TPCN:] Tetra[(4-cyanophenyl)-phen-4-yl] porphyrin has four arms attached to the meso-positions of the macrocycle. Each is build up from two chained phenyl rings with one end attached to the macrocycle and the other one attached to a C-N end group. Due to their flexibility, they are versatile connection segments \cite{fendt_modification_2009}.
%\end{itemize}
%
%\begin{figure}[]
%	\centering
%	\includegraphics[width=0.3\textwidth]{./images/molecules/TPCN}
%	\caption{TPCN molecule}
%	\label{fig:TPCN}
%\end{figure}

%%%%%%%%%%%%%%%%%%%%%%%%%%%%%%%%%%%%%%%%%%%%%%%%%%%%%%%%%%%%%%%%%%%%%%%%%%%%%%%%%%%%%%%%%%%
\begin{table}\centering
	\caption{Evaporation and degas temperatures used for different molecules.}
	\begin{tabular}{ccrc}
		Name			& Configuration & Degas [\SI{}{\degreeCelsius}]	& Evaporate [\SI{}{\degreeCelsius}]	\\ \hline \hline 
		TPCN			& ---		& ---		& 490		\\ \hline 
		\multirow{3}{*}{TBP}	&single		& \SI{4}{\hour} @ \SI{200}{\degreeCelsius}& 390	\\
		&cis		& ---		& \SI{400}{\degreeCelsius}\\
		&trans		& \SI{4}{\hour} @ \SI{200}{\degreeCelsius} + \SI{1}{\hour} @ \SI{270}{\degreeCelsius}&\SI{370}{\degreeCelsius}\\ \hline 
		\multirow{4}{*}{pyrene} & \multirow{3}{*}{cis}		& \SI{2}{\hour} @ \SI{180}{\degreeCelsius}&	\multirow{3}{*}{250}	\\
		&&+ \SI{1}{\hour} @ \SI{200}{\degreeCelsius} + \SI{10}{\minute} @ \SI{235}{\degreeCelsius} 	&\\
		&&+ \SI{1}{\hour} @ \SI{220}{\degreeCelsius}&\\ 
		&trans		& \SI{1}{\hour} @ \SI{230}{\degreeCelsius}		&\SI{265}{\degreeCelsius}		\\ \hline
		\multirow{3}{*}{DCDB} & \multirow{3}{*}{---} & 1h @ \SIrange{100}{150}{\degreeCelsius}& \multirow{3}{*}{\SIrange{220}{240}{\degreeCelsius}}\\
		&&+\SI{10}{\minute} @ \SI{170}{\degreeCelsius} + \SI{25}{\minute} @ \SI{200}{\degreeCelsius} & \\
		&&+ \SI{40}{\minute} @ \SI{220}{\degreeCelsius}&\\
	\end{tabular}
	\label{tab:molecule-temperatures}
\end{table}


%%%%%%%%%%%%%%%%%%%%%%%%%%%%%%%%%%%%%%%%%%%%%%%%%%%%%%%%%%%%%%%%%%%%%%%%%%%%%%%%%%%%%%%%%%%
%%%%%%%%%%%%%%%%%%%%%%%%%%%%%%%%%%%	helicenes %%%%%%%%%%%%%%%%%%%%%%%%%%%%%%%%%%%%%%%%%
\subsection{Helicene: Cyano functionalization of helicenes}
\label{sec:helicene}\index{molecules!Helicene}
\begin{itemize}
	\item[Dicyano-dibenzo-[5]helicene]: 7,8-Bis(cyano)-Dibenzo-helicene
\end{itemize}

\begin{figure}[]
	\centering
	\subfigure[Top view]{
		\includegraphics[width=0.3\textwidth]{./images/molecules/helicene}
		\label{fig:helicene-top}
	} \quad
	\subfigure[Sideview]{
		\includegraphics[width=0.3\textwidth]{./images/molecules/helicene-side}
		\label{fig:helicene-side}
	} \quad
	\caption{DCDB on copper surface. \subref{fig:helicene-top} Top view, \subref{fig:helicene-side} side view}
	\label{fig:helicene}
\end{figure}

Helicenes were first synthezised 1950's \cite{newman_synthesis_1956}. They consist of ortho condensed carbon rings that form a spiral due to overcrowding in their center. While they first drew attention due to their fluorescence properties \cite{vander_donckt_fluorescence_1968}, helicenes are interesting molecules because of their chiral feature. Two different turn directions exist, left and right. The molecules investigated in this work are fcuntionalized with two benzene rings at positions \underline{\qquad} and two cyano groups at positions seven and eight. For more information, please refer to \autoref{section:helicene}.

%%%%%%%%%%%%%%%%%%%%%%%%%%%%%%%%%%%%%%%%%%%%%%%%%%%%%%%%%%%%%%%%%%%%%%%%%%%%%%%%%%%%%%%%%%%
%%%%%%%%%%%%%%%%%%%%%%%%%%%%%%%%%%%	HBBNC + HBC   %%%%%%%%%%%%%%%%%%%%%%%%%%%%%%%%%%%%%%%%%
\subsection{Coronene: HBBNC and HBC}
\label{sec:hbc}\index{molecules!Coronene}
\begin{itemize}
	\item[HBBNC:] 2-8-14-trixylyl-hexaphenyl borazinocoronene
	\item[HBC:] 2,8,14-trixylyl-hexabenzocoronene	
\end{itemize}

\begin{figure}[]\centering
	\subfigure[HBBNC]{
		\includegraphics[width=0.3\textwidth]{./images/molecules/HBBNC}
		\label{fig:HBBNC}
	} \quad
	\subfigure[HBC]{
		\includegraphics[width=0.3\textwidth]{./images/molecules/HBC}
		\label{fig:HBC}
	} \quad
	\caption{\subref{fig:HBBNC} HBBNC and \subref{fig:HBC} HBC}
	\label{fig:HBBNC+HBC}
\end{figure}

While in 2015\cite{Krieg_construction_2015} and 2016 \cite{Ciccullo_Quasi-Free-Standing_2016} hexy-peri-Hexabenzoborazino coronene (HBBNC) was synthesized, its bad solubility prohibited experiments. In 2017 the synthesis \cite{dosso_synthesis_2017} of a soluble, BN-doped coronene derivative by substitution of the central carbon ring was successful. By using HBBNC the HOMO-LUMO band gap could be widened and shows blue-shifted emission properties\cite{dosso_synthesis_2017} compared to its all-carbon counterpart. Investigations with STM are performed on Au(111) \cite{Krieg_construction_2015}.
Here the central $(BN)_3$ core is oriented to point all nitrogen atoms towards the leg functionalization.
Due to the different electro negativity of the atomic species adsorption of gases in the central part can be interesting effects to look out for. In this thesis we focused on the geometric properties of this molecule first. Please refer to \autoref{section:HBBNC} for detailed information.
%\printbibliography
  \section{Sample preparation}
    \subsection{Etching copper foils}
Growing high quality \textit{h}-BN ad layers on polycrystalline copper foils requires a smooth surface, but as ordered Cu foils exhibit a root-mean-square (RMS) surface roughness $S_q$ of up to \SI{218}{\nm}\cite{bin_zhang_low-temperature_2012}. Steep, linear depressions, called striations are fabrication remnants due to the cold rolled foils and are observed on the surface\cite{kim_synthesis_2012-1}. Also some manufacturers apply a thin layer of chromium oxide for corrosion protection\cite{bin_zhang_low-temperature_2012}, that has to be removed prior \textit{h}-BN growth. A common procedure to reduce the roughness of a material is to mechanically polish the surface. When an even smaller RMS of height irregularities  
$$RMS\ \hat{=}\ S_q = \frac{1}{N}\sum_{n=1}^N\left(z_n-\bar{z}\right) \qquad z: \textnormal{heigth at point n}, \quad \bar{z} = \textnormal{mean heigth}$$ 
is needed, electrochemical polishing is an alternative.

\textcolor{red}{What are typical values for single crystal and foils? Check \cite{jinshan_electrochemical_2004} and experimental images in STM of Cu(111).}

The following gives a short introduction in chemical polishing as used for preparation of thin copper foils.\cite{antoine_polishing_1999, lilje_improved_2004, schulz_engeneering_2018}


\label{sec:etching}
\paragraph{Electrochemical cell}
The electrochemical cell, used to etch copper foils, is sketched in \autoref{fig:etching-setup}. A beaker is filled with etching solution and two electrodes are immersed. One electrode is the material to be polished (copper foil, working electrode), the other one is the counter electrode (copper in this case, counter electrode). Both are placed with \SI{3}{\centi \meter} apart and are oriented parallel. In this setup the working electrode has a surface of about \SI{1}{\square \centi \meter}, the counter electrode of about \SI{16}{\square \centi \meter}. The beaker is filled with etching solution until both electrodes are immersed.

An electric connection is made between working electrode (\SI{0}{\volt} < U < \SI{2}{\volt}) and the power supply. The counter electrode acts as ground. Both electrodes are fixed with alligator clamps to the wires. The current through the etchant depends on the applied potential and the etchant composition. The larger the current carried by the ionized copper atoms, the more material is transported from the working to the counter electrode.

\begin{figure}\centering
	\subfigure[Sketch of a typical setup used for electro polishing. A beaker is used to hold the aqueous cell medium. The copper foil and a counter electrode are immersed and connected a the + and - connections of a DC power supply. Image reproduced from \cite{stables_report_2008}]{
		\includegraphics[width=0.45\textwidth]{./images/cm1028854-fig2-d}
		\label{fig:etching-setup}
	} \quad%
	\subfigure[Current-voltage characteristic indicating different phases in the ething and polishing process. While at low voltage etching is the dominant process, a polishing plateau is formed at intermediate voltages. Exceeding a threshold (cusp point) leads to increased formation of excess oxygen in the oxygen pitting regime. \cite{luo_effect_2011}]{
		\includegraphics[width=0.45\textwidth]{./images/oxygen-pitting}
		\label{fig:oxygen-pitting}
	}
	\caption{Experimental setup and voltage characteristic used for electrochemical polishing of copper foils. \subref{fig:etching-setup} In the process the foil is connected as working electrode (+) and opposed by a counter electrode (-). Material is then transported from the working to the counter electrode resulting in a polished foil surface. \subref{fig:oxygen-pitting} Choosing proper voltage and current values within the polishing plateau is important for good results.}
	\label{fig:setup-and-characteristic}
\end{figure}

\paragraph{Aqueous etching solution}\index{electrochemical polishing}
Many different etching solutions are presented in literature, the most widely ones are summarized by Jinshan et al. in \cite{jinshan_electrochemical_2004}. Since the main goal is to achieve a flat surface, the resulting roughness of the surface is the most important parameter. 

Here two simple but efficient etching solutions that both result in a smooth surface and little etch pits are compared in \autoref{table:used-etching-solutions}. First, pure aqueous ortho-phospheric acid (\SI{85}{\percent}) was investigated as etchant \cite{jinshan_electrochemical_2004} with a anode-cathode potential of \SI{1.2}{\volt}. The limiting current is \SI{12}{\milli \ampere}. If now Ethylene-glycol (EG) (\SI{5}{\percent} of solutions volume) and deionized water (\SI{25}{\percent}) are added, the potential range to etch at remains the same, but the critical current is increased. Although the etching rate increases by a factor of 4, the resulting roughness remains the same \SI{5}{\nano \meter}, so foils etched with one of both recipes are of comparable quality. In contrast to etching recipes without EG where oxygen pitting is an issue (see \autoref{oxygen-pitting}) and more complex etching recipes (with reduced etching time but larger RMS) the chosen etchant composition (see \autoref{tab:composition-etching-solution-I}) is reported to give the best results.

\begin{table}\centering
	\caption{Used etching solutions (compare \cite[130]{jinshan_electrochemical_2004}). Note the change in the removal rate due to higher limiting currents in the solution after adding ethylene glycol to the solution.}
	\begin{tabular}{lcc}
		& I & II \\ \hline \hline
		$\SI{85}{\percent} H_3PO_4$ & 70 & 100 \\
		Ethylene-gylcol & 5 & 0 \\
		Deionized water & 25 & 0 \\ \hline
		Potential [\SI{}{\V}] & \multicolumn{2}{c}{\SI{1.2}{}} \\
		Current [\SI{}{\mA}] & 46 & 12\\
		Roughness [\SI{}{\nm}] & \multicolumn{2}{c}{\SI{5}{}} \\
		Removal rate [\SI{}{\micro\meter\per\minute}] & \SI{1,0}{} & \SI{0,26}{}\\
	\end{tabular}
	\label{table:used-etching-solutions}
\end{table}

\begin{table}
	\centering
	\caption{Volume and mass fractions for copper foil etching solution.}
	\begin{tabular}{lcccc}
		&unit	&$H_3PO_4$ (85\%)&	EG	&	$H_2O$	\\
		Dichte $\rho$   &[$g/cm^3$]	&	1.87	&	1.11	&	1.00	\\
%		$1/rho$		&[$cm^3/g$]	&	0.54	&	0.90	&	1.00	\\
		Anteil 		& \%		&	70	&	5	&	25	\\ \hline
		Menge gesamt    &[$cm^3$]	&		\multicolumn{3}{c}{150} 	\\
		Menge anteilig  &[$cm^3$]	&	105.00	&	7.50	&	37.50	\\
		Gewicht         &[g]		&	196.35	&	8.33	&	37.50	\\
	\end{tabular}
	\label{tab:composition-etching-solution-I}
\end{table}

\paragraph{Redox reaction}\index{electrochemical polishing!chemical reaction}

Electrons and atoms at the solid surface have higher energy states. Thus some of the atoms on the metal surface may lose electrons to form ions. These ions may also recombine with electrons and become atoms at another moment. Depending on the electronic structures, some metals (such as sodium) are easier than others (such as platinum) to ionize. Copper is relatively stable. Still, some of the surface atoms may be expected to ionize at a moment. The ionization process may be promoted when the metal is in touch with an aqueous solution because: 
\begin{itemize}
	\item Metal ions can not move in the metal electrode but can move through the solution, generating an electric current in solution when an potential is applied.
	\item Electrons can move freely in metal solid (electric current in a metal) but can not survive in solution because they will quickly recombine with positive ions
	\item Water dipoles and negative ions in solution may form a shell around the metal atom and drag the surface metal ions into the solution.
\end{itemize}

The electrode connected to the positive pole of the power supply is called anode, the one connected to the negative pole of the power supply is called cathode. When the applied voltage is high enough, electrons in the anode may be pumped out and the metal atoms on the anode surface will be oxidized (e.g., $Cu - 2e = Cu^{2+}$) and dissolved into the electrolyte solution. Under electrical field, the positive ions (cations) move to the cathode and negative ions (anions) move towards the anode. The cations may fetch electrons here and are reduced to neutral atoms (e.g., $Cu^{2-} + 2e = Cu$) at the cathode surface. Therefore, charge transfer between the two electrodes is carried out via the ion drift in the electrolyte and electron conduction in metal wire. When the coppper foil to be polished is connected to the anode, dissolution is processed at certain potential. Likewise, when the foil is connected to the cathode, it will result in deposition. For electro polishing of copper, the copper part to be polished is set to be anode while the cathode can be any conductive material (e.g. copper).

The critical potential at which the oxidation / reduction starts to occur is related to the standard redox potential for a specific anode material. The redox potential $E_O$ is a measure (in volts) of the electron affinity of a substance - its electro negativity - compared with hydrogen (which is set at \SI{0}{\volt}). Substances more strongly electronegative (i.e., capable of oxidizing or accepting electrons) than hydrogen have positive redox potentials (e.g., $Cu/Cu^{2+}$: $E_O = \SI{0.34}{\volt}$). Substances less electronegative (i.e., capable of reducing or giving up electrons) than hydrogen have negative redox potentials (e.g., $Cr^{3+}/Cr^{2+}$: $E_O = \SI{-1.07}{\volt}$)\cite{jinshan_electrochemical_2004}

\paragraph{Voltage-current-characteristic or polarization curve}
On a polycrystalline metal surface there are sites, such as defects and grain boundaries, where atoms are at higher energy states. In addition, due to arbitrary crystal orientation, there are different crystalline planes with different energy states of atoms on the electrode surface. Therefore, atoms at all these different sites and planes have different standard redox potential $E_O$, and as a result, have different dissolution rates.
%	 according to eq. \ref{dissolution-rate}. 
Such an anodic dissolution will not lead to polishing. Instead, a crystallographic etching is produced (reference [9, 33-35] within \cite{jinshan_electrochemical_2004}). This is true at lower current (or applied potential). This refers to the \textbf{etching regime} in \autoref{fig:oxygen-pitting} with $U<\SI{1.5}{\volt}$.
The plateau where the current remains almost constant with increasing voltage is referred to as \textbf{polishing plateau}. 
		
With continuing increase of applied potential, other reactions than Cu oxidation and reduction may occur and contribute to the increasing current. These reactions produce $H_2$ and $O_2$ bubbles, which occur at or reach the anode surface. This is known as \textbf{oxygen pitting}. Gas (oxygen or hydrogen) bubbles may block $Cu^{2+}$ ion transport and therefore terminate the electrochemical dissolution process on the area inside the bubbles. However, the residual solution on the surface area inside the bubbles may react with Cu atom and result in chemical etching. Depending on the chemical property of the electrolyte solution and the value of current density at which the electrochemical dissolution is occurring, the etching speed can be higher than the rate of electrochemical dissolution. In this case, pits will be produced on the anode and produce a rough surface. If etching does not occur inside the bubbles, or if its speed is slower than that of electrochemical dissolution process, the area inside the bubbles will remain and appears as protruding particles after the electrochemical dissolution process. In either case, a rough surface is produced. Approaches to reduce the effect of oxygen bubbling are done by altering the etching solution with different additives.

Overall, the values of the current plateau and the shape of a polarization curve depends on electrolyte solution, anode material, solution circulation, temperature, and the distance between anode and cathode. Of all the factors, the electrolyte is the most important one determining the polarization curve.

\paragraph{Leveling mechanisms}
Up to now, only the dissolution of copper surface atoms into solution and their deposition at the counter electrode was discussed. If the surface is polished evenly depends on several processes that are known as leveling mechanisms. First of all, the etching process relies on the fact that the current density (and thus the etching rate) is higher in protruding regions of the copper foil (Ohmic leveling). For an evenly wetted copper foil surface, protruding regions have less electrolyte solution between the electrodes. The lower electrical resistance causes a larger current density and thus larger dissolution rates at protruding regions.

Second, the surface geometry determines the shape of the electrostatic potential created at the working electrode. Convex regions at the surface have higher normal electric fields than concave ones, so that the force acting on a ionized copper surface atom is largest at protruding regions. Since the dissolution rate is proportional to the electric field strength, protrusions will be dissolved faster than cavities.\cite{jinshan_electrochemical_2004, Huo_Electrochemistry_2007, luo_effect_2011}

It was shown that best results are achieved with following points fulfilled.\cite{Huo_Electrochemical_2003}
\begin{itemize}
	\item The polarization curve shows a wide limiting current plateau (polishing plateau) with low limiting current.
	\item Etching is performed voltage regulated.
	\item Good circulation of the electrolyte solution to ensure even electro polishing.
	\item Prevent oxygen and hydrogen bubbles created in the etching process to reach the anode surface. Parallel aligned, vertical electrodes are preferred.
	\item Extremely close electrodes increase the effect of ohmic leveling, though gas bubbles reach the anode surface quicker.
\end{itemize}

\paragraph{After etching treatment and storage}
After etching, the samples are rinsed with deionized water to neutralize and remove the remaining etching solution. To seal the samples from ambient moisture and oxygen, they are stored in isopropanol.

%	\paragraph{Removed mass from working electrode}
%	``The current flow of every two electrons results in one copper atom dissolved on the anode and deposited on the cathode. Since $\SI{1}{\ampere}= \SI{1}{\coulomb \second}$, the charge of one electron $e = \SI{1.60218E16}{\coulomb}$, so the number of electrons (per second) in 1 A current is $N_e = \frac{I}{e}$; the number of copper atoms being oxidized or reduced $N_a= \frac{1}{2} N_e= \frac{I}{2e}$, the number of moles $N_m = \frac{N_a}{N_A} = \frac{I}{2eN_A}= \frac{I}{2F}$ where Avogadro's number $N_A = \SI{6.02214E23}{\per \mole}$. The weight of $N_m$ mole copper $W = N_m M = \frac{IM}{2 F}$ where M is the molecular weight of copper. Thats a volume, $V = W / d =\frac{IM}{2 F d}$ where d is the density of copper. Thus a current I produces a dissolution/deposition rate in thickness (\SI{}{\centi\meter \per \second}) \begin{equation} R_d=\frac{M}{2 FdA}I \label{dissolution-rate}\end{equation}
%	where A is the area of the electrode surface.
%	\cite[34]{jinshan_electrochemical_2004}

\subsection{Sample cleaning}
\label{sec:sample-cleaning}
Contaminations 
%in copper (Ag: \SI{0.8}{ppm}, Pb: \SI{0.3}{ppm}, Bi: \SI{0.8}{ppm}) and silver (Cu: \SI{2}{ppm}, Fe: \SI{2}{ppm}, Au: \SI{0.8}{ppm}, Ni: \SI{0.8}{ppm}) 
are removed by repeated sputter\footnote{$U_{accel}=$\SIrange{800}{1000}{\volt}, $T_{sample}\approx \SI{300}{\kelvin}$} and anneal cycles\footnote{Cu: $T_{sample}=\SI{750}{\celsius}$, Ag: $T_{sample}=\SI{450}{\celsius}$, Au\textbackslash Mica: $T_{sample}= 450$} in UHV. While sputter and anneal times are very similar for all substrates ($t_{sputter}\approx$ \SIrange{20}{30}{\minute} and $t_{anneal}\approx \SI{5}{\minute}$ respectively), the annealing temperature is chosen well below the melting point of the substrate but high enough to allow contamination segregation from the bulk to the surface and atomic reordering at the surface. Typical cool down temperatures $\leq 5 \frac{K}{s}$ result in a smooth, atomically flat surface with large terrace size. 

Before CVD growth of \textit{h}-BN on copper the last cleanaing cycle uses annealing temperature in the \textit{h}-BN growth regime to remove contaminations present at growth temperatures.

\subsection{\textit{h}-BN growth}
\label{sec:h-BN-growth}
\textit{h}-BN is grown by catalytic decomposition of the  precursor on hot copper substrates in UHV. Typical borazine partial pressures of $\SI{1e-7}{\milli \bar}$ and evaporation times of $\approx \SI{5}{\minute}$ are used to dose \SI{22}{\langmuir} while the sample is kept at \SI{750}{\celsius}. After dosage, the temperature is kept constant for another \SI{5}{\minute} to ensure complete transition of the precursor into the 2D \textit{h}-BN layer and self-healing of defects created at growth. The sample is cooled down with cooling rates $\leq 5 \frac{\SI{}{\kelvin}}{\SI{}{\second}}$ so that no wrinkles are observed at STM investigation temperatures of $\approx \SI{5}{\kelvin}$.

Borazine is used as chemical precursor. It is stored in an evacuated liquid cooler to maintain temperatures below \SI{5}{\celsius} and to ensure no water contaminants are present. Both, elevated temperature and water contaminants, cause the precursor to degenerate quickly in the storage to form boric acid $H_3BO_3$, ammonia $NH_3$ and hydrogen. Boric acid is a white, solid powder easily recognizable in the storage. A second indication of borazine decomposition is the smell of ammonia. The design can be found in \autoref{sec:borazine-cooler}.

\subsection{Molecule deposition}
\label{sec:molecule-deposition}
%%%%%%%%%%%%%%%%%%%%%%%%%%%%%%%%%%%%%%%%%%%%%%%%%%%%%%%%%%%%%%%%%%%%%%%%%%%%%%%%%%%%%%%%%%%
Molecules are sublimated in UHV by organic molecular beam epitaxy (OMBE). The two/four pocket evaporators resistively heat small quarz crucibles to the chosen temperature while the unused ones are water cooled ($T\leq \SI{20}{\celsius}$). Prior to deposition molecules undergo a degas procedure to remove unwanted chemicals from the powder. A shutter is used to choose the open pocket and allows for accurate timing of evaporation intervals.

\begin{table}\centering
	\caption{Evaporation and degas temperatures used for different molecules.}
	\begin{tabular}{ccrc}
		Name			& Configuration & Degas [\SI{}{\celsius}]	& Evaporate [\SI{}{\celsius}]	\\ \hline \hline 
		TPCN			& ---		& ---		& 490		\\ \hline 
		\multirow{3}{*}{TBP}	&single		& \SI{4}{\hour} @ \SI{200}{}& 390	\\
		&cis		& ---		& \SI{400}{\celsius}\\
		&trans		& \SI{4}{\hour} @ \SI{200}{} + \SI{1}{\hour} @ \SI{270}{}&\SI{370}{}\\ \hline 
		\multirow{4}{*}{pyrene} & \multirow{3}{*}{cis}		& \SI{2}{\hour} @ \SI{180}{}&	\multirow{3}{*}{250}	\\
		&&+ \SI{1}{\hour} @ \SI{200}{} + \SI{10}{\minute} @ \SI{235}{} 	&\\
		&&+ \SI{1}{\hour} @ \SI{220}{}&\\ 
		&trans		& \SI{1}{\hour} @ \SI{230}{}		&\SI{265}{}		\\ \hline
		\multirow{3}{*}{DCDB} & \multirow{3}{*}{---} & 1h @ \SIrange{100}{150}{}& \multirow{3}{*}{\SIrange{220}{240}{}}\\
		&&+\SI{10}{\minute} @ \SI{170}{} + \SI{25}{\minute} @ \SI{200}{} & \\
		&&+ \SI{40}{\minute} @ \SI{220}{}&\\
		\hline
		Helicene & --- & \textcolor{red}{\textbf{1h @ \SIrange{100}{150}{}}} & \\
	\end{tabular}
	\label{tab:molecule-temperatures}
\end{table}

%%%%%%%%%%%%%%%%%%%%%%%%%%%%%%%%%%%%%%%%%%%%%%%%%%%%%%%%%%%%%%%%%%%%%%%%%%%%%%%%%%%%%%%%%%%%%
%%%%%%%%%%%%%%%%%%%%%%%%%%%%%%%%%%%%%%%%%%%%%%%%%%%%%%%%%%%%%%%%%%%%%%%%%%%%%%%%%%%%%%%%%%%%%
\chapter{Epitaxial hexagonal boron nitride on copper foils}
  

\begin{itemize}	
	\item Challenges in mass production (example)
  	  \subitem Ease of use
 	  \subitem Costs
	    \subsubitem How to overcome (example)
\end{itemize}	

For this reason, we focused on the use of cheap substrates to achieve the very same functionality as on expensive single crystals.

\section{Pre-treatment of Cu-foils: Polishing}
  	\paragraph{Experiment realization}The first attempt to etch the Cu-foil was performed with the $5\%_{vol}$ EG, $25\%_{vol}\,H_2O$ filled up with phosphoric acid. The etching was performed in a \SI{200}{\ml} beaker, filled with \SI{150}{\ml} etching solution. The setup is as depicted in fig \ref{fig:etching-setup}. The potential was adjusted to be \SI{1.2}{\V} after some minutes. The current through the solution changes and is typically highest when the etching process started. 

After some minutes, the foil starts to change. The reflectivity changes, making the foil - shiny before etching - a little dull. After some additional time the foil begins to reflect light better again. This is the moment where the etching process is interrupted. The time inside the etching solution depends on the handling (like shaking the beaker or moving the foil in the solution - but was usually $\geq \SI{2}{\hour}$.\footnote{Since the perfect point to perform polishing varies in time a automated etching process has been developed \cite{palmieri_besides_2001}}

One has to be careful if reproducible results are needed. During the etching process (as more and more copper settles on the counter electrode), the current and therefore the etching rate decrease continuously. When the beaker is moved, some of the debris on the electrode changes (the electrode's) surrounding and the etching rate (limiting current) increases again. Front- and backside of the foil are suspect to different etching rates. The back side is generally more flat, the side facing the counter-electrode always shows some additional protrusions.

\paragraph{After etching treatment}
The foil is taken out and cleaned from remaining etchants with purified water first and isopropanol afterwards. Foils are be stored in ethanol to avoid oxidation. 

To further improve the quality of the foil, one can follow the documented recipe for annealing the foil in a $H_2$ atmosphere (\SI{10}{sscm}, \SI{1000}{\celsius}, 30min)\cite{kim_synthesis_2012} to increase the copper grain size and further smoothen the surface. 

So prepared foils are investigated in XPS (compare figure \ref{fig:xps-self-grown}) - (discussion of peaks can be found inside the text. Comparable experiments  are performed by \cite[8]{stables_report_2008}).
After the etching process one foil is investigated in SEM. It was stored for one day in isopropanol and blown dry with nitrogen. 

\paragraph{SEM}

\input{./includes/chapter/sem-technique}

\begin{figure}[]
	\begin{center}
		\includegraphics[height=6cm]{./images/Domenik_16031715.jpg}
		\includegraphics[height=6cm]{./images/Domenik_16031717.jpg}
	\end{center}
	\caption{SEM image of etched copper foil. Different contrast suggests different grain-orientation within the surface. a) Larger (\SI{570}{\micro \meter} x \SI{380}{\micro \meter}) image showing the contrast of different grains in the copper-foil, b) zoom (\SI{18}{\micro \meter} x \SI{12}{\micro \meter})} to a area with two different contrasts and their border.
	\label{fig:SEM-gb}
\end{figure}

One can see (\autoref{fig:SEM-gb}) that the surface imaged in different intensities which correspond to the different orientation of the copper grains within the foil\cite{wu_effects_2015}. The grain size may range from just a few \SI{}{\micro \meter} to several hundret \SI{}{\micro \meter} and in some cases even \SI{}{\milli \meter}. The grains are separated by grain boundaries. Large grains are preferred for growing graphene on copper foils because grain boundaries are subject to inhomogeneities within the graphene layer and provide a route for unwanted surface chemistry (copper oxidation for example). These effect can be also be used to highlight grain boundaries as shown in \cite{wu_effects_2015}.

Although not very rough, the copper foil shows surface variation. While some areas of the sample show some wavy surface, whereas other are much flatter and appear in a different intensity (\autoref{fig:SEM-surface}).

Neither estimation on the grainsize nor guesses for their absolute orientation have been done due to the lack of EBSD-data in the SEM setup.

\begin{figure}[]
	\begin{center}
		\includegraphics[height=6cm]{./images/Domenik_16031700.jpg}
		%\includegraphics[height=6cm]{../Daten/SEM/160317-Domenik/Domenik_16031717.jpg}
	\end{center}
	\caption{SEM image that shows different surface morphologies (\SI{5.6}{\micro \meter}x\SI{3.7}{\micro \meter})}
	\label{fig:SEM-surface}
\end{figure}

\paragraph{AFM}
\begin{figure}[] \centering
	\subfigure[RMS $\approx$\SI{13}{\nm}, contrast \SI{100}{\nm}]{
	\includegraphics[width=0.4\textwidth]{./images/as_bought0000.jpg}
	}
	\subfigure[RMS $\approx$ RMS \SI{9}{\nm}, contrast \SI{70}{\nm} in the right image]{
	\includegraphics[width=0.4\textwidth]{./images/as_bought0001.jpg}
	}
	
	\caption{AFM image of the \SI{0.25}{\mm} copper foil as bought from alfa aesar.}
	\label{fig:foil-afm-as-bought}
\end{figure}

Figure \ref{fig:foil-afm-as-bought} shows the striations that stem from the production process (from top to buttom).
\begin{figure}[] \centering
	\subfigure[RMS $\approx$\SI{9}{\nm} in the left image, contrast \SI{100}{\nm}]{
	\includegraphics[width=0.4\textwidth]{./images/polished0000.jpg}
}
	\subfigure[RMS $\approx$\SI{3}{\nm} in the right image, contrast \SI{70}{\nm}]{
	\includegraphics[width=0.4\textwidth]{./images/polished0001.jpg}
}
	\caption{AFM image of a copper foil polished 5h (according to table \ref{tab:used-etching-solution})}
	\label{fig:foil-afm-polished}
\end{figure}
After etching ($U=1.2V$,I=\SIrange{120}{250}{\mA}) for \SI{5}{\hour} in a solution (shown in table \ref{tab:used-etching-solution}) the striations have gone and the RMS value decreased by \SIrange{30}{45}{\percent} and an increase in foil quality is obvious even with bare eyes. Figure \ref{fig:foil-afm-as-bought} and \ref{fig:foil-afm-polished} show AFM images in the same size and contrast - before and after etching.
The circular hole is an effect of bubbles in the etching process where the bubble affects the rate of etching. The over all structure changes from a heterogenous sample height to a flat height contribution with only a little amount of defects. Those are sufficiently seperated in space to exhibit flat regions where the h-BN may grow unperturbated.



\paragraph{not done yet - maybe future?}
Some foil has been mechanically polished with 4k paper and several hours of Syton polishing. The roughness of these samples has been investigated also in AFM. These are comparable to the chemically polished ones, but are always slightly higher by $\approx 10\%$. Sometimes unwanted new scratches appear after mechanical polish.

\paragraph{STM}
The bought and chemically polished foils are mounted on a sample holder and loaded into the load lock. It is evacuated for \SIrange{2}{3}{\hour}, afterwards the valve is opened to the chamber. During transfer, no noteable increase in the base pressure is noted. The sample is put on the parking slot.

The sample was initially degassed with slowly increasing temperatures to remove adsorbates like $CO, CO_2$ and $H_2O$.

After some time of degassing, the sample was prepared with repeated sputter and anneal cycles. The annealing temperatures were increased up to \SI{800}{\degreeCelsius}. 
After that procedure, the sample was cooled down and observed in STM.
  \section{SEM \& STM of \textit{h}-BN on Cu-foil}
     % Beamtime April 2015
Further experiments were carried out to increase the cleanliness of the \textit{h}-BN on the polycrystalline copper foil. To reduce the amount of elements coming from the body of the foil, it is repeatedly sputtered and annealed to temperatures as high as \SI{800}{\celsius}. This may have also an improving influence on the grain size and amount of corrugation. Several attempts have been made which are described in summary below.
%%%%%%%%%%%%%%%%%%%%%%%%%%%%%%%%%%%%%%%%%%%%%%%%%%%%%%%%%%%%%%%%%%%%%%%%%%%%%%%%%%%%%%%%%%
\section{Characterization of clean copper foil}
\begin{itemize}
 \item After cleaning, the sample is investigated in STM. The foil shows a inhomogeneous topography, with parts of the sample showing very flat regions while others still remain heavily corrugated and not scan able in STM. 
 A first look onto the quite heterogeneous surface reveals flat areas with a typical roughness of $\approx \SI{70}{\pico\meter}$ exist (\autoref{fig:cu-foil-clean}). Areas with very large corrugations $\geq \SI{100}{nm}$ are hard to scan in STM and bad places for \textit{h}-BN growth. Although being flat, the polycrystalline foil shows a lot of unordered substrate steps and a dirty surface, covered with adsorbates imaged as small bright dots.
\end{itemize}
%%%%%%%%%%%%%%%%%%%%%%%%%%%%%%%%%%%%%%%%%%%%%%%%%%%%%%%

\begin{figure}[] \centering
%	\subfigure[Roughness $\approx \SI{60}{\pico\meter}$.]{%
%		\includegraphics[width=0.45\textwidth]{./images/F150331-124839}
%		\label{fig:30-31.03}
%	}
	\subfigure[Cleaned copper foil before \textit{h}-BN growth. Surface shows many facets, the roughness is \SI{70}{\pico\meter}.]{%
		\includegraphics[width=0.5\textwidth]{./images/F150331-125720}
		\label{fig:cu-foil-clean-stm}
	} \quad
%	\subfigure[STM image after \SI{4.5}{\langmuir} of borazine dosage on a \SI{750}{\celsius} hot copper foil surface. A small \textit{h}-BN island can be seen (lower right) on a largely uncovered copper foil background.]{%
%		\includegraphics[width=0.45\textwidth]{./images/F150416-192611}
%		\label{fig:F150416-192611}
%	}
%	\subfigure[STM image of \SI{22}{\langmuir} borazine dosed on a \SI{800}{\celsius} hot copper-foil surface. Several large islands can be seen that grow over Cu-foil step edges. Inset shows coverage with \textit{h}-BN ad layer in blue.]{
%	\includegraphics[width=0.35\textwidth]{./images/F150423-102732-with-inset}
%	\label{cu-foil-hBN-stm}
%}
	\subfigure[Typical height profile. the roughness is \SI{70}{\pico\meter}.]{%
	\includegraphics[width=0.5\textwidth]{./images/F150331-125720-profile}
	\label{fig:cu-foil-clean-profile}
}

\caption{Cu-foil \subref{fig:cu-foil-clean-stm} after repeated sputtering and annealing cycles. \subref{cu-foil-clean-profile} Height profile. Imaging parameters: \subref{fig:cu-foil-clean-stm} \SI{3.6}{\volt}, \SI{0.1}{\nano\ampere}, color scale \SIrange{0}{600}{\pico\meter}, Image width: \SI{88,6}{\nano \meter}, 
%	\subref{fig:F150416-192611} \SI{1}{\volt}, \SI{0.37}{\nano\ampere}, color scale \SIrange{0}{900}{\pico\meter}, Image width: \SI{44,3}{\nano \meter}.
%\subref{{cu-foil-hBN-stm}} \SI{4.7}{\volt}, \SI{0.2}{\nano\ampere}, color scale \SIrange{0}{7}{\nano\meter}, Image width: \SI{295}{\nano \meter}
	}
\end{figure}
%%%%%%%%%%%%%%%%%%%%%%%%%%%%%%%%%%%%%%%%%%%%%%%%%%%%%%%%%%%%%%%%%%%%%%%%%%%%%%%%%%%%%%%%%%
\begin{itemize}
 \item The foil was sputtered and annealed 4 times with temperatures of \SI{800}{\celsius}. Borazine was dosed at \SI{2e-7}{\milli \bar} for \SI{2.5}{\minute}. The sample was kept at \SI{750}{\celsius} for another 5 minutes after dosing. The sample was cooled down slowly. \autoref{fig:h-bn-overgrown-cu-1} shows some of the grown islands. The copper surface changes upon \textit{h}-BN growth and the terrace width increases below the \textit{h}-BN flakes. The typical faceting of the surface vanishes or can at least not be depicted because of the overgrowing \textit{h}-BN (\autoref{fig:h-bn-overgrown-cu-2}). 
\end{itemize}
% -----------BILDER ---- DISKUSSION: 21.04
%\begin{figure}
% \centering
% \includegraphics[width=0.7\textwidth]{./images/F150423-102732-with-inset}
% \caption{}
%\end{figure}
%%%%%%%%%%%%%%%%%%%%%%%%%%%%%%%%%%%%%%%%%%%%%%%%%%%%%%%%%%%%%%%%%%%%%%%%%%%%%%%%%%%%%%%%%%
%\begin{figure}
% \centering
%\subfigure[]{%
%	\includegraphics[width=0.45\textwidth]{./images/F150416-192611-detail1.png}
%	\label{fig:h-bn-overgrown-cu-1}
%} \quad %
%\subfigure[]{%
%	\includegraphics[width=0.45\textwidth]{./images/F150423-114214.jpg} 
% 	\label{fig:h-bn-overgrown-cu-2}
%}%
%\caption{STM topographies of \textit{h}-BN islands that overgrow Cu-foil facets. Imaging parameters: 		
% 	\subref{fig:h-bn-overgrown-cu-1} 
% 		\SI{1}{\volt}, \SI{0.37}{\nano\ampere}, 
% 		color scale \SIrange{0}{1.5}{\nano \meter}, 
% 		Image width: \SI{18}{\nano \meter}, 
% 	\subref{fig:h-bn-overgrown-cu-2} 
% 		\SI{3.5}{\volt}, \SI{0.5}{\nano\ampere}, 
% 		color scale \SIrange{0}{4}{\nano \meter}, 
% 		Image width: \SI{73,8}{\nano \meter}. 
%}%
%\label{fig:h-bn-overgrown-cu}
%\end{figure}

\begin{itemize}
	\item The sample was sputtered and annealed several times to temperatures of \SI{800}{\celsius}. Before the dosage it was held 5 minutes at \SI{750}{\celsius}. Borazine was dosed with the same pressure as before (\SI{1e-7}{\milli \bar}) but for 1min and at a lower temperature of \SI{750}{\celsius}. After the preparation the sample was kept at \SI{750}{\celsius} for another 1 minute. It was cooled down slowly (shown in \autoref{fig:F150416-192611} and \autoref{fig:h-bn-overgrown-cu}).
\end{itemize} 
\begin{figure}
	\centering
	\subfigure[]{%
		\includegraphics[width=0.45\textwidth]{./images/F150423-102732}
		\label{fig:h-bn-22L}
	} \quad %
	\subfigure[]{%
		\includegraphics[width=0.45\textwidth]{./images/F150423-102732-overlayed.png} 
		\label{fig:h-bn-22L-2}
	}%
	\caption{STM topographies of \textit{h}-BN islands that overgrow Cu-foil facets. Imaging parameters: 		
\SI{4.7}{\volt}, \SI{0.2}{\nano\ampere}, color scale \SIrange{0}{7}{\nano\meter}, Image width: \SI{295}{\nano \meter}
	}%
	\label{fig:h-bn-overgrown-cu}
\end{figure}
%%%%%%%%%%%%%%%%%%%%%%%%%%%%%%%%%%%%%%%%%%%%%%%%%%%%%%%%%%%%%%%%%%%%%%%%%%%%%%%%%%%%%%%%%%
points to point out:
\begin{itemize}
 \item Look at Messzeit-April.ppt power point presentation
 \item Stufenh\"ohe
 \item Beschaffenheit der stufen/facetts $\rightarrow$ material transport mechanism/strength differs under the h-BN compared to the bare cu-foil surface.
 \item Wechselwirkung BN-Wachstum und Facettenbildung
\end{itemize}


\paragraph{surface structure of \textit{h}-BN on Cu-foil}
During experiments some ``new'' structure appeared (compare figure \ref{fig:tpcn-on-cu-foil}).
The apparent height change between the both terraces is \SI{130}{\pico \meter} separated by a slim trench that is slightly lower than the right terrace (\SI{50}{\pico \meter}). The parallel stripes have an apparent corrugation of \SI{25}{\pico \meter} and are separated \SI{70}{\pico \meter} from each other and cover the whole image. 

The adsorbed TPCN molecules show different apparent heights in their molecular center. Some fragmented and heavily deformed molecules are visible.

\begin{figure}
 \centering
 \subfigure[Molecules on copper foil surface - supposed be be covered with \textit{h}-BN, maybe just free (maybe facetted) copper. Stripes not visible on the lower teracce, although present in the same orientation.]{
 \includegraphics[width=0.45\textwidth]{./images/F150810-113456}
 \label{fig:tpcn-on-cu-foil-stm}
 } \quad
 \subfigure[Line spectrum across the step shown in \subref{fig:tpcn-on-cu-foil-stm} perpendicular to the trench. No molecules were crossed.]{
	\includegraphics[width=0.45\textwidth]{./images/F150810-113456-line-spectra}
\label{fig:tpcn-on-cu-foil-spectrum}
}
 \caption{Surface structure of copper foil after deposition of TPCN molecules. Two terraces are visible, both covered with linear stripes - oxygen over layer (2x1)?- cu reconstruction? - maybe some very small (\SI{0.75}{\nm}) linear moire on a Cu(100) facet? Noise can be excluded due to the fact that the stripes do not occur on the molecules, but only on the substrate. Adsorbed TPCN molecules appear as cross shaped protrusions. Many deformed molecular cores visible throughout the image $\rightarrow$ strong substrate interaction $\rightarrow$ no \textit{h}-BN! Line spectrum shown in \subref{fig:tpcn-on-cu-foil-spectrum} indicating a period of \SI{70}{\pico \meter}. Imaging parameters: 		
 	\subref{fig:tpcn-on-cu-foil-stm} 
 	\SI{1.26}{\volt}, \SI{0.04}{\nano\ampere}, 
 	color scale \SIrange{0}{0.8}{\nano \meter}, 
 	Image width: \SI{40}{\nano \meter} }
\label{fig:tpcn-on-cu-foil}
\end{figure}

  \section{XPS of self-grown \textit{h}-BN/Cu-foils}
     \input{./includes/chapter/xps-self-grown-foils}
  \section{XPS of bought \textit{h}-BN/Cu-foils}
     The quality of the as-bought \textit{h}-BN on copper foils\cite{_graphene_2014} is examined in XPS.
%%%%%%%%%%%%%%%%%%%%%%%%%%%% make it better looking? %%%%%%%%%%%%%%%%%%%%%%%%%%%%%%%%%%%%%
\begin{figure}
\includegraphics[angle=90,width=1.2\textwidth]{./images/XPS-spectra-as-bought.pdf}
\caption{XPS spectra of as-bought \textit{h}-BN/Cu-foil sample\cite{_graphene_2014}. All spectra are taken at room temperature in as-bought state (black) and after annealing to  \SI{630}{\K} (blue) and  \SI{970}{\K} (red).}
\end{figure}
%%%%%%%%%%%%%%%%%%%%%%%%%%%% 
The XPS spectra shows contribution of different atomic species. There are peaks for the O-atoms (1s: \SIrange{529}{535}{\eV})), C-atoms (1s $\approx \SI{285}{\eV}$), N-atoms (1s $\approx \SI{398}{\eV}$), B-atoms (1s $\approx \SI{190}{\eV}$) and Cu-atoms ($3p_{1/2,3/2}$: \SIrange{70}{80}{\eV})). One would expect the shape of the 1s-peaks to be singlet-like (one peak, gauss shaped) and the 3p-peak to be a doublet (two close lying peaks with area-ratio 1/2:3/2=1:2).

\paragraph{O1s}
Position varies with temperature. The signal at room temperature(black) stems from adsorbed water and CO. These desorp with increasing temperature(blue). When going to higher temperatures(red) this peak increases again and shifts to higher binding energies. Not present in self-grown \textit{h}-BN (figure \ref{fig:xps-self-grown})

\paragraph{C1s}
The C1s Peak decreases with increasing temperature and retains its position. This has the same  reason as for the O1s peak (desorption of CO due to the heating). Some of the carbon remains on the surface - even at temperatures as high as \SI{970}{\K}.

\paragraph{N1s/B1s}
The nitrogen/boron peaks show some temperature related changes. There is little change upon annealing to \SI{630}{\K}, both peaks shrink, but stay almost constant in their position in binding energy (sightly shifted to lower binding energies by about \SI{0.2}{\eV}). Position is [N1s: \SI{398.1}{\eV} | B1s: \SI{190.2}{\eV}]

\paragraph{Cu3p}
The copper peak exhibits an increase in area when increasing the temperature. This is because some of the water and CO adsorbate desorb and more and more copper is contributing to the signal. This peak is a doublet, so both signals come from the same chemical copper surrounding.


The $Cu(OH)_2$ O1s peak is expected to be at \SIrange{531.3}{531.7}{\eV}\cite{deroubaix_x-ray_1992} which may explain the shoulder of the O1s peak to higher binding energies (O1s metal: \SI{531}{\eV}). Nitrates ($NO_3$) have binding energies in the range from \SIrange{532.5}{533.5}{\eV}\cite[45]{wanger_handbook_1979}. This would imply either an replacement of nitrogen with oxygen, or some kind of oxygen on top or below the nitrogen in the BN. As proven by Simonov et al. in \cite{simonov_controllable_2012} the (!atomic!) oxygen tends to replace the nitrogen in the \textit{h}-BN/Ir(111) system when it is annealed to \SI{600}{\degreeCelsius} (compare figure eight therein). Thus it forms $B_{x}N_{y}O_{1-x-y}$ over-layers. The longer the oxidation time the higher the amount of replaced nitrogen (figure two therein). If this effect is responsible for the O1s peak at high temperatures is questionable, since the oxygen has to be cracked somehow - where no process can be thought of (no catalytic cracking at metal surface possible - full ML, thermal energy to low to reach binding energies of $O_2$ (no citation here, nothing found - just a guess)).
%%%%%%%%%%%%%%%%%%%%%%%%%%%%%%%%%%%%%%%%%%%%%%%%%%%%%%%%%%%%%%%%%%%%%%%%%%%%%%%%%%%%
% % This refers to the analysis of the series with unknown temperature reading
% The Cu/B/N-peaks have the expected shape, representing the singlet/doublet structure of the atoms. The O1s peak look different though. The peak should exhibit a single peak, while the recorded spectrum showd a clear double-peak structure. It consists of the expected O1s core level, shifed to lower binding energies and a second contribution, shifted to higher binding energies.
% 
% There are different species expected to be present on the unprepared sample surface. These are namely $CO$/$CO_2$, $CuO$/$Cu_2O$ and $H_2O$. They are availble to the surface due to storage at athmospheric conditions. While hydroxy- compounds shift the O1s-peak to higher BE's, metal oxydes push it to lower BE's \cite{wanger_handbook_1979}. The $H_2O$ peak is expected to be at $\approx \SI{533}{\eV}$ (ausm Kopf - Quelle willi H2O/St($\approx 534$), H2O/Ir (531,9)).
% 
% A contribution of $B_xO_x$/$N_xO_x$ species would result in a broadening of B/N-peaks ($>\SI{191,5}{\eV}$ \cite[6386]{kidambi_observing_2013}) and an increase in the O-signal \cite[6386]{kidambi_observing_2013}. A typical shape of the O-peak for Cu (metal), $CU_2O$ and $CuO$ can be seen in \cite[41]{deroubaix_x-ray_1992}.
%%%%%%%%%%%%%%%%%%%%%%%%%%%%%%%%%%%%%%%%%%%%%%%%%%%%%%%%%%%%%%%%%%%%%%%%%%%%%%%%%%%%
\paragraph{An exchange of O with B or N would be easily visible in XPS (due to changed N/B surroundings. Not sure if the signal of oxygen is large enough for that. Check DATA - confirm maybe}
  \section{Application: Molecular functionalization with TPCN}
     %--------- This is TPCN on h-BN/Cu-foil !
The Cu(111) support for the h-BN growth is reaplaced by a polycrystalline copper foil. The goal is to achieve the same ordering of molecules on the h-BN surface. The h-BN layer has been prepared by a dose of \SI{5E-7}{\milli\bar} borazine for 20min (4500\,L). During dosage the foil has been kept at \SI{820}{\degreeCelsius}.

When a h-BN spacer layer is introduced, the molecules decouple from the substrate, lowering their interaction with the afore-mentioned. This can be seen in a change of the molecules' footprint (rectangular $\rightarrow$ square).

They do not form ordered networks (like chains or squares) and lie rather loosely on the h-BN layer (compare 150807.142226.dat). They can easily be moved with the STM tip (1V, 10nA). In some areas, denser TPCN islands form. Here they lie right next to each other, each slightly shifted to match the neighbouring molecules and to achieve the dense packed regions. The same motiv was already investigated in the same system \cite{urgel_controlling_2015}.

During scanning (I=\SI{0.1}{\nA}, \SI{0.9}{\V} <U<\SI{1.3}{\V} ) of a group of molecules, a single molecules could be pushed out of the group (compare figure \ref{fig:TPCN-manipulation}. While the chain initially consisted of 3 molecules in a row, after scanning one of the molecular units moved to the left while the remaining two stay at their positions. A closer look to the moved molecule's geometry reveals deformation of the legs.

It was shown that the imminic nitrogen species within a 2H-TPP molecule strongly interact a Cu(111) surface, thus orient along high symmetry directions. .\cite{haq_clean_2011, buchner_diffusion_2011, gonzalez-moreno_following_2011, diller_self-metalation_2012, ditze_activation_2012,rojas_self-assembly_2010} Rotation and diffusion are limited.
\begin{figure}[!h]
%Gemessen im Oktober (um den 10ten) ... $$$
 \centering
 \subfigure[Image 1]{
 \includegraphics[width=0.3\textwidth]{./images/manipulation-2}
 }
 \subfigure[Image 2]{
 \includegraphics[width=0.3\textwidth]{./images/manipulation-1}
 }
 \subfigure[Overlay]{
 \includegraphics[width=0.3\textwidth]{./images/TPCN-manipulation}
 }
 \caption{Position change of TPCN group members. Central molecule is manipulated, color indicates its initial (a, green hue in c ) and final (b, red hue in c) position. Image (c) is created via an overlay of two sequential images. The upper and lower molecules do not shift thus sharing the same color.}
 \label{fig:TPCN-manipulation}
\end{figure}

\newpage
\begin{figure}[!h]
 \centering
 \subfigure[Loosely orderd molecules on the h-BN/Cu-foil surface.]{
  \includegraphics[width=0.45\textwidth]{./images/F150807-160006.jpg}
 }
 \subfigure[ Molecules do not always show ordering but in dense areas they do.]{
  \includegraphics[width=0.45\textwidth]{./images/F150807-142226.jpg}
 }
\caption{There they form a motiv like in figure \ref{fig:TPCN-manipulation}a).}
\end{figure}

TPCN without added cobald form similar pattern on the h-BN/Cu-foil system (compare fig. 2b in \cite{urgel_controlling_2015}). Although the ordered areas were quiete rare, an ordered region has been found. Here the molecules are not strictly equi-distant or -rotated which makes it difficult to give an accurate unit cell for this type of motiv.
%--------- Describe how the TPCN form that network on h-BN --------- 

\newpage
\paragraph{Adding Co}
Introducing some cobald (15min, \SI{90}{\celsius}) in the system, this self-assembly changes. The molecules now form a 2D network, too, but are further apart. Their only connection point to the other molecules is the tip of their legs pointing to the adjacent leg of the neighboring molecule.

\begin{figure}[!h]
 \centering
  \subfigure[Zoomed view ($\SI{10}\times\SI{10}{\square\nm}$)]{
  \includegraphics[width=0.45\textwidth]{./images/F150814-090450_01.jpg}
  }
  \subfigure[Zoomed view ($\SI{20}\times\SI{35}{\square\nm}$)]{
  \includegraphics[width=0.45\textwidth]{./images/F150814-115601-cut1-overlay}
  }
\caption{Self-Assembled monolayer for TPCN on h-BN/Cu-foil. The cobald atoms sit right in between the molecules and faciliate a regular, ordered arrangement of the TPCN.}
\end{figure}

No sign for metallation (brighter center of porphine core) or cobald adatoms (bright spots in between the molecules) is observed. Because this type of binding is already reported \cite{urgel_controlling_2015}, similar binding mechanisms are derived for this system.

Molecules arrange periodically with center-center distances of about \SI{2.3}{\nano \meter}. This leaves a little void space in between 4 TPCN molecule's legs, space where a Co atom may be located. This would result in a distance of \SI{1.5}{\angstrom} between the end of a TPCN leg (its N-center) and the center of the cobald atom. Typical binding distances for Co-NC are reported \cite{schlickum_metalorganic_2007, przychodzen_supramolecular_2006} and in good agreement.

%---------- Build models in blender for correct spacings etc. ---------
  \section{Conclusion}
     \input{./includes/chapter/foils-conclusion}
%\printbibliography	
%%%%%%%%%%%%%%%%%%%%%%%%%%%%%%%%%%%%%%%%%%%
%%\chapter{TPCN}
%%  \section{on Cu(111)}
%%    \input{./includes/chapter/TPCN-on-Cu111}
%%   \section{on h-BN on Cu(111)}
%%    % This is TPCN on h-BN on Cu(111).

%%\printbibliography
%%%%%%%%%%%%%%%%%%%%%%%%%%%%%%%%%%%%%%%%%%%%%%%%%%%%%%%%%%%%%%%%%%%%
%%%%%%%%%%%%%%%%%%%%%%%%%%%%%%%%%%%%%%%%%%%%%%
\chapter{Bis- \& Tetra-pyridin-4-ylethynyl functionalized pyrene molecules on \textit{h}-BN\/Cu(111)}
%\section{on h-BN on Cu(111)}
\label{section:pyrene}
\section{Abstract}
The position and number of functional groups of pyridin-4-ylethynyl functionalized pyrene molecules control their self-assembly and electronic properties. To access these in UHV, decoupling of the underlying substrate (Cu(111)) is mandatory and achieved here with a \textit{h}-BN spacer layer. As a result unperturbed HOMO and LUMO states are resolved in STM. While STS shows a band gap that decreases with increasing number of functional groups, it is not affected by their position and the molecular assembly observed STM. The gap between pronounced HOMO/LUMO states is modulated by the electronically corrugated \textit{h}-BN/Cu(111) interface and predominantly determined by the larger shift of the LUMO states. UV/Vis measurements in solution reveal a high quantum yield of the fluorescence emission at wavelengths consistent with ab-initio DFT calculations. Finally, the ability of trans-pyrene to act as host system for cis-pyrene is shown.

\section{Introduction}
Pyrene’s optical properties\cite{Figueira-Duarte_Pyrene_2011} make it a promising candidate for potential applications. 1,3,6,8-tetrasubstituted pyrenes are used in a variety of applications, including blue\cite{Moorthy_Steric_2007,Sonar_pyrenes_2010,Feng_Pyrene_2012}, yellow\cite{Sonar_pyrenes_2010}, green\cite{Chang_efficient_2012} and multilayered\cite{Thomas_pyrene_2012} OLEDs. 

The emergence of these compounds in applications is based on fundamental research in (not only, but including) surface science conducted under conditions where unperturbed photophysical properties of the molecule can be controlled and tuned. 

Optical properties are often investigated on transparent insulating bulk materials (Differential Reflectance Spectroscopy (DRS) - PTCDA/mica)\cite{Proehl_Formation_2004}, (Reflectance anisotropy spectroscopy - $\alpha$-quaterthiophene/potassium hydrogen phthalate)\cite{Bussetti_reflectance_2009} but are also possible on nontransparent HOPG(photoluminescence - quater – (4T) and sexithiophene (6T) films/HOPG)\cite{Schneider_Morphology_2002} or SiO2 (pentacene, perfluoropentacene, and diindenoperylene/SiO2)\cite{Heinemeyer_Real-Time_2010} surfaces. 

These space averaging techniques are complemented with measurements on atomic length scales where sub-molecular topographic and electronic structure are investigated by STM and STS in UHV. These need a conducting support for the molecules to adsorb on, making the choice of a suitable substrate important.

Molecules adsorbed on metal surfaces interact with their support, resulting in luminescence quenching (Photoluminescence - Quaterthiophene and PTCDA on Ag(111))\cite{Gebauer_Luminescence_2004} and broadened frontier molecular orbitals. Mediated by the presence of a metal, these exhibit considerable interaction with low-lying orbitals, changing their shape\cite{Chavy_Interpretation_1993}. To minimize the interaction with a metallic substrate, spacer layers of insulating materials are used. 

Recent years of research have increased the variety of these while a continuous decrease in layer thickness could be achieved. For example, ultrathin (~6) layers of NaCl (pentacene/NaCl)\cite{Repp_molecules_2005} and KCl\cite{Koslowski_adsorption_2017} are utilized for direct imaging of unperturbed molecular orbitals in STM. The thinnest spacer (since it consists only of a single layer of atoms) is provided by a \textit{h}-BN monolayer. Its large band gap is used to minimize interaction with the metallic substrate as shown experimentally\cite{joshi_boron_2012} and theoretically (DFT – silicene/\textit{h}-BN/Cu(111))\cite{Kanno_Electronic_2014}. Here self-assembly and electronic properties can be studied in STM/STS and compared with other unperturbed systems as well as with theory. 

In this work we propose functionalized pyrene building-blocks\cite{Casas-Solvas_Synthesis_2014,Feng_functionalization_2016} for self-assembled regular molecular arrays on surfaces. Functionalization with pyridin-4-ylethynyl\cite{Figueira-Duarte_Pyrene_2011} makes pyrene an versatile agent for controlled self-assembly. Cis- , trans- and tetra functionalized pyridin-4-ylethynyl-pyrene was already investigated on Ag(111) system resulting in one-dimensional coordination chains, two-dimensional arrays and chiral, porous kagom\'e networks, where assembly is controlled by the number and position of substituents.\cite{Kaposi_Supramolecular_2016} 
Hereinafter the influences of self-assembly, leg functionalization\cite{Kurata_donor_2017} and intra moir\'e position\cite{Sushobhan_Control_2014} on the band gap is investigated. Mandatory electronic decoupling is achieved either on \textit{h}-BN/Cu(111) or in solution where the molecules’ density of states is not influenced by a supporting metal.

Pyrenes show interesting optoelectronic properties \cite{Crawford_experimental_2011, Lee_enhanced_2012, Feng_functionalization_2016, Maeda_alkynylpyrenes_2006, Kurata_donor_2017} and their assembly was investigated \cite{pham_self-assembly_2014, matena_aggregation_2010, della_pia_anomalous_2014, pham_comparing_2016}. Here they are used to investigate the influence of the number and position of functional groups on these properties. The very same species have been investigated on Cu(111) albeit data adsorbed on \textit{h}-BN/Cu(111) was lacking up to this point. The nano-pattering effect of the \textit{h}-BN substrate is used here to modulate the wide band gap of the species and therefor their optical properties.

\section{The molecule}

We investigate three different derivatives of pyridin-4-ylethynyl substituted pyrenes shown in \autoref{fig:pyrene-fig1}: \subref{fig:pyrene-fig1-tetra} Tetra-pyrene has four functional groups added at position 1,4,6 and 8. The point symmetric result is therefore neither chiral nor bears a dipole moment. \subref{fig:pyrene-fig1-trans} Trans-pyrene is substituted at the positions 1 and 6 (“equatorial” positions) resulting in a pro-chiral molecule. \subref{fig:pyrene-fig1-cis} Cis-pyrene is functionalized at longitudinal positions 1 and 8. With both electron rich groups being on the same side, they create a permanent dipole moment of 4.1 D in this non-chiral molecule.

\begin{figure}[h!] \centering
\subfigure[]{
		\includegraphics[width=0.25\textwidth]{./images/paper/pyrene/tetra-model}
		\label{fig:pyrene-fig1-tetra}
	}
\subfigure[]{
		\includegraphics[width=0.25\textwidth]{./images/paper/pyrene/trans-model}
		\label{fig:pyrene-fig1-trans}
	}
\subfigure[]{
		\includegraphics[width=0.25\textwidth]{./images/paper/pyrene/cis-model}
		\label{fig:pyrene-fig1-cis}
	}
	\caption{AM1 relaxed functionalized pyrene species in gas phase. Structure of \subref{fig:pyrene-fig1-tetra} tetra-, \subref{fig:pyrene-fig1-trans} trans- and \subref{fig:pyrene-fig1-cis} cis-pyridil functionalized pyrene molecule. All species are virtually flat.}
	\label{fig:pyrene-fig1}
\end{figure}

The 4-ylethynyl functionalization leads to a certain degree of freedom to rotate the group in plane or tilt the pyridil ring around the C-C bond connecting it to the rigid pyrene core.
% There is a graphics in Appendix S13, but not measured by me :/

%\begin{figure}[] \centering
%	\includegraphics[width=0.7\textwidth]{./images/paper/pyrene/figure-2}
%	\caption{Schematic drawing of the frontier Kohn−Sham orbitals for trans- \& tetra-pyridil-ethynyl substituted pyrene derivatives, together with orbital correlation diagram in comparison of the molecular orbitals (MOs) for pyrene itself, at the B3LYP/6-31G** level of DFT.}
%	\label{fig:pyrene-fig2}
%\end{figure}

%To evaluate the effect of the substitution on the pyrene core DFT calculations were performed (B3LYP/6-31G** level of theory, in vacuum). The frontier Kohn-Sham orbitals of pyrene and di- and tetra-substituted pyridylethynyl pyrene are shown in \autoref{fig:pyrene-fig2} and \autoref{fig:pyrene-S4}, \autoref{fig:pyrene-S8}. These show that pyrenes have large orbital coefficients at the 1-, 3-, 6- and 8-positions, with the nodal plane going through the 2- and 7- positions.\cite{Kurata_donor_2017,Maeda_alkynylpyrenes_2006,Diring_Luminescent_2009,Crawford_experimental_2011,Ji_Electron_2015,Lee_enhanced_2012} Consequently, orbital interactions between the pyrene and the pyridylethynyl MOs have an effect on stabilizing the highest occupied (HOMO) and lowest unoccupied (LUMO) molecular orbital energy levels. While the HOMO stabilization plays only a small part, the considerable lowering of the LUMO energy levels lead to smaller HOMO-LUMO gaps. The picture of orbital interactions is similar in cis- and tetra-pyrene, with the HOMO-LUMO gap being influenced mostly by the number of substituents: The gap of tetra-substituted pyrene (\SI{2.54}{\eV}) becomes narrower than that of bi-substituted trans-pyrene (\SI{2.95}{\eV}), which is in accordance to experimental findings (vide infra) and previous literature reports.\cite{Maeda_alkynylpyrenes_2006, Diring_Luminescent_2009, Lee_enhanced_2012}
%%%
As both the electronic and optical properties of molecules are quenched or at least altered upon adsorption on metal surfaces it is necessary to decouple the functional pyrenes from the metallic substrate. Therefor we grow a single layer of \textit{h}-BN on a Cu(111) single crystal. Besides its insulating properties we have chosen this lattice mismatched system (\textbf{citation}) to make use of its spatially modulated surface potential (\textbf{citation}). Depending on the registry of adsorbate (B, N) and substrate (Cu) atoms the surface is divided in regions of larger (pore) and lower (wire) work function. This moir\'e is used here to fine tune the energy of molecular states.
When choosing a bias voltage close to the onset of the LUMO one can easily note molecules in the pore region already contributing to the tunneling current resulting in a bright protrusion in STM although \textit{h}-BN/Cu(111) forms a weakly corrugated layer (\textbf{citation}). 

\section{Results \& Discussion}
After deposition tetra-pyrene molecules assemble in dense packed islands shown in \autoref{fig:pyrene-fig3}. Every molecule within the island shows the same orientation. The unit cell is triclinic (\SI{1.63}{\nano \meter} $\times$ \SI{1.50}{\nano \meter}, \SI{92}{\degree}) and holds a single molecule. The binding motif is guided by the functional groups where a pyridil ring forms a bond with the adjacent pyridil ring of the neighboring molecule. Assuming flat pyridil groups parallel to the surface, the assembly is characterized by two intermolecular distances $d_1$ and $d_2$ between two pyridil groups, $d_1$ ($d_2$) pointing to the long (short) side of the pyrene core as indicated by black lines connecting neighboring nitrogen and hydrogen termini in the inset of \autoref{fig:pyrene-fig3b}. 

\begin{figure}[] \centering
	\subfigure[]{
		\includegraphics[width=0.7\textwidth]{./images/paper/pyrene/figure-3a}
		\label{fig:pyrene-fig3a}
	}
	\subfigure[]{
		\includegraphics[width=0.7\textwidth]{./images/paper/pyrene/figure-3b}
		\label{fig:pyrene-fig3b}
	}
	\caption{STM topography of tetra-pyrene adsorbed on \textit{h}-BN/Cu(111). \subref{fig:pyrene-fig3a} Hexagonal moir\'e superstructure visible as bright protrusions. \subref{fig:pyrene-fig3b} After RT deposition the molecules assemble in dense packed islands in a square unit cell (black square). Molecular model superimposed. Inset (\SI{2}{\nano \meter} $\times$ \SI{2}{\nano \meter}) shows inter molecular distances $d_1$ and $d_2$. 
		Imaging parameters: 
		\subref{fig:pyrene-fig3a} \SI{2}{\volt}, \SI{0.1}{\nano \ampere}, \textcolor{red}{Image width:} and 
		\subref{fig:pyrene-fig3b} \SI{0.1}{\volt}, \SI{0.2}{\nano \ampere}, \textcolor{red}{Image width:}
	}
	\label{fig:pyrene-fig3}
\end{figure}

$d_1$ and $d_2$ equal \SI{0.288}{\nano \meter} and \SI{0.254}{\nano \meter} respectively and are both comparable to experimental\cite[5]{Kaposi_Supramolecular_2016} (\SI{0.27 \pm 0.05}{\nano \meter}) and theoretical\cite[5]{Arras_Nature_2012} (\SI{0.27}{\nano \meter}) results (\textcolor{red}{\textbf{find more of those!}}). Though hard to address quantitatively in STM a small tilt of the pyridil termini is likely present to compensate for the close proximity of the H-terminated pyridil rings and is observed as minor contrast differences within the four legs of the molecule. This tilt may enable interactions between the N terminated pyridil group and the pi system of the neighboring pyridil group, giving rise to an additional binding component beside a strict N-H hydrogen bonding.

%%
In addition to the topographic structure, STS reveals the electronic structure of the assembly. All spectra shown in \autoref{fig:pyrene-fig4b} are taken on the center of a molecule. Colored spectra and points (\autoref{fig:pyrene-fig4a}) indicate their distance to the \textit{h}-BN pore that is recognized as bright protrusion in \autoref{fig:pyrene-fig4b}. The darker the color of the points/spectra the larger the lateral distance to the pore. Molecular orbital energies are indicated by yellow (HOMO) and blue (LUMO) boxes. A blue arrow illustrates the electronic gap between both. Spectra have been fitted with a gauss function after background subtraction to determine the corresponding electronic states. The energy onset of HOMO is located between \SIrange{-1541}{-1583}{\milli \volt} with a dependence on the position within the moir\'e unit cell of 42 mV. The LUMO emerges at 
\SIrange{886}{1198}{\milli \volt}
%(886 ± 1) – (1198 ± 1) mV, 
with a notable shift of \SI{312}{\milli \volt} that compares well to the reported change in work function of the \textit{h}-BN/Cu(111) substrate.\cite{Sushobhan_Control_2014,Liu_Interplay_2015,Schulz_Templated_2013,urgel_controlling_2015} This is in good agreement with literature reporting different level alignment for occupied and unoccupied molecular orbitals where the larger shift of the LUMO mainly determines the electronic gap. It is smaller close to the pores.\cite{Kumar_Molecular_2017} The large gap between HOMO and LUMO (\SI{2.56}{\eV}) benefits molecular orbital resolution (\autoref{fig:pyrene-S4}) and indicates efficient decoupling from the metallic substrate due to the insulating \textit{h}-BN spacer. 

\begin{figure}[] \centering
	\subfigure[]{
	\includegraphics[height=5cm]{./images/paper/pyrene/figure-4b}
	\label{fig:pyrene-fig4b}
	}
	\subfigure[]{
		\includegraphics[height=5cm]{./images/paper/pyrene/figure-4a-mod}
		\label{fig:pyrene-fig4a}
	}
	\caption{Spatial variation of molecular energy states within the moir\'e of tetra-pyrene. \subref{fig:pyrene-fig4a} STS on varying positions within moir\'e unit cell. HOMO and LUMO are indicated by yellow and blue boxes connected by an arrow indicating the electronic gap. A black vertical line is drawn close to the onset of LUMO+1 states at which energy the topography in \subref{fig:pyrene-fig4b}) was recorded. \subref{fig:pyrene-fig4b} STM topography where colored points indicate positions of spectra shown in \subref{fig:pyrene-fig4a}. Imaging parameter: \SI{11.07}{\nano \meter}, \SI{2.05}{\volt}, \SI{0.8}{\nano \ampere}}
	\label{fig:pyrene-fig4}
\end{figure}

In addition to the topographic structure, STS reveals the electronic structure of the assembly. All spectra shown in \autoref{fig:pyrene-fig4a} are taken on the center of a molecule. Colored spectra and points indicate their distance to the \textit{h}-BN pore that is recognized as bright protrusion in \autoref{fig:pyrene-fig4b}. The darker the color of the points/spectra the larger the lateral distance to the pore. Molecular orbital energies are indicated by yellow (HOMO) and blue (LUMO) boxes. A blue arrow illustrates the electronic gap between both. Spectra have been fitted with a gauss function after background subtraction to determine the corresponding electronic states. The energy onset of HOMO is located between \SIrange{-1541}{-1583}{\milli \volt} with a dependence on the position within the moir\'e unit cell of 42 mV. The LUMO emerges at \SIrange{886}{1198}{\milli \volt}, with a notable shift of \SI{312}{\milli \volt} that compares well to the reported change in work function of the \textit{h}-BN/Cu(111) substrate.\cite{Sushobhan_Control_2014,Liu_Interplay_2015,Schulz_Templated_2013,urgel_controlling_2015} This is in good agreement with literature reporting different level alignment for occupied and unoccupied molecular orbitals where the larger shift of the LUMO mainly determines the electronic gap. It is smaller close to the pores.\cite{Kumar_Molecular_2017} The large gap between HOMO and LUMO (\SI{2.56}{\eV}) benefits molecular orbital resolution (\autoref{fig:pyrene-S4}) and indicates efficient decoupling from the metallic substrate due to the insulating \textit{h}-BN spacer. 

\begin{figure}[] \centering
	\subfigure[]{
		\includegraphics[width=0.7\textwidth]{./images/paper/pyrene/figure-5a}
		\label{fig:pyrene-fig5a}
	}
	\subfigure[]{
		\includegraphics[width=0.7\textwidth]{./images/paper/pyrene/figure-5b}
		\label{fig:pyrene-fig5b}
	}
	\caption{Self-assembly of trans-pyrene on \textit{h}-BN/Cu(111). \subref{fig:pyrene-fig5a} STM topography of two homo-chiral domains with minor lateral offset to each other resulting in a narrow transition region between both. \subref{fig:pyrene-fig5b} Enlarged view on one domain with overlaid models and unit cell (black rhombus). Pyridil-pyrene core connections assemble open porous networks of hexagons and triangles (kagom\'e lattice) with well-defined binding distances $d$ (black lines in inset). Image parameters: \SI{1}{\volt}, \SI{0.1}{\nano \ampere}, Image width: \subref{fig:pyrene-fig5a} \SI{24}{\nano \meter}, \subref{fig:pyrene-fig5b} \SI{11}{\nano \meter}}
	\label{fig:pyrene-fig5}
\end{figure}

Reducing the number of substituents by depositing trans-pyrene onto the \textit{h}-BN/Cu(111) surface results in a drastic change in assembly. Now the molecules form open porous networks with a kagom\'e pattern \textbf{(citation)} as shown in \autoref{fig:pyrene-fig5}. The pro-chiral character of the molecule directly translates to the on surface assembly. A result is the formation of mirror domains (\autoref{fig:pyrene-S9}). Each of the two consist of molecules with the same chirality (homo-chiral) while no domains with mixed chirality (hetero-chiral) are observed. Although a major surface area is covered by this regular pattern, homo-chiral domains are connected by narrow intermediate regions to compensate a lateral lattice offset. The connecting region between two domains with opposite chirality is usually larger. The unit cell of the kagom\'e pattern is ?rhombic? with (3.04 x 2.93) nm long unit cell vectors and holds three molecules. The hexagonal pores feature an edge length of about 1 nm. Although not every part of the surface was regularly covered with kagom\'e patterns of either chirality, the binding motif is controlled by N-H interactions between the nitrogen in the pyridil legs and the hydrogen terminated pyrene-core of their nearest neighbor. The distance d = (0.20 ± 0.02) nm (black lines in inset of \autoref{fig:pyrene-fig5}) is smaller compared to the on reported in the dense packed assembly of tetra-pyrene and compares (how) to some literature values … … (\textbf{citation!}). The shortened binding distance builds up stress in the assembly, a reason for the limited domain size of the assembly. For the distance given above, the molecules are assumed to lie flat on the surface – including the pyridil rings. A way to release some of the adlayer stress is to increase the tilt or rotation angle of the pyridil group, both increasing the binding distance.


\begin{figure}[] \centering
	\subfigure[]{
		\includegraphics[height=5cm]{./images/paper/pyrene/figure-6b}
		\label{fig:pyrene-fig6b}
	}
	\subfigure[]{
		\includegraphics[height=5cm]{./images/paper/pyrene/figure-6a}
		\label{fig:pyrene-fig6a}
	}
	
	\caption{Spatial variation of the molecular states within the kagom\'e lattice formed by trans-pyrene on \textit{h}-BN/Cu(111). \subref{fig:pyrene-fig5a} Position and shift of HOMO (yellow) and LUMO (blue). \subref{fig:pyrene-fig5b} STM topography of a region where spectra across a moir\'e pore are taken. Colored points indicate the distance to the pores’ center. Imaging bias voltage is indicated in \subref{fig:pyrene-fig5a} by a black vertical line. Imaging parameter: \SI{11.07}{\nano \meter}, \SI{1.6}{\volt}, \SI{0.2}{\nano \ampere}
	}
	\label{fig:pyrene-fig6}
\end{figure}

To investigate the influence of the number of substituents on the electronic structure, STS is a capable technique. For trans-pyrene the HOMO in located between 
%(-1368 ± 4) and (-1490 ± 3) mV 
\SI{-1368}{\milli \volt} \& \SI{-1409}{\milli \volt}
and a LUMO between 
%(1478 ± 1) and (1816 ± 1) mV.
\SI{1478}{\milli \volt} \& \SI{1816}{\milli \volt}
 The resulting average gap between HOMO and LUMO is \SI{3051}{\milli \volt}. Molecular orbitals on pore positions of the \textit{h}-BN layer are again shifted to lower energies as compared to molecular orbitals on wire positions. The shift in LUMO energy (\SI{338}{\milli \volt}) is larger than the shift in the occupied states (\SI{122}{\milli \volt}). 
ST spectra show a large bandgap in between the HOMO and LUMO (\SI{3.05}{\eV}), which matches DFT results (\SI{2.95}{\eV}) very well. Compared to the electronic gap of the tetra-substituted species, the gap of the bis-substituted trans-pyrene is \SI{378}{\milli \volt} larger, following the trend of increasing electronic gaps with decreasing number of functional groups reflected in the DFT calculations.

\begin{figure}[] \centering
	\subfigure[]{
		\includegraphics[width=0.7\textwidth]{./images/paper/pyrene/figure-7a}
		\label{fig:pyrene-fig7a}
	}
	\subfigure[]{
		\includegraphics[width=0.7\textwidth]{./images/paper/pyrene/figure-7b}
		\label{fig:pyrene-fig7b}
	}
	\caption{Molecular self-assembly upon adsorption of cis-pyrene on \textit{h}-BN/Cu(111). White arrows indicate the preferred growth direction. \subref{fig:pyrene-fig7a} The electronic corrugation of the \textit{h}-BN (moir\'e) is visible as protrusions in LT-STM. \subref{fig:pyrene-fig7b} Overlaid molecular models of the binding motif. Unit cell is sketched as black rectangle. Enlarged inset depicts the binding distances ($d_1/d_2$) within this motif. Image parameters: \subref{fig:pyrene-fig7a} \SI{1.38}{\volt}, \SI{0.02}{\nano \ampere}, \textcolor{red}{Image width:}, \subref{fig:pyrene-fig7b} \SI{1}{\volt}, \SI{0.27}{\nano \ampere}, \textcolor{red}{Image width:}.
	}
	\label{fig:pyrene-fig7}
\end{figure}

To rule out a drastic influence of the assembly on the electronic structure we deposited cis- functionalized pyrenes. These molecules form extended, well ordered, dense packed islands, composed of rows along a preferred growth direction (compare island perimeter and white arrow in \autoref{fig:pyrene-fig7a}). Every row is composed of molecules that interconnect like tooth in a zip fastener. Molecules on one side of the fastener show the same orientation where those in the other half are rotated by \SI{180}{\degree}. The interconnection between both in shown in the inset of \autoref{fig:pyrene-fig7b} and stabilized by two bonds along $d_1$ and $d_2$. The two sides of the zip are bound via N-H binding along $d_2$ (\SI{0.298 \pm 0.01}{\nano \meter}). Within each side of the zip connection a bond between the nitrogen terminated leg and hydrogen terminated core of the trans-pyrene molecules guides chain formation along $d_1$ (\SI{0.284 \pm 0.01}{\nano \meter}, resulting in the above mentioned preferred growth direction. Stability within an island (i.e. perpendicular to the indicated growth direction) is maintained through vdW interactions of a molecules’ passivated pyrene-core to its neighbors’ with an average distance of \SI{0.26 \pm 0.05}{\nano \meter}. A unit cell within the island (black rectangle in \autoref{fig:pyrene-fig7b}, \SI{2.27}{\nano \meter} $\times$ \SI{1.57}{\nano \meter}) is made up of 2 molecules. Binding distances compare well to those present for tetra-pyrene on \textit{h}-BN where extended islands are found, too. The binding distance is comparable to those on Ag(111) as well. \cite{Kaposi_Supramolecular_2016}
The electronic structure of cis-pyrene is studied in STS. Cis-pyrene adsorbed on \textit{h}-BN (\autoref{fig:pyrene-fig8}) show a HOMO located between \SI{-1606}{\milli \volt} \& \SI{-1689}{\milli \volt} and a LUMO between \SI{1242}{\milli \volt} \& \SI{1602}{\milli \volt}. This results in an average gap between HOMO and LUMO of \SI{3.11}{\volt}. Again the shift of the LUMO is more pronounced (\SI{375}{\milli \volt}) than for the HOMO (\SI{84}{\milli \volt}). 

\begin{figure}[] \centering
	\subfigure[]{
		\includegraphics[height=5cm]{./images/paper/pyrene/figure-8b}
		\label{fig:pyrene-fig8b}
	}
	\subfigure[]{
		\includegraphics[height=5cm]{./images/paper/pyrene/figure-8a}
		\label{fig:pyrene-fig8a}
	}
	\caption{\subref{fig:pyrene-fig8a} STS measurements on cis-pyrene on \textit{h}-BN/Cu(111) across a moir\'e unit cell. HOMO (yellow), LUMO (blue) and the bias voltage used in \subref{fig:pyrene-fig8b} (black vertical line). \subref{fig:pyrene-fig8b} STM topography showing moir\'e pores with colored points indicating the positions of spectra in \subref{fig:pyrene-fig8a}. Imaging parameter: \SI{11.07}{\nano \meter}, \SI{1.5}{\volt}, \SI{0.1}{\nano \ampere}.
	}
	\label{fig:pyrene-fig8}
\end{figure}

The effect of increasing number of substituents and the accompanied change in self–assembly affects its electronic structure. For the dense packed motifs (trans- \& cis-pyrene) the band gap increases with decreasing number of substituents (2.56 \& 3.11 eV). Changing the surface tessellation while maintaining the number of substituents (trans- \& cis-pyrene) results in almost no change (3.05 \& 3.11 eV). This indicates that changing the number of substituents has a larger impact on the electronic gap than an increased screening introduced by a larger number of nearest neighbors.

\begin{figure}[] \centering
	\subfigure[]{
		\includegraphics[height=5cm]{./images/paper/pyrene/figure-9a}
		\label{fig:pyrene-fig9a}
	}
	\subfigure[]{
		\includegraphics[height=5cm]{./images/paper/pyrene/figure-9b}
		\label{fig:pyrene-fig9b}
	}
	\caption{Structure of unoccupied frontier orbital. \subref{fig:pyrene-fig9a} STM image recorded close to the onset of the LUMO energy. White dashed lines indicate the perimeter of two molecules. Imaging parameters: \SI{1.5}{\volt}, \SI{0.27}{\nano \ampere}, \textcolor{red}{Image width:}. \subref{fig:pyrene-fig9b} DFT-calculation.
	}
	\label{fig:pyrene-fig9}
\end{figure}

The previous STS measurements show the \textit{h}-BN’s ability to effectively decouple the electronic systems of molecules and metallic substrate. This can be used to image the shape of frontier orbitals directly in STM. Recording STM topography images close to the onset of the LUMO frontier orbital resolution is achieved (\autoref{fig:pyrene-fig8b}). A closer look (\autoref{fig:pyrene-fig9a}) reveals the real space distribution. For better visibility two molecules are outlined in white. One can recognize four central lobes along the short axes of the pyrene core and three dots on the leg positions. Comparing the contrast of these features with DFT calculation suggests a tunneling process where tunneling is indeed mediated by the first unoccupied molecular orbital as calculated by DFT in gas phase (\autoref{fig:pyrene-fig9b}).


%\begin{figure}[] \centering
%	\includegraphics[width=0.7\textwidth]{./images/paper/pyrene/figure-10}
%	\caption{a, b) UV/Vis absorption (solid line) and emission spectra (dotted line) of trans-pyrene (a, orange line) and tetra-pyrene (b, red line) in toluene (c = \SI{e-6}{\mole}) at room temperature. Insets: photographs of toluene solutions under daylight and upon irradiation with a hand-held UV lamp ($\lambda_{ex} = \SI{365}{\nano \meter}$). c, d) Calculated transitions (blue and purple lines), transition dipole moments ($\mu_{eg}$), and oscillator strengths ($f$) as determined by TD-DFT (CAM-B3LYP, 6-31G*).}
%	\label{fig:pyrene-fig10}
%\end{figure}


%The photophysical properties, in diluted toluene solutions, of trans- \& tetra-pyrene are displayed in \autoref{fig:pyrene-fig10}(a) and (b). The absorption spectrum of trans-pyrene shows two main absorption peaks at \SI{416}{\nano \meter} and \SI{394}{\nano \meter}, with another lower absorption band located at higher energy (300 nm). Exciting the low energy absorption peak leads to strong emission at \SI{433}{\nano \meter} and \SI{458}{\nano \meter}, with a quantum yield of \SI{96}{\percent} (determined using Coumarin 153 in ethanol solution as reference). As expected, the tetra-substituted pyrene shows absorption peaks that are bathochromically shifted towards lower energies, with two main bands observed at \SI{463}{\nano \meter} and \SI{438}{\nano \meter} nm and a higher energy, more allowed, peak at \SI{338}{\nano \meter}. Excitation of tetra-pyrene in the lowest energy bands also leads to strong emission, found at \SI{481}{\nano \meter} and \SI{513}{\nano \meter}, with a quantum yield of \SI{84}{\percent}. The same optical transitions seem to be involved in both the absorption and emission processes, as confirmed by a good mirror symmetry of the fluorescent spectra compared to lowest energy absorption transitions. Moreover, the small Stokes shifts (\SI{17}{\nano \meter} and \SI{18}{\nano \meter} for trans- and tetra-pyrene, respectively) are due to the similar geometric structures of the ground and excited states, while the matching excitation and absorption spectra point to an efficient radiative deactivation of the excited state (\autoref{fig:pyrene-S11}).\cite{Diring_Luminescent_2009}
%In relation to the above experimental results, we performed time-dependent density function method (TD-DFT) calculations (CAM-B3LYP/6-31G**, toluene CPCM solvation), \cite{Kurata_donor_2017, Ji_Electron_2015} summarized in Figures 10c, d and S2. The main transitions for trans-pyrene seem to originate from the $H \rightarrow L$ (estimated at $\lambda = \SI{411}{\nano\meter}, f = 1.82$) and $H-1 \rightarrow L / H \rightarrow L+2 transitions (estimated at \lambda = \SI{273}{\nano \meter}, f = 1.10$), with the transition dipole moments aligned along the 1- and 6-positions (towards the pyridylethynyl termini) and the short molecular axes of the central pyrene core. On the other hand, tetra-pyrenes transition dipole moments are aligned along the long and short molecular axes of the pyrene core, with the two main transitions having $H \rightarrow L$ (estimated at $\lambda = \SI{472}{\nano \meter}, f = 1.60$) and $H-1 \rightarrow L / H \rightarrow L+1$ (estimated at $\lambda = \SI{323}{\nano \meter}, f = 2.84$) contributions.

\section{Summary}
Here we investigated a benchmark system for unperturbed pyridil functionalized pyrenes on an inert substrate in UHV. Bis- \& Tetra-pyridin-4-ylethynyl functionalized pyrene molecules are investigated with STM/STS on a \textit{h}-BN/Cu(111) surface as well as by means of UV/Vis spectroscopy in solution and ab initio calculations in gas-phase.
Reminiscent of adsorption on Ag(111)(citation) substrates the pyrene core adsorbs flat on the substrate. The different assemblies on \textit{h}-BN/Cu(111) are a direct result of the stabilizing binding motifs that can be tuned by the number \& position of functional groups. Depended on these design considerations open porous networks and dense packed assemblies are formed on the surface.
TETRA: For tetra-pyrene the dense packed assembly is a result of an interaction solely between pyridil legs. A rotation of the pyridil ring makes its delocalized electronic pi system accessible for the nitrogen terminated pyridil ring of  its neighbors.
TRANS: For trans-pyrene the position of the pyridil legs cause a kagom\'e network solely stabilized by connections between pyridil rings and pyrene cores, constructing a binding to stabilize open, hexagonal pores. Only the pyrene core is assumed to adsorb flat on the surface. The shorter binding distance introduces stress in the adlayer, compensated by the pyridil leg being rotated around the C-C bond and therefor increasing the binding distance. The molecules’ pro-chiral property transfers to the assembly and forms two homo-chiral mirror domains. 
In contrast to the adsorption on Ag(111), where dense packed structures are achieved only at higher coverages(citation), cis-pyrene on \textit{h}-BN/Cu(111) develops extended islands. The reported (citation) Ag-adatom mediated head-to-head coupling motif is not observed here, but only on the bare metal substrate. This is because in contrast to Ag(111) there are no metal ad-atoms available on \textit{h}-BN. The binding motif within the assembly shows both, interaction between rotated legs, and towards flat adsorbed pyrene cores.
STS shows prominent HOMO/LUMO features that are a result of the efficient decoupling from the metallic substrate via the \textit{h}-BN layer. These shift in the same direction but by different amount related to the molecules' position in the moir\'e. The electronic gap is predominantly determined by the larger shift of unoccupied states which are closer to fermi energy on moir\'e pore sites. Hence a smaller gap is created on pores while the gap is larger on wire sites, highlighting the use of \textit{h}-BN as a work function template for adsorbates.
While an increasing number of functional groups reduces the electronic gap in accordance with other reports on similar systems (citation), the assemblies’ effect on the band gap is investigated by bis-substituted pyrene derivatives (trans- \& cis-pyrene). Here the number of substituents remains the same and the assemblies’ effect on the band gap can be investigated. A minor change between open-porous (trans-pyrene, \SI{3.05}{\eV}) and dense packed (cis-pyrene, \SI{3.11}{\eV}) assemblies (\SI{60}{\milli \eV}) can be linked to the increased screening for dense packed assemblies \textcolor{red}{\textbf{(citation for similar systems!)}}.

%Since optical properties are a directly linked to the band gap UV/Vis adsorption measurements are used as complementary technique and reveals peaks at \SI{416}{\nano \meter}/\SI{394}{\nano \meter} (trans) and \SI{463}{\nano \meter}/\SI{438}{\nano \meter} (tetra). Emission spectra show distinct features at \SI{433}{\nano \meter}/\SI{458}{\nano \meter}, \SI{96}{\percent}(trans) and \SI{481}{\nano \meter}/\SI{513}{\nano \meter}, \SI{84}{\percent} (tetra). These correlate well with TD-DFT calculations predicting $H \rightarrow L$ \& $H-1 \rightarrow L / H \rightarrow L+2$ (\SI{411}{\nano \meter} \& \SI{273}{\nano \meter} for trans-pyrene) and $H \rightarrow L \& H-1 \rightarrow L / H \rightarrow L+1$ (\SI{472}{\nano \meter} \& \SI{323}{\nano \meter} for tetra-pyrene) transitions. Ab initio calculations are in agreement with STS and UV/Vis measurements supporting electronically decoupled molecules as well for the \textit{h}-BN/Cu(111) surface in UHV as in solution. They also confirm the trend of increasing band gap with decreasing number of substituents as found in STS. 

\section{Conclusion}
The novel, optically active bis- \& tetra-pyridin-4-ylethynyl functionalized pyrene molecule self-assembles on the isolating \textit{h}-BN/Cu(111) surface in networks on the \SI{}{\nano \meter} scale that are stabilized by an attractive interaction of pyridil legs. While cis- \& tetra-functionalization results in close packed assemblies, trans-functionalization leads to open porous networks on \textit{h}-BN/Cu(111) that may be further investigated regarding their role as host for different molecules. The electronic gap is decreasing with increasing number of substituents. The electronically corrugated \textit{h}-BN/Cu(111) interface acts as work function template for the adsorbed molecules where the size of the gap is related to the adsorption site within the moir\'e. 
%The optical properties of the molecules are addressed by means of UV-/Vis spectroscopy and show distinct fluorescence emission with high quantum yield that are applicable in a variety of devices and applications.
%\printbibliography
%%%%%%%%%%%%%%%%%%%%%%%%%%%%%%%%%%%%%%%%%%
%%%%%%%%%%%%%%%%%%%%%%%%%%%%%%%%%%%%%%%%%%%%%%%%%%%%%%%%%%%%%%%%%%%%
\chapter{Borazine functionalized coronene}
Coronene ($C_{24}H_{12}$, known as [6]circulene or superbenzene) is a polycyclic aromatic hydrocarbon made of six carbon rings to form a molecule reminiscent of a small graphene flake. It belongs to the family of circulenes where a central polygon is enclosed by different numbers of fused benzenoids. For example [5]circulene (corannulene), [6]circulene (coronene), [7]circulene,[2][3][4][5] and higher orders could be synthesized and show different conformation. While species with [5] or [7] benzene rings are bowl shaped, coronene is flat.

\section{HBC \& HBBNC}
% This is HBBNC and HBC
\label{section_HBBNC}

\begin{wrapfigure}{R}{5cm}\centering
	\includegraphics[angle=90,width=5cm]{./images/molecules/max-zoom/HBBNC-600}
	\caption{HBBNC}
	\label{fig:HBBNC-molecule}
\end{wrapfigure}

HBC and HBBNC are modifications of coronene. First, for both species six benzo groups are added to form a larger molecular backbone. For both species three 2,6-Dimethylphenyl groups are added to extend the molecule that now resembles a triangular footprint. While HBC features a central carbon ring, HBBNC is functionalized with a central borazine ring instead.
Both species have the same number of atoms and molecular weight. The difference between both becomes apparent when electronic properties are compared (in gas phase).


The regular covalent sp2 hybridization results in an evenly distributed electron density in HBC where the central region of the molecule shows considerable depletion. Changing the central carbon ring to a borazine ring changes the electron density. Now electrons are redistributed from the coronene parts towards the central borazine ring. Because the bond between B and N shows an added ionic character the aromacity is interrupted and the extended electron pi system is altered. Comparable to the difference between graphene (perfect C-C bonds, conductor) and h-BN (Ionic B-N bonds, insulator) the band gap present for HBC is 0.4 eV smaller than for HBBNC, changing its optoelectronic properties.


\begin{figure}[]\centering
		\includegraphics[width=0.7\textwidth]{./images/Dosso-combined}
	\caption{Taken from \cite{dosso_synthesis_2017}}
	\label{}
\end{figure}

The present functionalization of the coronene molecule is twofold. Di-Methylphenyl groups are added to guide the formation of self-assembled islands of the molecule on the surface. The functionalized core of the molecule is used to create an adsorption platform for polar molecules.

After RT adsorption of HBBNC on Ag(111) different assemblies are found. For low coverage the dominating pattern is a hexagon made up of six molecules in two different orientations with respect to the substrate. Although the molecule is not chiral in gas phase, adsorption on the Ag(111) surface leads to the formation an mirror image and therefore a second type of hexagon. The internal structure of these hexamers can be revealed to lateral manipulation of a hexamer with the STM tip. Three things can be concluded: 1. A hexagon is made up of six intact molecules with alternating orientations, 2. The bright features between two neighboring molecules can be attributed to rotated dimethylphenyl rings. This only occurred when two molecules are close to each other and show the right rotation, 3. Molecules not incorporated in a hexamer appear flat with no pronounced apparent height above the legs.

\begin{wrapfigure}{R}{5cm}\centering
	\includegraphics[width=5cm]{./images/hbbnc-ag-111-rt}
	\caption{}
	\label{}
\end{wrapfigure}


From the observation of smaller hexamer fragments, it seems like the growth mechanism of the hexamers is already fixed in an early state of the assembly and depends on the adsorption site of the second molecule attaching to the first. Two neighboring molecules never show the same orientation and connect to each other with parallel edges, slightly shifted.
Molecule by molecule than arranges to match the steric restrictions from the already formed hexamer. This efficient guiding mechanism leads to most molecules finish hexamer assemblies. 
Besides the dominant motif monomers, dimers and smaller agglomerations are observed, too.


The electronic structure of single HBBNC molecules is investigated with STS after disassembly of a hexamer into its comprising single molecules. There is a pronounced electronic feature around 650 mV on the molecular center, features at 1200 mV and 1600 mV can be attributed to the leg and edge positions respectively. The surface state of the substrate (~ -50mV next to the molecule) vanishes/shifts below the molecule. The calculated band gap of 2.52 eV is not observed directly because …
The fact that the band gap is not symmetric around the Fermi energy is because….
Charge transfer between molecule and substrate is possible? What would be the result? Calculate!
Why/how does the surface state shift?

%\FloatBarrier
%\newpage

\begin{wrapfigure}{R}{5cm}\centering
	\includegraphics[width=5cm]{./images/hbbnc-ag-111-rt-linespectrum}
	\caption{}
	\label{}
\end{wrapfigure}

Increasing the coverage leads to a new assembly being formed. Besides hexamers alongside their chiral counterpart form on the surface (solid and dashed blue circle), chains of dimers assemble in islands. Within these, dimer chains exist in two different orientations (solid and dashed green boxes) and are separated by single molecules (see supporting information). 
These chains are oriented along the high symmetry directions of the substrate and 30° off. Molecular orientation within the chains is slightly different than in the hexamer assemblies. Here molecules attach with a smaller lateral shift to form linear chains. 
Tell about
binding distances and unit cell.
The protrusions between two neighboring molecules can be attributed to a rotation of the leg functionalization to avoid steric hindrance and to stabilize the assembly.
A second new binding motif is observed (white square).  It is made up of four molecules, whose close proximity in the center form pattern reminiscent of a clover-leaf. It can be seen that (in contrast to hexamers) molecules can attach to the rims and evolve into more extended structures.

\begin{figure}[] \centering
	\includegraphics[width=0.7\textwidth]{./images/hbbnc-ag-111-rt-med-coverage}
	\caption{}
	\label{}
\end{figure}

Further increasing the coverage results in dense regions being formed. The dominating pattern is the clover-leaf rarely observed in the medium coverage phase. It is present here in two different orientations (white and blue box) and distributed such that neighboring squares do not show the same orientation. Two squares with the same orientation (white/blue) are separated by lines made up of four bright spots aligned parallel to the square edge. Squares with different rotations do not show these connections between them, but a single protrusion with larger apparent height.
Binding motif, binding distances.


\begin{figure}[] \centering
	\includegraphics[width=0.7\textwidth]{./images/hbbnc-ag-111-rt-high-coverage}
	\caption{}
	\label{}
\end{figure}

A sample prepared at RT has been annealed to \SI{350}{\celsius} and \SI{420}{\celsius}.

After the sample is annealed to 350°C only monomers and few random agglomerates are imaged on the surface. Although the molecules undergo the same temperature range where hexamers are formed (from RT to 5K) no regular assembly is imaged. Because the assembly is guided by the presence of the dimethylphenyl group a closer look to the molecular conformation is taken. Three different types can be distinguished. 1: Flat molecules, 2: Molecules with a single protrusion close to the leg position, 3: Molecules with a leg missing.
The unstable imaging conditions above some of the molecules legs indicate their flexibility while the others seem to be rigidly connected to the molecular backbone and therefor imaged stable. The vanishing hexagon binding motif - observed for the un-annealed sample - underpins the importance for the molecules leg to adopt the assembly. This flexibility of all legs is not present any more after annealing to 350°C and hexamer formation is suppressed. Molecular orientation?
Increasing the annealing temperature to 420°C results in a percolated network where monomers present at 350°C coalesce and connect via their legs. Lateral manipulation attempts have been done, showing a stiff connection between neighboring molecules. Opposing to the previous preparations, the assembly could not be divided into single monomers. This rigid connection indicates a covalent coupling of the molecules. 
Molecular orientation?

\begin{figure}[] \centering
	\includegraphics[width=0.7\textwidth]{./images/hbbnc-ag-111-annealed}
	\caption{}
	\label{}
\end{figure}

Since the formation of a new bond is expected to change the core levels of participating elements (Carbon, C 1s), XPS measurements are done and presented in the following.

Preparations have been done to further investigate the connection between molecules annealed to 420°C. Sub-ML as well as multilayer preparations have been annealed to 420°C to quantify the change in binding energy. After RT deposition of a sub-ML HBBNC on Ag(111) a single C 1s peak is observed that grows with increasing coverage but maintains its position.
Is there no shift after adsorption of a multilayer?
After annealing a multilayer preparation the C 1s position shifts to lower binding energies by (~1eV), a behavior typical for cyclodehydrogenation and ring closure reactions of e.g. porphins (CITATION). What are other mechanisms for a shift towards lower binding energies => screening should change in monolayer/multilayer regimes? The area below the peak drops to a sub-ML coverage, a clear indication for desorption of the second and third layer. A covalently coupled layer would not desorp, so coupling in the lowest layer takes place after multilayer desorption.
It can be concluded that annealing HBBNC on a Ag(111) surface leads to the formation of a covalent network stabilized via covalent bonds formed across the legs.

\begin{figure}[] \centering
	\includegraphics[width=0.7\textwidth]{./images/hbbnc-xps1}
	\caption{}
	\label{}
\end{figure}

To further check the conformational changes in the molecule after annealing, AFM measurements are done. nc-AFM has the big advantage over STM that it is less sensible to electronic changes in the molecule – more closely resembling the true geometric shape.

Nc-AFM measurements are a complementary method to investigate the molecules before and after annealing to 420°C. 

Before annealing
Before annealing the adsorption geometry of monomers is investigated. By scanning the same molecule in different heights, elevated parts of the molecule can be easily distinguished by their larger interaction force with the tip (the involved larger frequency shift is shown as protrusion in nc-AFM images). In contrast to STM measurement where no obvious change in apparent height was observed at RT between legs, AFM measurements show that the dimethylphenyl legs of a monomer do not lie flat on the surface but have an elevated and lower lying part. While the initial orientation on the surface is likely determined at adsorption, the legs are able to rotate under the influence of the AFM tip (see SI).

After annealing
IS that a bond? What does the contrast mean? CITE; CITE, CITE
The triangular molecules that appeared flat in STM (after annealing to 420°C) reveal their interesting geometric properties when investigated by means of nc-AFM. It is observed that many of the molecules appear to have their dimethylphenyl groups aligned planar to the surface. A behavior expected for a ring closure reaction between the hexabenzol groups and dimethylphenyl legs. EXAMPLES for AFM showing that in other systems. In the present case, almost all molecules showed some defined contrast in the connection region. DISCUSS distances, directions and excess of carbon rings.
When the legs rotate parallel to the surface, the molecular backbone comes closer to the metal => charge transfer/screening and such things!!

\begin{figure}[] \centering
	\includegraphics[width=0.7\textwidth]{./images/hbbnc-annealed-afm}
	\caption{}
	\label{}
\end{figure}

To have the molecule adsorbed more even on the surface, another substrate is chosen. Silver is known to have a larger impact on adsorption geometries than Au(111) has (bowl shape of PAH on Ag(111) => CITE!).

In contrast to adsorption on Ag(111) where almost all molecules are incorporated into hexamers, molecules separate into monomers on Au(111). Molecules follow the herringbone reconstruction visible as bright stipes in the STM image. 
(These divide the surface into regions of fcc and hcp reconstruction.
Which one is broad, which one is small)
Molecules show two different appearances. While some appear to adsorb flat in STM, others already show a protrusion above one of their legs. This protrusion is not caused by parts of the herringbone reconstruction, two molecules on very similar adsorption sites within the reconstruction show different appearances.
What are conformational changes here?

\begin{figure}[] \centering
	\includegraphics[width=0.7\textwidth]{./images/hbbnc-au-111-rt}
	\caption{}
	\label{}
\end{figure}

\paragraph{HBC}It is reported that for HBC, no stable second layer of molecules can be found for a strong electron donor like HBC \cite{de_feyter_two-dimensional_2003}.
%%%%%%%%%%%%%%%%%%%%%%%%%%%%%%%%%%%%%%%%%%%%%%%%%%%%%%%%%%%%%%%%%%%%
\chapter{Nitro functionalized Porphine}
    Within this section, TBP molecules with different amount and alignment of their functional group(s) is investigated. Please refer to \autoref{chapter:used-molecules} for detailed information on the molecules.


  \section{Single leg functionalization}
   \subsection{on Cu(111)}
      % TBP on Cu(111)
\label{sec:single-TBP-Cu111}
\begin{wrapfigure}{r}{5cm}\centering
	\includegraphics[angle=90, width=5cm]{./images/molecules/max-zoom/TBP-single-600}
	\caption{TBP with three di-tert-butyl and a single nitro phenyl group added at the meso position.}
	\label{fig:}
\end{wrapfigure}
When adsorbed at room temperature, TBP distributes equally on the surface, forms unordered islands and decorates step edges. Molecules orient their main axis (connecting line from one di-tert-buytl-phenyl ring across the center to the nitrophenyl ring) along the dense packed substrate rows most often, less are \SI{15}{\degree} of \textcolor{red}{\textbf{((Refer to image?))}}. Several binding motifs (as shown in \autoref{fig:binding-motifs-TBP-Cu111}) are observed, namely
\begin{itemize}
 \item A dimer, where molecules lie ``head-to-head'', functional groups ($NO_2$) pointing at each other
 \item A ``triangle'', where molecules are rotated \SI{120}{\degree} and functional groups point towards a shared center. Although this motif does not occur very often (or at least under very flexible angles), it is given as an example where the functional groups point to each other. Similar motifs (like 3 molecules in \SI{90}{\degree} are observed together with other orientations. 
 \item Chains with different length appear, where the nitro group of molecule 1 points to the di-tert-butyl group of molecule 2 (``head-to-tail''). At the connection points, molecules appear brighter, promoting a physical overlap of the two molecules.
\end{itemize}

Center-center distances are typically \SI{1.78}{\nano \meter} (for the head-to-tail) and \SI{1.5}{\nano \meter} for the head-to-head connection. 

\begin{figure}[]
	\centering
	\subfigure[Overview image]{
		\includegraphics[width=0.7\textwidth]{./images/F151128-083339-44nm}
		\label{fig:single-TBP-cu111-overview}
	}
	
	\subfigure[Monomer]{
		\includegraphics[width=0.22\textwidth]{./images/F151128-083339-monomer-3nm.png}
		\label{fig:single-TBP-cu111-monomer}	
	}
	\subfigure[Chain]{
		\includegraphics[width=0.22\textwidth]{./images/F151128-083339-chain-5nm}
		\label{fig:single-TBP-cu111-chain}	
	}
	\subfigure[Dimer]{
		\includegraphics[width=0.22\textwidth]{./images/F151128-083339-dimer-5nm.png}
		\label{fig:single-TBP-cu111-dimer}	
	}
	% \subfigure[Model representation of the binding motifs. See text for more details.]{
	%  \includegraphics[width=0.4\textwidth]{./images/TBP-motivs-on-Cu111}
	% }
	\caption{RT adsorbed single nitro functionalized TBP on Cu(111) and their most abundant binding motifs. \subref{fig:single-TBP-cu111-overview} Each of the binding motif can be found as well in the overview STM data, as well as in the enlarged images (b-d). All images recorded with \SI{-500}{\milli\volt}, \SI{0.1}{\nano\ampere}, color scale \SIrange{0}{300}{\pico\meter}. Image width: \subref{fig:single-TBP-cu111-overview} \SI{44}{\nano\meter}, 
		\subref{fig:single-TBP-cu111-monomer} \SI{3}{\nano\meter}, 	
		\subref{fig:single-TBP-cu111-chain} \& 
		\subref{fig:single-TBP-cu111-dimer} \SI{5}{\nano\meter}
	}
	\label{fig:binding-motifs-TBP-Cu111}
\end{figure}

\paragraph{``head-to-head''}
To model the occurring binding motifs, deformations of the molecules have to be taken into account. Because nitro groups face each other in the ``head-to-head'' connection, their distance would be to small to facilitate a similar binding mechanism like for the TPCN on copper (where copper surface ad atoms promote binding between nitrogens), so no free space between the facing nitro groups is observed. Because the distance is so small, the phenyl ring (with attached nitro group) rotates by \SI{45}{\degree}, to make the phenyl ring stand upright. When the second molecule does the same, both match each other with negligible lateral shift, reproducing the STM images best. Similar binding motivs are reported in \cite{kato_dispersive_2008} for non-covalent cross linking of dicarboxylic acids in hydrogels. Although the situation on a metal-surface may change considerably (only 2D - no 3D, metal present - will change chemistry), the observed binding motif matches very well.

\paragraph{``head-to-tail''}
The chain motif ``head-to-tail'' is reconstructed using the unique contrast of the TBP molecule. When the center-center distance is measured, molecules are modeled that distance away from each other. These models show a physical overlap between molecules, which in not possible because of steric hindrance. To solve the problem, the nitro-group (head) of one molecule if rotated by \SI{35}{\degree} out of the plane (like pulling the nitro-group upwards, not rotating the group left/right). 

%---------------- models bauen und bsp bilder einf\"ugen.  ---------------- 

\paragraph{Flexible tert-butyl-groups}
\begin{figure}\centering
	\includegraphics[width=0.44\textwidth]{./images/F151128-083339-10x4-overlay.png}
	\caption{Different appearances of TBP on Cu(111). While most of monomers (center in image above) show even heights with their tert-butyl functions, some (left) do posses an elevated tert-butyl group. The orientation of the tert-butyl groups is aligned with the high symmetry crystal direction (indicated by white lines) most often. Image recorded with  \SI{-500}{\milli\volt}, \SI{0.1}{\nano\ampere}. Image width: \SI{10}{\nano \meter}}
	\label{fig:TPB-butyl-flexibility-SMT}
\end{figure}

Another interesting fact is that butyl groups of TBP seem to orient them self (as far as steric hindrance allows for) along the dense packed rows of the copper substrate. Again, one has to be careful when reconstructing geometrical information from STM images. Like the distortion of legs in the TPCN molecule, this rotation can be explained by a rotation of single butyl groups. Although the phenyl ring remains at the same position/rotation, tert-butyl groups are allowed to rotate such that they appear in different heights. Because STM (constant current) follows equipotential lines, the whole phenyl-di-tert-butyl-complex looks rotated in plane, although it may not be. This is confirmed in literature\cite{heim_surface-assisted_2010, heim_self-assembly_2010}.

%%%%%%%%%%%%%%%%%%%%%%%%%%%%%%%%%%%%%%%%%%%%%%%%%%%%%%%%%%%%%%%%%%%%%%%%%%%%%%%%%%
   \subsection{on Ag(100)}
      % This is TBP on Ag(100):
\label{sec:single-TBP-Ag100}
%%%%%%%%%%%%%%%%%%%%%%%%%%%%%%%%%%%%%%%%%%%%%%%%%%%%%%%%%
Molecules are adsorbed on Ag(100) at RT. The resulting conglomerates are shown in \autoref{fig:single-TBP-Ag100-RT}. The very most surface area is covered with unregular patterns. The step edges are covered, assuming a sufficient large mobility at RT to move from the terrace to the nearest step edge. The only free step edges observed are due to tip formings on the sample surface since these are created after the molecules are stuck on the surface because of the low temperatures during measurement.
%%%%%%%%%%%%%%%%%%%%%%%%%% Annealing %%%%%%%%%%%%%%%%%%%%%%%%%%%%%%%%%%%%%
\paragraph{Annealing}
The RT adsorption is annealed to \SI{170}{\celsius} for \SI{10}{\min} and investigated in LT-STM again (\autoref{fig:single-TBP-Ag100-annealed}). No big changes are visible, neither in the formation of new assemblies nor in the distribution of molecules at terraces or at step edges. No chain formation could be observed.

\begin{figure}[] \centering
	\subfigure[Adsorption at RT]{\includegraphics[width=0.45\textwidth]{./images/F150615-121334-cut.png}
		\label{fig:single-TBP-Ag100-RT}
	}
	\subfigure[RT adsorption annealed to \SI{170}{\celsius} for \SI{10}{\min}]{\includegraphics[width=0.45\textwidth]{./images/F150616-102758-44nm.png}
		\label{fig:single-TBP-Ag100-annealed}
	}
	\caption{Annealing after RT adsorption of molecules on Ag(100). \subref{fig:single-TBP-Ag100-RT} STM data of molecules adsorbed at RT (Scan parameters: $U_b=\SI{1}{\volt}, I_t=\SI{0,03}{\nano \ampere}$), \subref{fig:single-TBP-Ag100-annealed} After annealing for \SI{10}{\min} to \SI{170}{\celsius} (Scan parameters: $U_b=\SI{1}{\volt}, I_t=\SI{0,1}{\nano \ampere}$). Color scale \SIrange{0}{600}{\pico\meter}. Image width: \SI{44}{\nm}.}
	\label{fig:single-TBP-Ag100-annealing}
\end{figure}

%%%%%%%%%%%%%%%%%%%%%%%%%% Assembly models %%%%%%%%%%%%%%%%%%%%%%%%%%%%%%%%%%%%%
\paragraph{Assembly}
Since no regular self-assembled islands are present on the surface, more detail is put on the only repeating binding motifs on this surface. One of this configurations resembles a cross (\autoref{fig:single-TBP-Ag100-cross}), while the second one is a variation of the dimer motif (\autoref{fig:single-TBP-Ag100-doubledimer}).

%%%%%%%%%%%%%%%%%%%%%%%% Dimer %%%%%%%%%%%%%%%%%%%%%%%%%%%%%%%%

While on copper, two molecules may form a dimer in head-to-head of head-to-tail configuration, on silver some form tetramers from two parallel merged dimers. While one dimer looks like two ``U'''s with facing open ends ($\in \ni$), the other dimer is shifted to closely match the first dimer best and lies parallel.

%\begin{figure}[] \centering
%
%	\caption{Dimer configuration of TBP adsorbed on Ag(100) at RT. \subref{fig:single-TBP-Ag100-dimer} STM data. Scan parameters: $U_b=\SI{0.328}{\volt}, I_t=\SI{0.035}{\nano \ampere}$, color scale \SIrange{0}{300}{\pico\meter}. Image width: \SI{5}{\nm}. \subref{fig:single-TBP-Ag100-dimer-model} Model representation in the same size.}
%	\label{fig:single-TBP-Ag100-dimer}
%\end{figure}

%%%%%%%%%%%%%%%%%%%%%%%% Double Dimer %%%%%%%%%%%%%%%%%%%%%%%%%%%%%%%%
\begin{figure}[] \centering
	\subfigure[]{  \includegraphics[width=0.3\textwidth]{./images/F150612-153409-5nm.png}
	\label{fig:single-TBP-Ag100-dimer-STM}
	}
	\subfigure[]{  \includegraphics[width=0.3\textwidth]{./images/F150612-144915-6nm.png}
		\label{fig:single-TBP-Ag100-doubledimer-STM}
	}
	\subfigure[]{\includegraphics[width=0.3\textwidth]{./images/F150612-154558-10nm.png}
	\label{fig:single-TBP-Ag100-cross-STM}
	}
	\subfigure[]{  \includegraphics[width=0.3\textwidth]{./images/F150612-153409-5nm-model.png}
	\label{fig:single-TBP-Ag100-dimer-model}
	}
	\subfigure[]{  \includegraphics[width=0.3\textwidth]{./images/F150612-144915-6nm-model}
		\label{fig:single-TBP-Ag100-doubledimer-model}
	}
	\subfigure[]{\includegraphics[width=0.3\textwidth]{./images/F150612-154558-10nm-model3.png}
	\label{fig:single-TBP-Ag100-cross-model}
	}
	\caption{Different observed binding configurations of TBP adsorbed on Ag(100) at RT. \subref{fig:single-TBP-Ag100-dimer-STM} STM data of dimer configuration. Scan parameters: $U_b=\SI{0.328}{\volt}, I_t=\SI{0.035}{\nano \ampere}$, Image width: \SI{5}{\nm}. \subref{fig:single-TBP-Ag100-dimer-model} Model representation. \subref{fig:single-TBP-Ag100-doubledimer-STM} STM data of two coalescent dimers. Scan parameters: $U_b=\SI{0.097}{\volt}, I_t=\SI{0.035}{\nano \ampere}$, Image width: \SI{6}{\nm}. \subref{fig:single-TBP-Ag100-doubledimer-model} Model representation. \subref{fig:single-TBP-Ag100-cross-STM} A cross consisting of four TBP molecules. Scan parameters: $U_b=\SI{2.3}{\volt}, I_t=\SI{0,035}{\nano \ampere}$, Image width: \SI{10}{\nm}. \subref{fig:single-TBP-Ag100-cross-model} Model representation. Color scale in all STM images \SIrange{0}{300}{\pico\meter}}
	\label{fig:single-TBP-Ag100-doubledimer}
\end{figure}
%%%%%%%%%%%%%%%%%%%%%%%% Cross %%%%%%%%%%%%%%%%%%%%%%%%%%%%%%%%
Another motif looks like a cross and shown in \autoref{fig:single-TBP-Ag100-cross}. Build out of four molecules, wherer each is rotated by \SI{90}{\degree} with respect to its preliminary neighbor. One can distinguish four di-tert-butyl groups from the central cross. Although there is no atom directly in the center, the cross looks bright in its center (in STM), which is somehow counterintuitive. 

%\begin{figure}[] \centering
%
%	\caption{\subref{fig:single-TBP-Ag100-cross-STM} A cross consisting of four TBP molecules. Scan parameters: $U_b=\SI{2.3}{\volt}, I_t=\SI{0,035}{\nano \ampere}$, color scale \SIrange{0}{300}{\pico\meter}. Image width: \SI{10}{\nm}. \subref{fig:single-TBP-Ag100-cross-model} Model representation in the same size.}
%	\label{fig:single-TBP-Ag100-cross}
%\end{figure}

\paragraph{Flexible Tert-Butyl-Functions}
\autoref{fig:single-TBP-Ag100-doubledimer-STM} shows an interesting feature of the tert-butyl functions.

\begin{itemize}
 \item Butyl groups within TBP feature different contrasts (look rotated), while the orientation of the butyl-groups doesn't follow the close packed substrate rows. ---------------- find image and explain
 \item TBP molecules have been heated on silver substrate for \SI{10}{\minute} at \SI{170}{\celsius}. The resulting sample did not feature chain-formation or improved ordering.
\end{itemize}

%%%%%%%%%%%%%%%%%% Single ordered area => Appendix? %%%%%%%%%%%%%%%%%%%%
%-------------- Add graphic to explain!-------------- 

%%%%%%%%%%%%%%%%%% Spectra %%%%%%%%%%%%%%%%%%%%
\paragraph{Spectroscopy}
\textcolor{red}{\textbf{
Some spectroscopy could be achieved that shows different typical features for different areas in the molecule. Note that the spectra were done for molecules sitting on a Ag(100) surface.
There is a clear indication, that the macrocycle of the molecule contributes to the broad peak in the dI/dV data at around \SI{1}{\V}, while the nitro groups dominate the spectra at around \SI{600}{\milli \V}. 
Look at the corresponding .pptx file for the spectra and the corresponding IGOR-files dimer/quatermer1-2 for the spectra.
}}
   \subsection{on \textit{h}-BN}
      %%%%%%%%%%%%%%%%%%%%%%%%%%%TBP on h-BN/Cu(111)
\label{section:TBP-on-hBN}
\begin{figure}[] \centering
	\subfigure[\textit{h}-BN grown on Cu(111).]{
		\includegraphics[width=0.3\textwidth]{./images/F151130-135150-40nm.png}
		\label{fig:h-BN-Cu111}
	}
	\subfigure[TBP after adsorption on \textit{h}-BN/Cu(111)]{
		\includegraphics[width=0.3\textwidth]{./images/F160617-110201-40nm.png}
		\label{fig:single-TBP-hBN-cu111}
	}
	\subfigure[TBP after adsorption on \textit{h}-BN/Cu foil.]{\includegraphics[width=0.3\textwidth]{./images/F151123-110602-40nm.png}
		\label{fig:single-TBP-hBN-cu-foil}
	}
	\caption{STM topographs of \textit{h}-BN grown on copper with subsequent molecular adsorption. \subref{fig:h-BN-Cu111} shows a \textit{h}-BN layer grown on Cu(111) by CVD. \subref{fig:single-TBP-hBN-cu111} shows the sample after evaporating TBP molecules at RT. \subref{fig:single-TBP-hBN-cu-foil} shows empty \textit{h}-BN islands grown on the polycrystalline copper foil and molecules on a copper terrace. 
		Scan parameters: \subref{fig:h-BN-Cu111} $U_b=\SI{2.273}{\volt}, I_t=\SI{0.048}{\nano \ampere}$, color scale \SIrange{0}{100}{\pico \meter}. \subref{fig:single-TBP-hBN-cu111} $U_b=\SI{1.074}{\volt}, I_t=\SI{0.033}{\nano \ampere}$, color scale \SIrange{0}{300}{\pico \meter}. \subref{fig:single-TBP-hBN-cu-foil} $U_b=\SI{2.585}{\volt}, I_t=\SI{0.032}{\nano \ampere}$, color scale \SIrange{0}{1500}{\pico \meter}.  All images are \SI{40}{\nano \meter} wide.}
	\label{TBP-on-hBN}
\end{figure}

\paragraph{\textit{h}-BN grown on Cu(111)}
\textcolor{red}{\textbf{
Further experiments have can done to investigate the behavior of TBP on \textit{h}-BN. When adsorbed on \textit{h}-BN/Cu(111), molecules show a high mobility that makes the molecules move away from the \textit{h}-BN islands. Some molecules could be resolved at defects or close to the perimeter of the \textit{h}-BN islands. This is in line with other observations for adsorbates
}}

%single-TPB-on-h-bn-cu-foil
\paragraph{\textit{h}-BN grown on Cu-foil}
\textcolor{red}{\textbf{
Molecules adsorb on the BN surface and STM imaging is hard due to molecules that can be moved on the rather 'slippy' surface of the insulating BN. Nevertheless some agglomerations of the molecules leave free BN spots where no molecules are. As the preparation of the BN should result in a closed BN layer on top of the Cu-foil no movement of molecules to free Cu areas should be observed, making these free regions BN regions.
Why the molecules are not distributed homogenously on the BN remains topic to speculation.
Spectroscopy has been tried intensively but without reproduceable results.
Unlike the adsorption on Ag(100) and Cu(111) no formation of di- and quatermers has been observed.
% See experiments in June '16
}}
%\printbibliography
	\subsection{Conclusion}
      \textcolor{red}{\textbf{
		The driving force for orienting the whole molecule on the surface remains speculative. On Ag(100), neither an orientation of the molecules main axis with respect to the substrate, nor a orientation of butyl-groups along the dense packed substrate rows can be seen - which again favors Cu-substrate interactions as dominant role.
When the copper is exchanged with silver to act as substrate, TBP behaves quite different. Although the distribution is homogeneous on the surface, the interaction between molecules look different. While on copper the most abundant binding motif is the head-to-head dimer, this motif does not appear on silver as often as on copper. Two other motifs emerge on silver.
The interaction between the butyl-phenyl groups is considered to be van der Waals like \cite{iacovita_controlling_2012}, stabilizing the conglomerate.
}}

  \section{Double leg functionalization}
   \subsection{on Cu(111)}
      % tbp-double on Cu(111)
\begin{wrapfigure}{r}{5cm}\centering
	\includegraphics[angle=90, width=5cm]{./images/molecules/max-zoom/TBP-trans-600}
	\caption{}
\end{wrapfigure}

When depositing trans-TBP on Cu(111) at room temperature no long range ordering can be achieved. The molecules arrange rather arbitrarily as can be seen in  \autoref{fig:two-leg-trans-cu111-rt}.

\begin{figure}[h]
 \centering
 \subfigure[]{
 \includegraphics[width=0.3\textwidth]{./images/F160425-172349-40nm}
 %IMAGE SCANNED COARSE!!
 \label{fig:two-leg-trans-cu111-rt}
 }
 \subfigure[New preparation adsorbed at \SI{70}{\celsius}]{
 \includegraphics[width=0.3\textwidth]{./images/F160427-121720-40nm}
 \label{fig:two-leg-trans-cu111-70c} 
 %IMAGE SCANNED COARSE!!
 }
 \subfigure[... and heated for \SI{10}{\minute} to \SI{170}{\celsius}]{
 \includegraphics[width=0.3\textwidth]{./images/F160427-142006-40nm}
 \label{fig:two-leg-trans-cu111-170c}
  %IMAGE SCANNED COARSE!!
 }
\caption{Molecules adsorbed on Cu(111) at RT and subsequently annealed to different temperatures. \subref{fig:two-leg-trans-cu111-rt} Adsorption at room temperature did not show extended long range order. \subref{fig:two-leg-trans-cu111-70c}  Adsorption at \SI{70}{\celsius} and \subref{fig:two-leg-trans-cu111-170c} annealing to \SI{170}{\celsius} for \SI{10}{\minute} improves the chain length slightly. All images are \SI{40}{\nano \meter} wide. Scan parameters: \subref{fig:two-leg-trans-cu111-rt} $U_b=\SI{1.2}{\volt}, I_t=\SI{0.041}{\nano \ampere}$, \subref{fig:two-leg-trans-cu111-70c} $U_b=\SI{0.5}{\volt}, I_t=\SI{0.038}{\nano \ampere}$, \subref{fig:two-leg-trans-cu111-170c} $U_b=\SI{0.522}{\volt}, I_t=\SI{0.021}{\nano \ampere}$}
\label{fig:two-leg-trans-cu111}
%ALL OF THE IMAGES SCANNED COARSE!!
\end{figure}

The molecules tend to connect in a defined angle to its next neighbor, forming different binding motifs. These are predominantly different kind of chain formation (see figure \ref{fig:two-leg-trans-cu111-motifs}).
\begin{itemize}
 \item The molecules are ordered such that they form a straight chain (\autoref{trans-nitro-on-cu111-70-straight-chain}).
 \item The molecules arrange in chains, but each molecule has an offset of about a half of its width to the next neighbor or the molecules attach in chains, but show a kink. \autoref{trans-nitro-on-cu111-70-shifted-chain}
\end{itemize}

\begin{figure}[h]
 \centering
 \subfigure[Straight chain]{
 \includegraphics[width=0.45\textwidth]{./images/F160427-154618-R}
\label{trans-nitro-on-cu111-70-straight-chain}} \qquad
 \subfigure[Shited offset chain, interrupted by a kink]{
 \includegraphics[width=0.45\textwidth]{./images/trans-nitro-on-cu111-120.png}
\label{trans-nitro-on-cu111-70-shifted-chain}}
\caption{All motifs exist at every temperature, although the chain length increases with temperature. It also looks like the chains are getting more offset- and kinked-like chains than at lower temperatures.}
\label{fig:two-leg-trans-cu111-motifs}
\end{figure}

\begin{figure}[h]
	\centering
	\begin{minipage}{0.45\textwidth}
		\subfigure[]{
			\includegraphics[width=\textwidth]{./images/F160427-154618-R-model}
			\label{trans-nitro-on-cu111-70-straight-chain-II}
		}
	\end{minipage}
	\begin{minipage}{0.45\textwidth}
		\subfigure[]{
			\includegraphics[width=0.45\textwidth]{./images/F160427-154618-R-gas-phase-top}
			\includegraphics[width=0.45\textwidth]{./images/F160427-154618-R-cu111-top}
			\label{trans-nitro-on-cu111-70-shifted-chain-top-views}
		}
		\subfigure[]{
			\includegraphics[width=0.45\textwidth]{./images/F160427-154618-R-gas-phase-side}
			\includegraphics[width=0.45\textwidth]{./images/F160427-154618-R-cu111-side}
			\label{trans-nitro-on-cu111-70-shifted-chain-side-views}
		}
	\end{minipage}
	\caption{Straight chain binding motif on Cu(111). \subref{trans-nitro-on-cu111-70-straight-chain-II} shows an STM image together with the dense packed row indication of the substrate (white lines). Colored bars indicate the rotation of the di-tert-butyl-groups. Arrows point at places where ad-atoms are considred.
		\subref{trans-nitro-on-cu111-70-shifted-chain-top-views} Top views (\SI{6}{\nano \meter} wide) showing the molecules geometry in gas-phase (left) and after adsorption and assembly (right). Although the exact adsorption site is not known, it is considered to by on a bridge site as for 2H-P/Cu(111).
		\subref{trans-nitro-on-cu111-70-shifted-chain-side-views} Side views of above shown configurations.
	}
	\label{fig:two-leg-trans-cu111-motifs-1}
\end{figure}

During modeling \autoref{fig:two-leg-trans-cu111-motifs-1} several points became clear. 
\begin{itemize}
	\item First consider the even apparent height of the di-tert-butyl groups. It indicates that both groups in a legs have comparable heights and it is likely that the phenyl ring bearing these groups is rotated for an even alignment of the tert-butyl groups with regard to the substrate level.
	\item Orientation of di-tert-butyl phenyl groups is the same within a single molecule but alternates (by $\approx \SI{10}{\degree}$) in neighboring molecules in a chain. This is indicated by blue and green lines in \autoref{trans-nitro-on-cu111-70-straight-chain}, each representing a common orientation.
	\item Second the minor contrast variations in the central porphine core change as the orientation of the di-tert-butyl-groups. Free base porphine core is likely to adsorb with its axis  - formed by opposing nitrogens in the core - aligned parallel to the dense packed crystal direction\cite{rojas_surface_2012}. In the present case, the molecule is lifted from the substrate by the bulky di-tert-butyl groups. Hence the porphine core interaction with the crystal substrate is considerable lower than in the 2H-P case. Still, every second molecule has the same orientation, while neighboring molecules are rotated by \SI{30}{\degree}.
	\item The gap between di-tert-butyl-phenyl groups of neighboring molecules is larger on one side of the chain than on the other and shows a larger apparent height (white arrows in \autoref{trans-nitro-on-cu111-70-straight-chain}). Although identification of surface ad-atoms is not straightforward with an STM, they are believed to originate from the copper surface.
\end{itemize} 
The best fitting model consists of molecules with a center-center distance of \SI{1.9 \pm 0.1}{\nano \meter}

Having a closer look to the nitro groups, one recognizes a close proximity of these to each other. Also note the light protrusions in between two adjacent molecules' butyl groups (adatom?). If the legs are rotated by just \SI{15}{\degree}, the nitro groups would point to these protrusions. This rotation costs not much energy and is about \SI{25}{\kilo\J/per\mol} \textcolor{red}{\textbf{(( please cite something, value is for rotated phenyl ring at a porphine core I guess ))}}. Considering these protrusions as Cu-ad atoms (already occurred in chapter \ref{chapter:TPCN-adatoms} as protrusions in between TPCN chains which may change their position in discrete position in the molecule.) This Cu-ad atom may direct the binding of the nitro groups towards it, making them bend outwards. The position of the cooper atom itself may rely on its registry to the substrate - preferring a threefold coordination site as known for copper  \textcolor{red}{\textbf{(( citation ))}}.

The second motif is a chain motif, too. Orientation of molecular axis and dense packed substrate atom rows are the same and again the di-tert-butyl groups orient along them. The difference is a lateral offset between the molecules to shift each of them by half a molecules width. The center-center distances are \SI{1.9 \pm 0.1}{\nano \meter}. It is harder to quantify a possible orientation of the nitro-phenyl groups, since as well straight as well as bended configurations match the assembly. In this binding motif, stable connections between molecules are most likely due to nitro-phenyl groups pointing to di-tert-butyl groups and therefor stabilizing the assembly.


   \subsection{on Ag(100)}
      % TBP-double-Ag100
\paragraph{Unit cell}
When adsorbed on a square (100) silver surface, the molecules interestingly arrange in a trihexagonal tiling (see figure \ref{fig:two-leg-trans-ag100-motif}). The molecules at the perimeter of this island is nicely distinguishable and continuing their regular pattern to the center of the island results in an accurate description of the assembly. The unit cell is determined to be $\underline{\qquad \qquad}$ and the hexagonal unit cell is shown in \autoref{fig:two-leg-trans-ag100-unit-cell}, bearing three molecules.\footnote{Similar open porous network can be created, e.g. cyano functionalized triarylamines on Au(111) \cite{gottardi_cyano-functionalized_2014}.}

\paragraph{Molecular orientation}
The molecules are arranged so that each molecule has one of its di-tert-butyl-groups in one hexagonal pores and the other in the neighboring one. Each pore is made up of six molecules arranged on a hexagon with $\underline{\qquad \qquad}$ long edges. Each vertex is occupied by a single molecule, neighboring molecules on the hexagon are rotated by \SI{60}{\degree}. The pores are created by free space where the di-tert-butyl-groups point towards each other. The nitro-phenyl groups point towards the intermediate space where smaller triangular openings are formed. At their edges the nitro-phenyl groups connect to the neighboring di-tert-butyl groups.

Considering a former orientation calibration on Ag(100) where the direction of the dense packed crystal direction was determined, the orientation with regard to the substrate is given as white lines in \autoref{fig:two-leg-trans-ag100-unit-cell}: The long and short axis of the unit cell (marked as green cross in \subref{fig:two-leg-trans-ag100-unit-cell}) is almost collinear, just differing by less than \SI{10}{\degree}. Since the calibration was done with another preparation the angle calibration may not be \SI{100}{\percent} accurate because the sample was moved in the meantime. That may result in an little angle uncertainty. Please see  \autoref{F160429-185245-R-model-2-crystal-orientation.png} in \fullref{appendix:TBP} for a detailed image.

\paragraph{Contrast within single molecule}
A closer look to the geometries in high resolution STM data gives clue to the rotation of the di-tert-butyl-groups and is visualized in \autoref{fig:two-leg-trans-ag100-single-molecule}. Focusing on the STM contrast of a single molecule, one can see that it is dominated by the di-tert-butyl-groups on both sides of the molecule. These look like small triangles in the STM with a single brighter protrusion enclosed by the footprint. The bright protrusion is never on the same side of the triangular footprint thus the di-tert-butyl-groups are believed to be rotated in two different directions - lifting opposite parts of the functional group.
\begin{figure}[]
	\centering
	\subfigure[STM topography of several islands grown next to a step edge. Areas with trihexagonal tiling as well as some domain boundaries are visible.]{
		\includegraphics[width=\textwidth]{./images/F160429-172019}
		\label{fig:two-leg-trans-ag100-overview}
	} %COARSE MODE!
	\subfigure[Hexygonal unit cell with overlaid molecular models.]{
		\includegraphics[width=0.3\textwidth]{./images/F160429-185245-R-model}
		\label{fig:two-leg-trans-ag100-unit-cell}
	} \quad %COARSE MODE!
	\subfigure[Enlarged view on the molecules rotated di-tert-butyl-group and highest elements enclosed by brighter circles..]{
		\includegraphics[width=0.3\textwidth]{./images/F160429-185245-R-single-molecule}
		\label{fig:two-leg-trans-ag100-single-molecule}
	} %COARSE MODE!
	\caption{Trans-TBP adsorped on Ag(100) at room temperature. \subref{fig:two-leg-trans-ag100-overview} shows a large overview of the assembled molecules. The unit cell constituents are enlarged in \subref{fig:two-leg-trans-ag100-unit-cell} where parts of \subref{fig:two-leg-trans-ag100-overview} are shown with molecular models overlaid. \subref{fig:two-leg-trans-ag100-single-molecule} shows a single molecule crossing a horizontal plain to emphasize high lying part in the molecule that are marked with white circles and will appear brighter in STM. All images recorded with \SI{437}{\milli\volt}, \SI{0.1}{\nano\ampere}, color scale \SIrange{0}{650}{\pico\meter}
	}
	\label{fig:two-leg-trans-ag100-motif}
\end{figure}

\paragraph{Domain boundaries}
The observed domain boundaries are imaged in \autoref{fig:two-leg-trans-ag100-domain-boundary}. On both sides the regular tiling is proceeded, but both are shifted with respect to each other by $\underline{\qquad \qquad}$. This offset results in the wrong alignment of molecules from one  domain with respect to the other domain and a discontinued growth. The resulting free area at the domain boundary is occupied by molecules from one domain that bear the wrong orientation the proceed the growth of the second domain and vice versa. This can be nicely seen in 		\autoref{fig:two-leg-trans-ag100-domain-molecular-model}, where the misalignment of one domain (left) with respect to the other (right) causes two cavities to open up between the two (lower image part). While these two are unoccupied and reveal the substrate, another type of cavity can be formed directly seen on top of the two aforementioned. Here the cavity is filled with a single molecule so that both di-tert-butyl groups interlock with the open cavity. Please note that some is the assembly pores are filled, too. Here the space of the pore prohibits a complete molecule to fit in, the observed adsorbates in the pores are most likely molecular fragments like tert-butyl-groups that were incorporated by the assembly during the island growth.

\begin{figure}[]
	\centering
	\subfigure[Domain boundary]{
		\includegraphics[width=0.35\textwidth]{./images/F160429-185245-R--2}
		\label{fig:two-leg-trans-ag100-domain-overview}
	} %COARSE MODE!
	\subfigure[Model representation of domain boundary]{
		\includegraphics[width=0.35\textwidth]{./images/F160429-185245-R--2-model}
		\label{fig:two-leg-trans-ag100-domain-model}
	} %COARSE MODE!
	\subfigure[Molecular model of domain boundary, overlaid with two unit cells]{
		\includegraphics[width=0.7\textwidth]{./images/F160429-185245-R--domain-overview}
		\label{fig:two-leg-trans-ag100-domain-molecular-model}
	} \quad %COARSE MODE!
	\caption{Domain boundary of trans-TBP adsorbed on Ag(100) at RT. \subref{fig:two-leg-trans-ag100-domain-overview} shows an overview of the domain boundary together with its model representation in \subref{fig:two-leg-trans-ag100-domain-model}. The assembly close by is modeled in \subref{fig:two-leg-trans-ag100-domain-molecular-model} where parts of \subref{fig:two-leg-trans-ag100-domain-overview} are shown and molecular models overlaid. All images recorded with \SI{1.3}{\volt}, \SI{0.1}{\nano\ampere}, color scale \SIrange{0}{650}{\pico\meter}
	}
	\label{fig:two-leg-trans-ag100-domain-boundary}
\end{figure}
%\printbibliography
%%%%%%%%%%%%%%%%%%%%%%%%%%%%%%%%%%%%%%%%%
\chapter{Functionalized helicene molecules on Ag(111), Ag(100) and \textit{h}-BN supports}
% Here comes the helicene paper
\label{section:helicene}



%%\printbibliography
%%%%%%%%%%%%%%%%%%%%%%%%%%%%%%%%%%%%%%%%%%
\chapter{Summary}
      This is intentionally for all nice little things which does not certainly make it into this work: \cite{hertz_ueber_1887}

%%%%%%%%%%%%%%%%%%%%%%%%%%%%%%%%%%%%%%%%%%%%%%%%%%%%%%%%%%%%%%%%%%%%%%%%%%%%%%%%%%%%
%%%%%%%%%%%%%%%%%%%%%%%%%%%%%%%%%%%%%%%%%%%%%%%%%%%%%%%%%%%%%%%%%%%%%%%%%%%%%%%%%%%%
\backmatter{}
  \printbibliography
\chapter{Appendix}
%\addcontentsline{toc}{chapter}{Index}
%  \addcontentsline{toc}{chapter}{Appendix}
%  \cleardoublepage
\chapter*{Appendix}
%{\usekomafont{section} Appendix}
First Backmatter stuff.


%%%%%%%%%%% TBP on Ag(100) @ RT - single ordered island
\paragraph{Ordered areas}
Only a single ordered area of TBP on Ag(100) was found, but its structure could not be resolved properly due to tip issues (compare figure \ref{fig:hex-TBP-Ag100}). Its unit cell looks hexagonal with roughly \SI{1.7} {\nano \meter} period. 

\begin{figure}[h]
	\centering
	\includegraphics[width=0.5\textwidth]{./images/F151007-112800}
	\caption{TBP on Ag(100) showing some ordering}
	\label{fig:hex-TBP-Ag100}
\end{figure}

%%%%%%%%%%%%%%%%%%%%%%%%%%%%%%%%%%%%%%%%%%%%%%%%%%%%%%%%%%%%%%%%%%%%%%%%%%%%%%%%%%%%%%%%%%%%%%%%%%%
\section{Crystal facets}
  \input{./includes/chapter/backmatter/crystal-facet}
    %% Orbital calculations
\section{Pyrene}
In this section extended hückel theory calculations of functionalized pyrene molecules are shown. After structural relaxation via the AM1 semi-empirical method in the Hyperchem software, the result is exported to an IGOR script for presentation. Single molecular orbitals as well as their sum can be shown. For all molecules some basic information is given in a box. The number of atoms, electrons and calculated orbitals is given together with a ball and stick model of the corresponding molecule. Calculated HOMO/LUMO states are shown to the right with their energies noted below.
The left side of the pages shows occupied states (second column) and their sum (first column). The right side shows the unoccupied in the same manner.

The first two pages will show cis-pyrene as calculated in gas phase together with a model showing the rotation of pyridil legs and the influence on molecular orbitals. Trans- \& tetra-pyrene molecules are shown afterwards.

Due to the same molecular center, all the species show distinct symmetry at their pyrene center. 

%\begin{figure}[]\centering
%	\includegraphics[width=0.7\textwidth]{./images/paper/pyrene/figure-S1}
%	\caption{Top and side views of optimized structures for trans-(left) and tetra-pyrene (right) at the B3LYP/6-31G(d,p) level of theory.}
%	\label{fig:pyene-S1}
%\end{figure}


%\begin{figure}[]\centering
%	\includegraphics[width=0.7\textwidth]{./images/paper/pyrene/figure-S2}
%	\caption{Frontier molecular orbitals (isovalue contours ± 0.02 a.u.) for 1 (left) and 2 (right) from B3LYP/6-31G(d,p) calculations.}
%	\label{fig:pyene-S2}
%\end{figure}

%\begin{figure}[]\centering
%	\includegraphics[width=0.7\textwidth]{./images/paper/pyrene/figure-S3}
%	\caption{Dominant orbital contributions to the first and second excitations in trans-pyrene (a) and in tetra-pyrene (b), obtained from TD-DFT calculations (CAM-B3LYP/6-31G**, toluene CPCM solvation).}
%	\label{fig:pyene-S3}
%\end{figure}

\begin{figure}[]\centering
	\includegraphics[width=0.7\textwidth]{./images/paper/pyrene/figure-S4}
	\caption{Calculated (EHT) HOMO (left) and LUMO (right) states of trans-pyrene. Assignment of HOMO and LUMO orbitals through symmetry observations in STM topography images that resolve the molecular orbitals. Compared to the HOMO, where between two legs only one side features a double lobe, the LUMO has double lobes in between every leg.}
	\label{fig:pyene-S4}
\end{figure}

\begin{figure}[]\centering
	\subfigure{
	\includegraphics[width=0.35\textwidth]{./images/paper/pyrene/figure-S5a}
	\label{fig:pyrene-S5a}
	}
	\subfigure{
	\includegraphics[width=0.35\textwidth]{./images/paper/pyrene/figure-S5b}
	\label{fig:pyrene-S5b}
	}
	\caption{a) EHT calculated HOMO and b) LUMO states of cis-pyrene.}
	\label{fig:pyene-S5}
\end{figure}

\autoref{fig:pyene-S6} shows the emergence of LUMO states for trans-pyrene. The STM topographies show the same region with subsequently increased bias voltage. Starting close to the fermi level in the molecules’ band gap the contrast is determined by the shape of the molecule (upper left). Increasing the bias voltage (for upper left to lower right) reveals two things. First the contrast pattern changes at energies close to the proposed LUMO \SIrange{1.4}{1.6}{\volt} and LUMO + 1 \SIrange{2.2}{2.4}{\volt} energies, while the contrast between \SI{1.8}{\volt} and \SI{2.1}{\volt} remains constant. Second this change in contrast is not homogeneous across the surface but originates from the pores´ center.

\begin{figure}[]\centering
	\includegraphics[width=0.7\textwidth]{./images/paper/pyrene/figure-S6}
	\caption{STM image series depicting LUMO/LUMO+1 states with positive bias voltages (\SIrange{1.1}{2.5}{\volt}). Starts @ \SI{1.1}{\volt} and increases in steps of \SI{0.1}{\volt} for upper left to lower right, Image size is \SI{11}{\nano \meter} $\times$ \SI{11}{\nano \meter}.}
	\label{fig:pyene-S6}
\end{figure}

\begin{figure}[]\centering
	\includegraphics[width=0.7\textwidth]{./images/paper/pyrene/figure-S7}
	\caption{STM image series depicting HOMO-1 states with negative bias voltages \SIrange{-1.8}{-2.7}{\volt}. Starts @ \SI{-1.8}{\volt} and decreases in steps of \SI{0.1}{\volt} for upper left to lower right. Image size is \SI{5.5}{\nano \meter} $\times$ \SI{5.5}{\nano \meter}.
	}
	\label{fig:pyene-S7}
\end{figure}

\begin{figure}[]\centering
	\includegraphics[width=0.7\textwidth]{./images/paper/pyrene/figure-S8}
	\caption{Calculated (EHT) HOMO (top) and LUMO (bottom) states (in red/blue) superimposed on a model of trans-pyrene. Imaging parameter: \SI{-2.4}{\volt} (HOMO) \& 2.1 V (LUMO). Recorded with \SI{0.2}{\nano \ampere}, Image width: \SI{5.5}{\nano \meter}.
	}
	\label{fig:pyene-S8}
\end{figure}

\begin{figure}[]\centering
	\includegraphics[width=0.7\textwidth]{./images/paper/pyrene/figure-S9}
	\caption{Homochiral mirror domains of a self-assembled sub-ML of trans-pyrene on h-BN/Cu(111). a) STM image with overlaid molecular models and unit cells (colored in orange). b) Enlarged view on the molecular assembly in a). Imaging parameters: \SI{14}{\nano \meter} $\times$ \SI{25}{\nano \meter}, \SI{1.0}{\volt}, \SI{0.1}{\nano \ampere}.
	}
	\label{fig:pyene-S9}
\end{figure}

Some preparations were done to show the functionality of the trans-pyrene pores as host system for other molecules. First cis-pyrene was dosed on the surface at R, afterwards annealed to \SI{100}{\celsius}, cooled to RT followed by dosing trans-pyrene molecules. Two different, coexisting motifs occur which are shown in \autoref{fig:pyrene-S10}. First there is a dense packed phase (\autoref{fig:pyrene-S10a}, \autoref{fig:pyrene-S10c}). This mixed preparation does show a uniform growth. Alternating rows of either trans-pyrene or pairs of cis-pyrene molecules assemble in islands. The resulting triclinic unit cell (\SI{1.94}{\nano \meter} $\times$ \SI{1.97}{\nano \meter} \SI{75}{\degree}) incorporates 3 molecules – one trans-pyrene and two cis-pyrenes. The second phase is dominated by the kagome motif already observed in the homo-molecular preparations but with their larger pores partially filled with cis-pyrene species (\autoref{fig:pyrene-S10b}, \autoref{fig:pyrene-S10d}). The unit cell remains the very same, but contains an additional cis-pyrene molecule. Not all guests (cis-pyrene) molecules are oriented the same way, only discrete orientations are allowed (and three of six different ones can already be seen in \autoref{fig:pyrene-S10b}). This means that although cis-pyrene molecules were evaporated first their assembly becomes unstable at temperatures of about \SI{100}{\celsius}. This allows them to participate in the nucleation process after the trans-pyrene molecules are dosed at RT. Their assembly is still in progress while cooling down from RT to \SI{7}{\kelvin}. A fully occupied kagome cell bears 3 trans-pyrene molecules and a single cis-pyrene molecule.

\begin{figure}[]\centering	
	\subfigure[]{
	\includegraphics[width=0.35\textwidth]{./images/paper/pyrene/figure-S10a}
		\label{fig:pyrene-S10a}
	}
	\subfigure[]{
	\includegraphics[width=0.35\textwidth]{./images/paper/pyrene/figure-S10b}
		\label{fig:pyrene-S10b}
	}
	\subfigure[]{
	\includegraphics[width=0.35\textwidth]{./images/paper/pyrene/figure-S10c}
		\label{fig:pyrene-S10c}
	}
	\subfigure[]{
	\includegraphics[width=0.35\textwidth]{./images/paper/pyrene/figure-S10d}
		\label{fig:pyrene-S10d}
	}	
	\caption{Cis-pyrene/trans-pyrene mixed preparation on \textit{h}-BN/Cu (111). STM image of a self-assembled sub-ML with overlaid molecular models (\subref{fig:pyrene-S10a}, \subref{fig:pyrene-S10c}). Enlarged views on the molecular unit cells (\subref{fig:pyrene-S10b}, \subref{fig:pyrene-S10d}). \subref{fig:pyrene-S10a}, \subref{fig:pyrene-S10b}) shows a dense packed mixed motif while \subref{fig:pyrene-S10c} and \subref{fig:pyrene-S10d} show cis-pyrene molecules (guest) in a trans-pyrene kagome (host) network. Imaging parameters: \SI{1.0}{\volt}, \SI{0.06}{\nano \ampere}.
		%, \SI{}{\nano \meter} $\times$ \SI{}{\nano \meter}.
	}
	\label{fig:pyene-S10}
\end{figure}

%\begin{figure}[]\centering
%	\includegraphics[width=0.7\textwidth]{./images/paper/pyrene/figure-S11}
%	\caption{Absorption and emission spectra for trans- and tetra-pyrene.
%	}
%	\label{fig:pyene-S11}
%\end{figure}

%\begin{figure}[]\centering
%	\includegraphics[width=0.7\textwidth]{./images/paper/pyrene/figure-S12}
%	\caption{Mapped Electrostatic Potential of trans-pyrene (top) and tetra-pyrene (bottom).
%	}
%	\label{fig:pyene-S12}
%\end{figure}

%\begin{figure}[]\centering
%	\includegraphics[width=0.7\textwidth]{./images/paper/pyrene/figure-S13}
%	\caption{Model pyridylethynyl pyrene compound used for calculating the energy profile for the rotation around the triple bond, at the B3LYP/6-31G(d,p) level of theory.
%	}
%	\label{fig:pyene-S13}
%\end{figure}

%%%%%%%%%%%%%%%%%%%%%%%%%%%%%%%%%%%%%%%
% Change to larger page layout
%%%%%%%%%%%%%%%%%%%%%%%%%%%%%%%%%%%%%%%
\newgeometry{top=2cm,bottom=2cm}%
%
  
	%%%%%%%%%%%%%%%%%% Orbitals for cis-pyrene %%%%%%%%%%%%%%%%%%%%%%%%%%%%%%
	\graphicspath{{./images/molecules/orbitals/cis-pyrene/}}
\newgeometry{top=2cm,bottom=2cm}
	\begin{figure}[]
		\begin{minipage}{0.2\textwidth} \centering
			\mybox{\begin{tabular}{lr}
				\# Atoms &  \\
				\# $e^-$ & 146 \\
				\# Orbitals & 85 \\
				$E_{Gap}$ &  \SI{}{\electronvolt}\\
			\end{tabular}
			}
		\end{minipage}
		\hfill
		\begin{minipage}{0.4\textwidth} \centering 
			\begin{tabular}{c|c}
				\multicolumn{2}{c}{\textbf{Cis-Pyrene}} \\
				HOMO & LUMO \\
				\includegraphics[width=0.45\textwidth]{homo} &
				\includegraphics[width=0.45\textwidth]{lumo} \\
				\SI{-11.601}{\electronvolt} & \SI{-10.134 }{\electronvolt} \\
			\end{tabular}
		\end{minipage}
		\hfill
		%
		\begin{minipage}{0.2\textwidth} \centering
			Insert fancy Energy graph here...	
		\end{minipage}
	\end{figure}
	%%%%%%%%%%%%%%%%%%%%%%%%%%%%%%%%%%%%%%%%%%%%%%%%%%%%%%%%%%%%%%%%%%%%%%%%%%%%%%%%%%
	%%%%%%%%%%%%%%%%%%%%%%%%%%%%%%%%%%%%%%%%%%%%%%%%%%%%%%%%%%%%%%%%%%%%%%%%%%%%%%%%%%
	\begin{figure}[]
		\centering
		\subfigure[]{
			\includegraphics[width=0.2\textwidth]{int-homo-1}
			\label{fig:}
		}
		\subfigure[HOMO - 1]{
			\includegraphics[width=0.2\textwidth]{homo-1}
			\label{fig:homo-1}	
		}
		\subfigure[LUMO + 1]{
			\includegraphics[width=0.2\textwidth]{lumo-1}
			\label{fig:lumo+1}	
		}
		\subfigure[]{
			\includegraphics[width=0.2\textwidth]{int-lumo-1}
			\label{fig:}	
		}
		%%%%%%%%%%%%%%%%%%%%%%%%%%%%%%%%%%%%%%%%%
		\subfigure[]{
			\includegraphics[width=0.2\textwidth]{int-homo-2}
			\label{fig:}
		}
		\subfigure[HOMO - 2]{
			\includegraphics[width=0.2\textwidth]{homo-2}
			\label{fig:homo-2}	
		}
		\subfigure[LUMO + 2]{
			\includegraphics[width=0.2\textwidth]{lumo-2}
			\label{fig:lumo+2}	
		}
		\subfigure[]{
			\includegraphics[width=0.2\textwidth]{int-lumo-2}
			\label{fig:}	
		}
		%%%%%%%%%%%%%%%%%%%%%%%%%%%%%%%%%%%%%%%%%
		\subfigure[]{
			\includegraphics[width=0.2\textwidth]{int-homo-3}
			\label{fig:}
		}
		\subfigure[HOMO - 3]{
			\includegraphics[width=0.2\textwidth]{homo-3}
			\label{fig:homo-3}	
		}
		\subfigure[LUMO + 3]{
			\includegraphics[width=0.2\textwidth]{lumo-3}
			\label{fig:lumo+3}	
		}
		\subfigure[]{
			\includegraphics[width=0.2\textwidth]{int-lumo-3}
			\label{fig:}	
		}
		%%%%%%%%%%%%%%%%%%%%%%%%%%%%%%%%%%%%%%%%%
		\subfigure[]{
			\includegraphics[width=0.2\textwidth]{int-homo-4}
			\label{fig:}
		}
		\subfigure[HOMO - 4]{
			\includegraphics[width=0.2\textwidth]{homo-4}
			\label{fig:homo-4}	
		}
		\subfigure[LUMO + 4]{
			\includegraphics[width=0.2\textwidth]{lumo-4}
			\label{fig:lumo+4}	
		}
		\subfigure[]{
			\includegraphics[width=0.2\textwidth]{int-lumo-4}
			\label{fig:}	
		}
		%%%%%%%%%%%%%%%%%%%%%%%%%%%%%%%%%%%%%%%%%
		\subfigure[]{
			\includegraphics[width=0.2\textwidth]{int-homo-5}
			\label{fig:}
		}
		\subfigure[HOMO - 5]{
			\includegraphics[width=0.2\textwidth]{homo-5}
			\label{fig:homo-5}	
		}
		\subfigure[LUMO + 5]{
			\includegraphics[width=0.2\textwidth]{lumo-5}
			\label{fig:lumo+5}	
		}
		\subfigure[]{
			\includegraphics[width=0.2\textwidth]{int-lumo-5}
			\label{fig:}	
		}
		\caption{EHT calculated molecular orbitals. HOMO and LUMO states together with five neighboring states (shown in the same column). Inner two columns show HOMO \& LUMO states. Outer columns show integration of states as STM image estimation if all states to $E_F$ were contributing equally. Images are \SI{2.5}{\nano \meter} wide}
		\label{fig:}
	\end{figure}
	\vfill
\restoregeometry
  \input{./includes/chapter/backmatter/EHT-cis-pyrene-60}
  \input{./includes/chapter/backmatter/EHT-trans-pyrene}
  \input{./includes/chapter/backmatter/EHT-tetra-pyrene}
\restoregeometry
%%%%%%%%%%%%%%%%%%%%%%%%%%%%%%%%%%%%%%%
     %% TBP-Appendix
 \section{TBP}
 % Appendix for TBP chapter
\label{appendix:TBP}
%%%%%%%%%%%%%%%%%%%%%%%%%%%%%%%%%%%%%%%%%%%%%%%%%%%%%%%%%%%%%%%%%%%%%%%%%%%%%%%%%%
%%%%%%%%%%% TBP on Ag(100) @ RT - single ordered island
\paragraph{Ordered areas}
Only a single ordered area of TBP on Ag(100) was found, but its structure could not be resolved properly due to tip issues (compare figure \ref{fig:hex-TBP-Ag100}). Its unit cell looks hexagonal with roughly \SI{1.7} {\nano \meter} period. 

\begin{figure}[h]
	\centering
	\includegraphics[width=0.5\textwidth]{./images/F151007-112800}
	\caption{TBP on Ag(100) showing some ordering}
	\label{fig:hex-TBP-Ag100}
\end{figure}
%%%%%%%%%%%%%%%%%%%%%%%%%%%%%%%%%%%%%%%%%%%%%%%%%%%%%%%%%%%%%%%%%%%%%%%%%%%%%%%%%%%%%%%%%%%%%%%%%%%
%%%%%%%%%%%%%%%%%%%%%%%%%%%%%%%%%%%%%%%%%%%%%%%%%%%%%%%%%%%%%%%%%%%%%%%%%%%%%%%%%%
\paragraph{TBP on Ag(100) - Symmetry relations}
	\begin{figure}[]
		\centering
		\includegraphics[width=\textwidth]{./images/F160429-185245-R-model-2-crystal-orientation.png}
		\caption{Symmetry relations between TBP molecule and Ag(100) crystal substrate. The same molecular model is highlighted in two different ways, emphasizing the two molecular axis (red/blue). Since the assembly is made up of three different orientations, the three rotated axis sets are shown. The assemblies derived unit cell is shown as shaded background with short and long symmetry axis highlighted in green. The crystal orientation from another preparation on the same single crystal is shown in grey.}
		\label{F160429-185245-R-model-2-crystal-orientation.png}			
	\end{figure}
	\vfill
\restoregeometry
% \paragraph{How to determine molecules' distance}
To determine the distance between molecules, one has to carefully choose the points of interest. As a problem of STM imaging the contour of the molecules sometimes appears as more or less fuzzy shape. There is no sharp edge that one could take as start or end point of the profile. Therefore the center of the molecule is often used as reference point to measure the distance between two molecules (compare fig. \ref{fig:distance-molecules}). As the molecule has a square footprint, one can use the center in one direction (along profile 2/3) to determine the center in the other direction (profile 1). As one can see the three profiles match leading to a consistent center of the dimer. This is also shown as depression in profile 1. 

\begin{figure}[]
	\centering
	\subfigure[Molecule with chosen profiles (1-3) indicated as white lines.]{
		\includegraphics[width=0.45\textwidth]{./images/F150612-163956-dimer-loose.jpg}
	} \quad
	\subfigure[Profiles 1-3 indicated in a).  Local minima in profile 2/3 indicate central positions in profile 1.]{
		\includegraphics[width=0.45\textwidth]{./images/F150612-163956-profile-dimer-loose.jpg}
	}
	\caption{Sketch of how to determine the distance between two molecules. As the molecule is square (with the exception of one direction, one can determine the center of the molecule by comparing two \SI{90}{\deg} rotated profiles. Profile 1 goes through the symmetry axis, while profile 2 and 3 intersect profile 1 at the center. As the profile 2 and 3 look the same when starting at the buthyl groups, one can use the depression in profile 2 and 3 to determine the center of the molecule in profile 1.}
	\label{fig:distance-molecules}
\end{figure}
 %%%%%%%%%%%%%%%%%%%%
%% HBBNC-Appendix
\section{HBBNC}
\begin{figure}[] \centering
	\includegraphics[width=0.7\textwidth]{./images/hbbnc-maps}
	\caption{Choosing the spectroscopy energy to match one of the spectral maxima of the line spectra reveals the spatial distribution of electronic states as shown in the STS map. While at 650 meV only the core contributes to the DOS, energies of 1200 meV and 1600 meV are located on the leg and edge positions respectively.}
	\label{}
\end{figure}

\begin{figure}[] \centering
	\includegraphics[width=0.7\textwidth]{./images/hbbnc-maps2}
	\caption{Comparison of two hexamers (chiral twins). While in STM the most apparent change is the orientation of the molecules within the hexamer and the resulting change in the protrusion between them, dI/dV maps (600 meV) clearly highlight the turn direction within the hexamer.}
	\label{}
\end{figure}

\begin{figure}[] \centering
	\includegraphics[width=0.7\textwidth]{./images/hbbnc-ag-111-leg-flip}
	\caption{Conformational change of a monomer after several scans with the AFM. First the molecule starts with an orientation of the legs in 1: clockwise, 2: counterclockwise, 3: clockwise. After several scans in close proximity (how far?!) legs at positions 2 and 3 flip around and change their orientation to 1: clockwise, 2: clockwise, 3: counterclockwise. This results in a changed adsorption geometry. First the upper right edge is lifted from the substrate, while after the leg rearrangement the lower right edge is lifted from the surface. This is caused by the lower lying dimethyl groups lifting the phenyl ring and the molecules edge.}
	\label{}
\end{figure}

\begin{figure}[] \centering
	\includegraphics[width=0.7\textwidth]{./images/hbbnc-ag-111-rt-med-coverage-spacer-mol}
	\caption{Rows are separated by two sets of spacer molecules (orange/yellow). 
		\textcolor{red}{\textbf{What is their position in the unit cell? Rotation? Separation to all the other molecules?}}
	}
	\label{}
\end{figure}

 
%%%%%%%%%%%%%%%%%%%%%%%%%%%%%%%%%%%%%%%
 %%%%%%%%%%%%%%%%%%%%
 %% HBBNC-Appendix
 \section{Helicene}
 \label{appendix:helicene}
Some experiments were done to test the thermal stability of the self-assembly. Therefor the molecules are deposited at room temperature in subsequently annealed to higher temperatures and investigated in LT-STM (\autoref{fig:hel-fig-S1}). Moderate annealing to \SI{100}{\celsius} leads to almost no changes, the chains maintain their shapes. At intermediate temperatures a ring closure reaction between the first and last carbon ring is induced, resulting in a new species. Its appearance is flat in LT-STM. As the temperature increases, they start to disintegrate into smaller fragments with no dominant binding motif.

\begin{figure} \centering
	\includegraphics[width=0.7\textwidth]{./images/paper/helicene/fig-S1}
	\caption{After adsorption at RT on Ag(111) the same sample was subsequently heated to 100/150/170 \SI{}{\celsius} (a-c) and investigated in LT-STM. The higher the annealing temperature gets, the shorter the chains become. At intermediate temperature ring closure reactions form flat compounds. At longer annealing times at elevated temperature (\SI{175}{\celsius}, \SI{30}{\minute}, d) the molecules start to form denser but more unordered configurations while the typical chain length of the assembly further decreases.}
	\label{fig:hel-fig-S1}
\end{figure}

A closer look revealed that some of the molecules start to change their appearance after annealing.  The height of single molecules reduces and the distinct bright spot in the molecule vanishes (\autoref{fig:hel-fig-S1}a)). Modelling a dcdb-[5]H with Hyperchem (AM1) where first and last carbon rings are fused together matches this shape very well, pointing to an temperature induced ring closure reaction (\autoref{fig:hel-fig-S2}b)), which suppresses chirality of the new formed species. These often but not attach to the initial molecular assembly or agglomerate in unordered configurations. A cyclodehydogenation is observed for double 5[H] on Au(111) at \SI{380}{\celsius} \cite{Wang_Heteroatom-doped_2017} and dibenzo[i,o]heptahelicene on Ag(111) at \SIrange{247}{397}{\celsius} \cite{Stetsovych_helical_2016}. Heptahelicene/Cu(111) decomposes above \SI{227}{\celsius} \cite{Ernst_two-dimensional_2001}. Investigation of [7]H on Ni(111) and Ni(100) revealed dehydrogenation temperatures not below \SI{227}{\celsius} and \SI{127}{\celsius} respectively.\cite{Ernst_Adsorption_2003}

\begin{figure} \centering
	\includegraphics[width=0.7\textwidth]{./images/paper/helicene/fig-S2}
	\caption{Upon heating to the Ag(111) to \SI{170}{\celsius} for \SI{10}{\minute} some molecules start to change their typical appearance (red circle in a)). They lose their bright feature and become flat. A molecule’s model representation with its first and last carbon rings connected is shown in b) closely resembling the footprint of the new species.}
	\label{fig:hel-fig-S2}
\end{figure}

Chains formed at RT on Ag(111) bear significantly more molecules than those formed on Ag(100) and are almost exclusively made of two strands - thus are 4 molecules wide. This is shown as histogram representation in \autoref{fig:hel-fig-S3}. Only some chains are three strands wide.

\begin{figure} \centering
	\includegraphics[width=0.7\textwidth]{./images/paper/helicene/fig-S3}
	\caption{(a) Topography of a molecular chain along the Ag(100) high symmetry direction recorded at \SI{400}{\milli \volt}. (b-h) dI/dV  maps ranging from \SIrange{600}{300}{\milli \volt}. The periodic height modulation visible in the topography can be recognized in the dI/dV maps. The width of the features decreases with decreasing bias voltage.}
	\label{fig:hel-fig-S3}
\end{figure} 
%%%%%%%%%%%%%%%%%%%%%%%%%%%%%%%%%%%%%%%
%\begin{landscape}
%	\section{Python interface for a Maxigauge-TPG256A}
%	\pythonexternal{./includes/chapter/backmatter/Maxigauge-TPG256A.py} %see commands for this 
%\end{landscape}
%%%%%%%%%%%%%%%%%%%%%%%%%%%%%%%%%%%%%%%
\section{Peltier cooling unit}
Here the setup of the peltier-cooling unit is described. It was used for cooled liquid storage of borazine. 
\end{document}