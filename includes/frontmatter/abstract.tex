\chapter{\abstractname}
Good quality, two dimensional, hexagonal boron nitride (\textit{h}-BN) islands require a clean and flat surface to grow on. Techniques to chemically polish the surface are shortly reviewed. After polish the growth of sub-monolayer \textit{h}-BN islands is investigated on poly crystalline copper surfaces by means of STM and XPS. The use of \textit{h}-BN grown on polycrystalline copper foils as insulating and electronically decoupling substrate is shown by reproducing molecular adsorption known on \textit{h}-BN grown on single crystalline copper.

Different molecular species are investigated with regard to their electronic properties and structures formed by self-assembly on metallic and insulating 2D \textit{h}-BN substrates.

Bis- \& Tetra-pyridin-4-ylethynyl functionalized pyrene molcules are adsorped on \textit{h}-BN/Cu(111) to show the diversity of self-assembled molecules structures that can be steered by the number and position of functional groups. It is shown that their opto-electronic properties and assembly after adsorption are well determined by the chemical design of the molecule and show the same trend as gas-phase calculations: An decrease in electronic band gap with increasing number of functional groups. Frontier orbital resolution in STM with modified tip conditions and a wide band gap in STS show efficient decoupling from the metallic substrate by the \textit{h}-BN layer. It modifies the substrate surface potential, so that the band gap is changed locally in a well defined, periodic manner, following the periodicity of \textit{h}-BN's superstructure on Cu(111).

Furthermore coronene molecules are used to determine the influence of a BN modified molecular center on the self-assembly and electronic structure on Ag(111) and Au(111) substrates and shows the importance of the side groups in the formation process of self-assembled structures. Investigation with nc-AFM is used to clarify sub-molecular structure and the formation of linked structures after annealing treatment in UHV.

Single and bis- nitro functionalized porphine molecules are adsorbed on Cu, Ag and \textit{h}-BN/Cu(111). Additional di-tert-butyl-phenyl side groups are used to further decouple the molecule from the substrate layer and increase mobility at low temperatures. The molecular self-assembly is controlled by the number of functional groups, so that bis-functionalized molecules adsorbed on Ag(100) form superstructures with hexagonal symmetry, mismatching its 2-fold symmetric substrate symmetry.
	
Helicene molecules are used to investigate the influence of chiral properties in conjunction with a dipole moment introduced by functionalization with two cyano groups at the helicene's central carbon ring. Depending on the substrate, molecular assembly varies from chains formed with specific orientation to the metal substrate's high symmetry directions to dense packed islands formed after adsorption on \textit{h}-BN/Cu(111). Annealing after adsorption on Ag resulted in a ring-closure reaction at the helicene's spiral terminations that lifts chirality.

At last a design for a peltier cooling unit is given, which is used to store liquids with a volume of serveral \SI{}{\milli \liter} at temperatures around \SI{0}{\celsius}.