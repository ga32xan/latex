\textcolor{red}{\textbf{
		The driving force for orienting the whole molecule on the surface remains speculative. On Ag(100), neither an orientation of the molecules main axis with respect to the substrate, nor a orientation of butyl-groups along the dense packed substrate rows can be seen - which again favors Cu-substrate interactions as dominant role.
When the copper is exchanged with silver to act as substrate, TBP behaves quite different. Although the distribution is homogeneous on the surface, the interaction between molecules look different. While on copper the most abundant binding motif is the head-to-head dimer, this motif does not appear on silver as often as on copper. Two other motifs emerge on silver.
The interaction between the butyl-phenyl groups is considered to be van der Waals like \cite{iacovita_controlling_2012}, stabilizing the conglomerate.
}}