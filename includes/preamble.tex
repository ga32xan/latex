\usepackage[utf8]{inputenc}
% when latex complains about unicode char U+2212 is not configured for use in latex use the line below
\DeclareUnicodeCharacter{2212}{-}% support older LaTeX versions
%\usepackage[latin1]{inputenc}
\usepackage[T1]{fontenc}
\usepackage{lmodern}

\usepackage[english]{babel}
\usepackage{csquotes}
\usepackage{amsmath}
\usepackage{textcomp}		%enables \textdegree to use as �
\usepackage{amsfonts}
\usepackage{amssymb}
\usepackage{graphicx}
%\usepackage{wasysym}		
\usepackage{braket}		% fur <A|H|B> <A| |A> oder <A>
%\DeclareGraphicsExtensions{.pdf,.png,.jpg}
\usepackage{siunitx}
\usepackage[hidelinks,breaklinks=true]{hyperref}
%\usepackage{url}
\usepackage[section]{placeins} %definiert \floatbarrier, mit option automatisch bei jeder section
%%%%%%%%%%%%%%%%%%%%%%%%%%%%%%%%%%%%%%%%%%%
% F�r subfigure Umgebung
\usepackage{subfigure}
%%%%%%%%%%%%%%%%%%%%%%%%%%%%%%%%%%%%%%%%%%%
\usepackage{caption}
\usepackage{microtype}
%\usepackage{subcaption}
\usepackage{multicol}
\usepackage{multirow}
%%%%%%%%%%%%%%%%%%%%%%%%%%%%
% fuer Stichwortverzeichnis
\usepackage{makeidx}

% Stichwortverzeichnis erstellen
\makeindex

%%%%%%%%%%%%%%%%%%%%%%%%%%%%%%%%%%%%%%%%%%%%%%%%%%%%%%%
\usepackage[style=numeric,backend=biber,refsection=chapter]{biblatex} 	%kommentieren f�r use of BibTeX
% bibliogryphy-styles: alphabetic, numeric, chem-angew, ieee, nature, science
\addbibresource{./bib.bib}  % relative to root directory (where the file that includes this file is located)! do NOT OMIT .bib ending
%\bibliography{./bib}		% outdated
%avoids ugly line breaks within bibligraphy
\addto\bibsetup{\setlength{\emergencystretch}{1.5em}} 	
% Zum Verwalten der Zitate benutze ich Zotero, zum Erzeugen der .bib-Datein f�r Latex wird die Exportfunktion von Zotero benutzt (rechtsklick auf ``Meine Bibliothek'' im linken Reiter: Option Biblatex in die Datei bib-zotero-export.bib aus welcher ich dann die betreffenden Zitate auf Richtigkeit \"uberpr\"ufe und in die bib.bib kopiere.
%
% \printbibliography nach jedem chapter und option biblatex [refsection=chapter] 
% legt bibliography nach jedem Chapter an
%
%%%%%%%%%%%%%%%%%%%%%%%%%%%%%%%%%%%%%%%%%%%%%%%%%%%%%%%
\author{Domenik Matthias Zimmermann}
\title{Molecular functionalisation of h-BN}

%%%%%%%%%%%%%%%%%%%%%%%%%%%%%%%%%%%%%%%%%%%%%%%%%%%%%%%
\usepackage[top=2.5cm,left=2.5cm,right=3.5cm,bottom=3.5cm]{geometry}

%%%%%%%%%%%%\usepackage{fancyhdr}
\usepackage{xcolor}
% \usepackage{titlesec}
%Fu�zeile
% \renewcommand{\footrulewidth}{1pt}
% \lfoot[]{\thepage} \rfoot[\thepage]{}

% Kopfnote
% \setlength{\headheight}{15.2pt}
%\renewcommand{\headrulewidth}{1pt}
% \renewcommand{\sectionmark}[1]{\markright{#1}{}} \lhead[]{\leftmark} \rhead[\leftmark]{}

% Section �berschriften gestalten
% \definecolor{gray75}{gray}{0.75}
% \newcommand{\hsp}{\hspace{20pt}}
% \titleformat{\section}[hang]{\huge\bfseries}{\thesection\hsp\textcolor{gray75}{|}\hsp}{0pt}{\huge\bfseries}
%%%%%%%%%%%%%%%%%%%%%%%%%%%%%%%%%%%%%%%%%%%%%%%%%%%%%%
\usepackage{wrapfig}
